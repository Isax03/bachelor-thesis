\section{Sviluppo}
\label{sec:development}

La fase di sviluppo del software rappresenta il cuore del ciclo di vita del software.
In questa sezione vengono analizzate le best practices di scrittura del codice
consigliate da enti specializzati come OWASP e SEI CERT, che si occupano di sicurezza
e qualità del software.

Viene inoltre esaminata l'importanza di utilizzare librerie che implementano funzioni
e metodi per rafforzare la memory safety anche in linguaggi considerati generalmente
non memory safe, come C e C++.

Infine, viene trattata l'analisi statica del codice, una tecnica che permette di
rilevare vulnerabilità e problemi di qualità del codice senza la necessità di
eseguire il programma.

\subsection{Best Practices nel Codice}
\label{sec:best-practices-codice}

Il primo passo per garantire la \textit{memory safety} è rappresentato dalle
scelte effettuate durante la scrittura del codice. Anche utilizzando linguaggi moderni
e progettati con meccanismi di sicurezza intrinseci, la qualità del software
dipende inevitabilmente dalle decisioni adottate dallo sviluppatore. Per questo motivo
esistono linee guida consolidate, note come \textit{secure coding practices},
che aiutano a prevenire vulnerabilità legate alla gestione della memoria, come buffer
overflow, dangling pointer o memory leak. L'applicazione di queste best practices
consente di ridurre drasticamente il rischio di exploit e aumentare la robustezza
del software, fin dalle sue fondamenta.

Una delle principali fonti di riferimento è la \textbf{OWASP Secure Coding
Practices Quick Reference Guide}\cite{owasp_best_practices}, che fornisce una checklist
focalizzata su vari aspetti della sicurezza del software, tra cui la gestione
della memoria.

Di seguito si riportano alcune delle principali best practices relative alla
sicurezza della memoria:

\begin{itemize}
  \item \textbf{Validare input e output da fonti non affidabili}: tutti i dati esterni
    devono essere sottoposti a controlli per evitare che valori malformati causino
    overflow o corruzione di memoria.

  \item \textbf{Verificare le dimensioni dei buffer}: prima di accedere o scrivere
    su un buffer, è necessario assicurarsi che sia sufficientemente grande da
    contenere i dati previsti.

  \item \textbf{Gestire correttamente la terminazione delle stringhe}: quando si
    usano funzioni che accettano una lunghezza in byte, bisogna garantire la
    presenza del terminatore \texttt{NULL}.

  \item \textbf{Controllare i limiti dei buffer nei cicli}: è importante assicurarsi
    che ogni iterazione non superi i limiti di memoria del buffer.

  \item \textbf{Troncare le stringhe in input}: limitare la lunghezza delle stringhe
    in input prima di passarle ad altre funzioni riduce i rischi di overflow.

  \item \textbf{Chiudere esplicitamente le risorse}: è buona pratica liberare esplicitamente
    memoria, file e altri handle di sistema, senza fare affidamento sul garbage collector
    (se presente).

  \item \textbf{Evitare l'uso di funzioni note come vulnerabili}: funzioni standard
    come \texttt{gets}, \texttt{strcpy} e \texttt{sprintf} sono intrinsecamente pericolose
    e dovrebbero essere sostituite con alternative più sicure.

  \item \textbf{Liberare correttamente la memoria}: la memoria dinamica va liberata
    in tutti i punti di uscita del programma, compresi quelli dovuti a
    condizioni di errore.

  \item \textbf{Sovrascrivere i dati sensibili prima della deallocazione}: quando
    si gestiscono dati come password o chiavi crittografiche, è consigliabile
    sovrascrivere la memoria per evitare che rimangano accessibili in chiaro.

  \item \textbf{Liberare correttamente la memoria anche in caso di errore}: la memoria
    allocata dinamicamente deve essere sempre liberata, anche nei percorsi di
    esecuzione alternativi dovuti a condizioni di errore, per evitare memory leak
    persistenti.
\end{itemize}

Oltre alle regole già formalizzate, risulta opportuno considerare anche altre
buone pratiche non sempre trattate esplicitamente, ma che possono contribuire a prevenire
i bug descritti nella \autoref{sec:vulnerability_types}. Un esempio
significativo è il controllo degli \textbf{integer overflow} durante il calcolo degli
indici di un array: un valore che supera il massimo rappresentabile da un intero
può produrre un indice errato, portando a un accesso fuori dai limiti della memoria.

A complemento della guida proposta da \textit{OWASP}, i \textbf{SEI CERT Coding
Standards}\cite{cert_coding_standard} offrono una raccolta di regole dettagliate,
spesso accompagnate da esempi di codice non conforme (\textit{non-compliant}) e dalle
relative soluzioni corrette (\textit{compliant}). Tuttavia, le regole di SEI
CERT sono organizzate in base al linguaggio di programmazione e risultano molto
più numerose rispetto a quelle di OWASP, rendendo difficile presentarle tutte in
questo documento. In generale, oltre alle regole già menzionate di natura più generica,
è consigliabile approfondire le regole specifiche per il linguaggio utilizzato.

Queste pratiche, se correttamente implementate, costituiscono un primo livello
di difesa essenziale nella costruzione di software robusto e sicuro, specialmente
in linguaggi notoriamente \textit{memory unsafe} come C e C++.

\subsection{Librerie}
\label{sec:librerie}

Per migliorare la memory safety nei progetti software, risulta spesso
vantaggioso affidarsi a librerie progettate per offrire un livello di controllo
e protezione superiore rispetto a quello garantito dalle funzioni standard del
linguaggio. Questo aspetto è particolarmente rilevante nei linguaggi storicamente
non memory safe, come C e C++, dove l'allocazione manuale e la gestione esplicita
della memoria espongono facilmente a vulnerabilità come buffer overflow, use-after-free
o double free.

Nel corso degli anni sono state sviluppate numerose librerie che affrontano
direttamente queste problematiche. Alcune di esse, come \verb|safe-malloc| o
\verb|xmalloc|, fungono da wrapper per le funzioni di allocazione (malloc, free,
realloc), implementando controlli interni per verificare la correttezza delle
operazioni e rilevare anomalie durante l'utilizzo della heap. Altre, come
\verb|safestringlib|, sostituiscono le funzioni standard per la manipolazione delle
stringhe con versioni più sicure (strcpy\_s, memset\_s, ecc.), che includono verifiche
esplicite sulla lunghezza dei buffer per evitare scritture fuori dai limiti. La
libreria \verb|Sailfish Pool|, invece, introduce tecniche come la preallocazione
di pool di memoria e la zeroizzazione dei dati prima della deallocazione, particolarmente
utili in contesti in cui la privacy e la protezione dei dati sono fondamentali.

La necessità di rafforzare la sicurezza della memoria non riguarda esclusivamente
i linguaggi più esposti. Anche ambienti più moderni e progettati per prevenire
questi errori, come Rust, offrono librerie che fungono da meccanismi di
enforcement aggiuntivi. Un esempio significativo è rappresentato da zeroize, una
libreria che consente di azzerare in modo automatico e sicuro i contenuti di una
variabile prima che venga rilasciata o vada fuori dallo scope, riducendo il rischio
che dati sensibili permangano in memoria dopo l'uso. Questo caso dimostra come
la memory safety costituisca un obiettivo continuo, che può trarre beneficio,
anche in contesti intrinsecamente più sicuri, dall'utilizzo di strumenti dedicati
a casi d'uso specifici.

Un ulteriore aspetto rilevante da considerare nella selezione delle librerie è la
\textit{trasparenza del codice}. L'adozione di librerie open source consente agli
sviluppatori di ispezionare direttamente l'implementazione, incrementando la
fiducia nei meccanismi adottati e offrendo la possibilità di verificarne il
comportamento effettivo. Questo costituisce un vantaggio concreto rispetto a librerie
closed source o scarsamente documentate, in quanto riduce il rischio di
introdurre codice opaco o potenzialmente pericoloso all'interno del progetto.

In conclusione, l'utilizzo consapevole di librerie progettate per migliorare la memory
safety rappresenta una buona pratica, sia nei linguaggi tradizionalmente
vulnerabili sia in quelli più moderni. L'integrazione di questi strumenti nel flusso
di sviluppo consente di ridurre il margine di errore umano e incrementare la resilienza
del software contro le vulnerabilità più comuni.

\subsection{Analisi Statica}
\label{sec:analisi-statica}