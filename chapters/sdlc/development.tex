\section{Sviluppo}
\label{sec:development}

La fase di sviluppo del software rappresenta il cuore del ciclo di vita del software.
In questa sezione vengono analizzate le best practice di scrittura del codice
consigliate da enti specializzati come OWASP e SEI CERT, che si occupano di sicurezza
e qualità del software.

Viene inoltre esaminata l'importanza di utilizzare librerie che implementano funzioni
e metodi per rafforzare la memory safety anche in linguaggi considerati generalmente
non memory safe.

Infine, viene trattata l'analisi statica del codice, una tecnica che permette di
rilevare vulnerabilità e problemi di qualità del codice senza la necessità di eseguire
il programma.

\subsection{Best practice nel codice}
\label{sec:best-practices-codice}

La creazione di software con elevati standard di memory safety inizia dalle
decisioni prese durante la fase di implementazione del codice. Per quanto si possano
utilizzare linguaggi moderni e progettati con meccanismi di sicurezza intrinseci,
la qualità del software dipende inevitabilmente dalle decisioni adottate dallo
sviluppatore. Per questo motivo esistono linee guida consolidate, che aiutano a
prevenire vulnerabilità legate alla gestione della memoria. L'applicazione di queste
buone pratiche consente di ridurre drasticamente il rischio di exploit e aumentare
la robustezza del software fin dalle sue fondamenta, costituendo un primo
livello di difesa essenziale, specialmente in linguaggi notoriamente non memory safe.

Una delle principali fonti di riferimento è la \textbf{OWASP Secure Coding
Practices Quick Reference Guide}~\cite{owasp_best_practices}, che fornisce una checklist
focalizzata su vari aspetti della sicurezza del software, tra cui la gestione
della memoria.

Di seguito si riportano alcune delle principali best practice relative alla
sicurezza della memoria:

\begin{itemize}
  \item \textbf{Validare input e output da fonti non affidabili}: tutti i dati esterni
    devono essere sottoposti a controlli per evitare che valori malformati causino
    overflow o corruzione di memoria.

  \item \textbf{Verificare le dimensioni dei buffer}: prima di accedere o scrivere
    su un buffer, è necessario assicurarsi che sia sufficientemente grande da
    contenere i dati previsti.

  \item \textbf{Assicurare la corretta terminazione delle stringhe}: durante l'utilizzo
    di funzioni che richiedono una dimensione in byte, è essenziale garantire
    che il carattere terminatore \texttt{NULL} sia sempre presente e correttamente
    posizionato.

  \item \textbf{Controllare i limiti dei buffer nei cicli}: è importante assicurarsi
    che ogni iterazione non superi i limiti di memoria del buffer.

  \item \textbf{Troncare le stringhe in input}: limitare la lunghezza delle stringhe
    in input prima di passarle ad altre funzioni riduce i rischi di overflow.

  \item \textbf{Chiudere esplicitamente le risorse}: è buona pratica liberare esplicitamente
    memoria, file e altri handle di sistema, senza fare affidamento sul garbage collector
    (se presente).

  \item \textbf{Evitare l'uso di funzioni note come vulnerabili}: funzioni standard
    come \texttt{gets}, \texttt{strcpy} e \texttt{sprintf} sono intrinsecamente pericolose
    e dovrebbero essere sostituite con alternative più sicure (a tal proposito
    si veda la~\autoref{sec:librerie} nel seguito).

  \item \textbf{Liberare correttamente la memoria}: la memoria dinamica va liberata
    in tutti i punti di uscita del programma, compresi quelli dovuti a
    condizioni di errore (evita memory leak).

  \item \textbf{Sovrascrivere i dati sensibili prima della deallocazione}: quando
    si gestiscono dati come password o chiavi crittografiche, è consigliabile
    sovrascrivere la memoria per evitare che rimangano accessibili in chiaro (zeroization).
\end{itemize}

Oltre alle regole già formalizzate, risulta opportuno considerare anche altre
buone pratiche non sempre trattate esplicitamente, ma che possono contribuire a prevenire
i bug descritti nella~\autoref{sec:vulnerability_types}. Un esempio
significativo è il controllo degli \textbf{integer overflow} durante il calcolo degli
indici di un array: un valore che supera il massimo rappresentabile da un intero
può produrre un indice errato, portando a un accesso fuori dai limiti della memoria.

A complemento della guida proposta da \textit{OWASP}, i \textbf{SEI CERT Coding
Standard}~\cite{cert_coding_standard} offrono una raccolta di regole dettagliate,
spesso accompagnate da esempi di codice non conforme (\textit{non-compliant}) e
dalle relative soluzioni corrette (\textit{compliant}). Le regole di SEI CERT
sono organizzate in base al linguaggio di programmazione e risultano molto più
numerose rispetto a quelle di OWASP, rendendo difficile presentarle tutte in questo
documento. Tuttavia, per fornire un'idea della loro struttura e approccio, viene
riportato nell'\autoref{appendix:sei_cert} un esempio di regola SEI CERT
relativa alla gestione sicura degli array in C.

\subsection{Librerie}
\label{sec:librerie}

Per migliorare la memory safety nei progetti software, risulta spesso
vantaggioso affidarsi a librerie progettate per offrire un livello di controllo
e protezione superiore rispetto a quello garantito dalle funzioni standard del
linguaggio. Questo aspetto è particolarmente rilevante nei linguaggi storicamente
non memory safe, come C e C++, dove l'allocazione manuale e la gestione esplicita
della memoria espongono facilmente a vulnerabilità come buffer overflow, use
after free o double free.

Nel corso degli anni sono state sviluppate numerose librerie che affrontano
direttamente queste problematiche. Alcune di esse, come \textit{xmalloc}\footnote{Repository
GitHub: \url{https://github.com/rosingh/xmalloc}}, fungono da wrapper per le funzioni
di allocazione (\texttt{malloc()}, \texttt{free()}, \texttt{realloc()}), implementando
controlli interni per verificare la correttezza delle operazioni e rilevare anomalie
durante l'utilizzo della heap. Altre, come \textit{safestringlib}\footnote{Repository
GitHub: \url{https://github.com/intel/safestringlib}}, sostituiscono le funzioni
standard per la manipolazione delle stringhe con versioni più sicure (\texttt{strcpy\_s},
\texttt{memset\_s}, etc.), che includono verifiche esplicite sulla lunghezza dei
buffer per evitare scritture fuori dai limiti. La libreria \textit{Sailfish Pool}\footnote{Repository
GitHub: \url{https://github.com/dyne/sailfish-pool}} invece, introduce tecniche
come la preallocazione di pool di memoria e la zeroization dei dati prima della
deallocazione, particolarmente utili in contesti in cui la privacy e la protezione
dei dati sono essenziali.

La necessità di rafforzare la sicurezza della memoria non riguarda esclusivamente
i linguaggi più esposti: anche ambienti più moderni e progettati per prevenire
questi errori, come Rust, offrono librerie che fungono da meccanismi di
enforcement aggiuntivi. Un esempio significativo è rappresentato da \texttt{zeroize}\protect\footnote{Documentazione:
\url{https://docs.rs/zeroize/latest/zeroize/}}, una libreria che consente di azzerare
in modo automatico e sicuro i contenuti di una variabile prima che venga rilasciata
o vada fuori dallo scope, riducendo il rischio che dati sensibili permangano in
memoria dopo l'uso.

Un ulteriore aspetto rilevante da considerare nella selezione delle librerie è
la \textit{trasparenza del codice}. L'adozione di librerie open source consente
agli sviluppatori di ispezionare direttamente l'implementazione, aumentando la fiducia
nei meccanismi adottati e offrendo la possibilità di verificarne il comportamento
effettivo. Questo costituisce un vantaggio concreto rispetto a librerie closed
source o scarsamente documentate, in quanto riduce il rischio di introdurre codice
opaco o potenzialmente pericoloso all'interno del progetto.

In conclusione, l'utilizzo consapevole di librerie progettate per migliorare la
memory safety rappresenta una buona pratica, sia nei linguaggi tradizionalmente vulnerabili
sia in quelli più moderni.

\subsection{Analisi statica}
\label{sec:analisi-statica}

Dopo aver trattato le tecniche per scrivere codice sicuro, è utile introdurre un'ulteriore
forma di mitigazione: l'\textit{analisi statica}. A prima vista, questa potrebbe
sembrare una tecnica più affine alla fase di \textit{testing}, poiché si occupa di
rilevare errori e vulnerabilità. Tuttavia, l'evoluzione degli strumenti di sviluppo
e l'affermazione del principio di \textit{shift-left} hanno portato l'analisi
statica a essere adottata sempre più precocemente nel ciclo di vita del software,
al punto da diventare parte integrante della fase di sviluppo.

Con ``shift-left'' si intende una strategia che promuove l'anticipo delle attività
di verifica e validazione, allo scopo di intercettare i problemi il prima
possibile. In quest'ottica, l'analisi statica è oggi spesso integrata
direttamente negli ambienti di sviluppo (IDE) e nei sistemi di Continuous
Integration/Continuous Deployment (CI/CD), fornendo un feedback immediato agli sviluppatori
durante la scrittura del codice stesso.

L'analisi statica consiste nell'esaminare il codice sorgente (o, in alcuni casi,
il bytecode o il codice compilato) \textbf{senza eseguirlo}, alla ricerca di difetti
e problemi che spaziano dalle vulnerabilità legate alla memory safety fino ad altri
tipi di errori come divisioni per zero, codice morto o violazioni delle
convenzioni di stile. Il vantaggio principale è la possibilità di individuare problemi
in una fase precoce dello sviluppo, riducendo così i costi e la complessità
delle successive correzioni.

Nonostante la sua capacità di rilevazione precoce degli errori, l'analisi statica
presenta alcune limitazioni: una delle sfide principali è la gestione dei
\textbf{falsi positivi}, ovvero segnalazioni di problemi che in realtà non
costituiscono errori o vulnerabilità nel contesto specifico dell'applicazione. Questo
è dovuto al fatto che gli analizzatori non riescono sempre a comprendere il contesto
completo in cui il codice viene eseguito, portando a segnalazioni errate.

Un esempio pratico può essere quello del~\autoref{lst:copy_string}: in questo caso,
alcuni analizzatori statici, come Frama-C, Clang o cppcheck, approfonditi nella~\autoref{sec:initial_analysis},
potrebbero segnalare un potenziale buffer overflow, nonostante il controllo esplicito
della lunghezza della stringa di input. Questa situazione si verifica, ad esempio,
quando l'analizzatore non tiene conto del legame semantico tra la condizione
\texttt{if} e l'istruzione \texttt{strcpy}. Altri tool invece potrebbero adottare
strategie conservative che segnalano sempre funzioni come \texttt{strcpy} come
potenzialmente pericolose.

Allo stesso modo, l'analisi statica può anche portare a \textbf{falsi negativi},
ossia la mancata rilevazione di problemi reali, specialmente per vulnerabilità
complesse che dipendono da configurazioni di runtime o interazioni specifiche
con input esterni non prevedibili staticamente.

In sintesi, integrare l'analisi statica nel processo di sviluppo è una pratica altamente
raccomandata, che può rilevare dimenticanze o falle introdotte dallo
sviluppatore e dalle librerie usate, contribuendo al rafforzamento della sicurezza.
Tuttavia, è fondamentale considerare i suoi limiti e continuare a migliorare il software
anche nelle successive fasi del ciclo di vita, ovvero durante il testing, la distribuzione
e la manutenzione.

\bigskip
\begin{lstlisting}[language=C, caption={Copia sicura di una stringa con \texttt{strcpy()}}, label={lst:copy_string}]
#include <stdlib.h>
#include <string.h>
void copy_string(const char *input) {
  char buffer[64];
  if (strlen(input) < sizeof(buffer)){
    strcpy(buffer, input); // Sicuro: controllato prima
  }
}
\end{lstlisting}