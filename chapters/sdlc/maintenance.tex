\section{Mantenimento}
\label{sec:maintenance}

La fase di mantenimento rappresenta l'ultima, ma non meno rilevante, tappa del ciclo
di vita del software. In questo stadio, l'obiettivo principale è assicurare che il
sistema continui a funzionare in modo corretto e sicuro nel tempo, adattandosi a
nuove esigenze, aggiornamenti tecnologici e, soprattutto, alla scoperta di vulnerabilità
non precedentemente emerse.

Le pratiche di mantenimento qui descritte hanno carattere più generale rispetto alle
precedenti sezioni: pur essendo fondamentali per preservare la memory safety, si
applicano trasversalmente a tutti gli aspetti della sicurezza del software.
Poiché gran parte delle misure specifiche per la memory safety vengono definite
e applicate nelle fasi precedenti, il mantenimento si concentra su attività di supporto
e verifica continua. Queste pratiche, come il tracciamento delle dipendenze o il
monitoraggio dei comportamenti a runtime, permettono di preservare nel tempo l'efficacia
delle scelte già adottate, contribuendo alla sicurezza complessiva del sistema.

\subsection{Tracciamento delle Dipendenze}
\label{sec:tracciamento-dipendenze}

Un aspetto critico per la sicurezza della memoria nel lungo periodo è la
gestione delle dipendenze software, specialmente quando si utilizzano librerie esterne,
spesso sviluppate in linguaggi memory-unsafe come C e C++.

I sistemi moderni, a tal proposito, possono contare su strumenti di dependency
tracking automatizzati, in grado di identificare vulnerabilità note tramite l'integrazione
con database CVE, notificarne la presenza di versioni obsolete o insicure e
suggerire aggiornamenti sicuri o alternative memory-safe.

Framework come OWASP Dependency-Check\footnote{\url{https://owasp.org/www-project-dependency-check/}}
o strumenti integrati nei sistemi di CI/CD (es. GitHub Dependabot) automatizzano
parte di questo processo, facilitando la gestione reattiva delle vulnerabilità.
In contesti critici è inoltre consigliabile:
\begin{itemize}
  \item ridurre il numero di dipendenze esterne inutili (minimizzazione della
    superficie di attacco);

  \item preferire librerie attivamente manutenute e con audit pubblici di sicurezza;

  \item mantenere un \textit{Software Bill of Materials} (SBOM), utile sia per la
    verifica continua di vulnerabilità in fase di mantenimento, che nella
    successiva iterazione del SDLC in fase di pianificazione, per valutare l'impatto
    della sostituzione o dell'aggiornamento di una libreria.
\end{itemize}

In linguaggi come Rust, che enfatizzano la sicurezza di tipo e l'ownership, è
comunque fondamentale monitorare i crate esterni, specialmente quelli che usano blocchi
\texttt{unsafe} o effettuano Foreign Function Interface (FFI), ovvero l'invocazione
di funzioni scritte in altri linguaggi, verso codice legacy. Il tracciamento deve
essere quindi parte integrante del workflow di mantenimento.

\subsection{Monitoraggio e Logging}
\label{sec:monitoraggio-logging}

Oltre al tracciamento delle dipendenze, una strategia efficace di mantenimento prevede
l'adozione di sistemi di monitoraggio e logging, con l'obiettivo di individuare
comportamenti anomali potenzialmente legati a vulnerabilità di memoria.

In particolare, i meccanismi di logging possono aiutare a:
\begin{itemize}
  \item rilevare accessi fuori dai limiti o malfunzionamenti ripetuti;

  \item diagnosticare memory leak persistenti;

  \item correlare anomalie a specifici aggiornamenti di codice o librerie.
\end{itemize}

In sintesi, adottare tecniche di osservabilità avanzata permette di fare
manutenzione in modo più proattivo, prevenendo problemi prima che diventino gravi
e rendendo il sistema più robusto.