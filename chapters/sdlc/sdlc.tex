\chapter{Memory Safety nel SDLC}
\label{cha:sdlc}

Dopo aver introdotto i concetti fondamentali legati alla memory safety e aver
visto l'impatto che una gestione errata della memoria può avere sulla sicurezza,
sull'affidabilità e sulla stabilità del software, ci concentriamo ora su come
questi principi possano essere concretamente integrati all'interno del ciclo di
vita dello sviluppo del software \textit{(Software Development Life Cycle, SDLC)}.

La memory safety non è un obiettivo che può essere raggiunto in un singolo
momento del processo di sviluppo, ma deve essere perseguito lungo tutte le sue fasi.
Una falla memory-unsafe, infatti, può essere introdotta in qualsiasi momento, in
maniera più o meno diretta: dal linguaggio di programmazione utilizzato, passando
per la qualità del codice, fino ad arrivare all'ambiente di esecuzione. Per questo
motivo, ogni fase dello SDLC rappresenta un'opportunità per prevenire, individuare
e mitigare vulnerabilità legate all'uso insicuro della memoria. Comprendere il
contributo specifico di ciascuna fase alla sicurezza della memoria consente ai
team di sviluppo di adottare una strategia integrata e preventiva, riducendo significativamente
il rischio di introdurre vulnerabilità critiche nel sistema.

Va sottolineato fin da subito che le tecniche, gli strumenti e i linguaggi eventualmente
citati \textit{non} vanno intesi come universali o necessariamente ottimali per
ogni contesto. L'obiettivo di questo capitolo non è infatti quello di raccomandare
soluzioni specifiche, ma piuttosto fornire un \textit{quadro strutturato di
buone pratiche e approcci} che possono essere adattati in base ai requisiti del progetto,
alle risorse disponibili e al contesto applicativo.

\section{Fasi preliminari}
\label{sec:preliminary_stages}

\subsection{Scelta del Linguaggio}
\label{sec:linguaggio}
\section{Sviluppo}
\label{sec:development}

La fase di sviluppo del software rappresenta il cuore del ciclo di vita del software.
In questa sezione vengono analizzate le best practice di scrittura del codice
consigliate da enti specializzati come OWASP e SEI CERT, che si occupano di sicurezza
e qualità del software.

Viene inoltre esaminata l'importanza di utilizzare librerie che implementano funzioni
e metodi per rafforzare la memory safety anche in linguaggi considerati generalmente
non memory safe.

Infine, viene trattata l'analisi statica del codice, una tecnica che permette di
rilevare vulnerabilità e problemi di qualità del codice senza la necessità di eseguire
il programma.

\subsection{Best Practice nel Codice}
\label{sec:best-practices-codice}

La creazione di software con elevati standard di memory safety inizia dalle
decisioni prese durante la fase di implementazione del codice. Per quanto si possano
utilizzare linguaggi moderni e progettati con meccanismi di sicurezza intrinseci,
la qualità del software dipende inevitabilmente dalle decisioni adottate dallo
sviluppatore. Per questo motivo esistono linee guida consolidate, note come
\textit{secure coding practice}, che aiutano a prevenire vulnerabilità legate
alla gestione della memoria. L'applicazione di queste buone pratiche consente di
ridurre drasticamente il rischio di exploit e aumentare la robustezza del software
fin dalle sue fondamenta, costituendo un primo livello di difesa essenziale, specialmente
in linguaggi notoriamente non memory safe.

Una delle principali fonti di riferimento è la \textbf{OWASP Secure Coding
Practices Quick Reference Guide}\cite{owasp_best_practices}, che fornisce una
checklist focalizzata su vari aspetti della sicurezza del software, tra cui la gestione
della memoria.

Di seguito si riportano alcune delle principali best practice relative alla sicurezza
della memoria:

\begin{itemize}
  \item \textbf{Validare input e output da fonti non affidabili}: tutti i dati esterni
    devono essere sottoposti a controlli per evitare che valori malformati causino
    overflow o corruzione di memoria.

  \item \textbf{Verificare le dimensioni dei buffer}: prima di accedere o scrivere
    su un buffer, è necessario assicurarsi che sia sufficientemente grande da
    contenere i dati previsti.

  \item \textbf{Assicurare la corretta terminazione delle stringhe}: durante l'utilizzo
    di funzioni che richiedono una dimensione in byte, è essenziale garantire
    che il carattere terminatore \texttt{NULL} sia sempre presente e correttamente
    posizionato.

  \item \textbf{Controllare i limiti dei buffer nei cicli}: è importante assicurarsi
    che ogni iterazione non superi i limiti di memoria del buffer.

  \item \textbf{Troncare le stringhe in input}: limitare la lunghezza delle stringhe
    in input prima di passarle ad altre funzioni riduce i rischi di overflow.

  \item \textbf{Chiudere esplicitamente le risorse}: è buona pratica liberare esplicitamente
    memoria, file e altri handle di sistema, senza fare affidamento sul garbage collector
    (se presente).

  \item \textbf{Evitare l'uso di funzioni note come vulnerabili}: funzioni standard
    come \texttt{gets}, \texttt{strcpy} e \texttt{sprintf} sono intrinsecamente pericolose
    e dovrebbero essere sostituite con alternative più sicure.

  \item \textbf{Liberare correttamente la memoria}: la memoria dinamica va liberata
    in tutti i punti di uscita del programma, compresi quelli dovuti a
    condizioni di errore (evita memory leak).

  \item \textbf{Sovrascrivere i dati sensibili prima della deallocazione}: quando
    si gestiscono dati come password o chiavi crittografiche, è consigliabile
    sovrascrivere la memoria per evitare che rimangano accessibili in chiaro (zeroization).
\end{itemize}

Oltre alle regole già formalizzate, risulta opportuno considerare anche altre buone
pratiche non sempre trattate esplicitamente, ma che possono contribuire a
prevenire i bug descritti nella~\autoref{sec:vulnerability_types}. Un esempio significativo
è il controllo degli \textbf{integer overflow} durante il calcolo degli indici
di un array: un valore che supera il massimo rappresentabile da un intero può produrre
un indice errato, portando a un accesso fuori dai limiti della memoria.

A complemento della guida proposta da \textit{OWASP}, i \textbf{SEI CERT Coding
Standard}\cite{cert_coding_standard} offrono una raccolta di regole dettagliate,
spesso accompagnate da esempi di codice non conforme (\textit{non-compliant}) e
dalle relative soluzioni corrette (\textit{compliant}). Le regole di SEI CERT
sono organizzate in base al linguaggio di programmazione e risultano molto più
numerose rispetto a quelle di OWASP, rendendo difficile presentarle tutte in questo
documento. Tuttavia, per fornire un'idea della loro struttura e approccio, viene
riportato nell'\autoref{appendix:sei_cert} un esempio di regola SEI CERT
relativa alla gestione sicura degli array in C.

\subsection{Librerie}
\label{sec:librerie}

Per migliorare la memory safety nei progetti software, risulta spesso
vantaggioso affidarsi a librerie progettate per offrire un livello di controllo
e protezione superiore rispetto a quello garantito dalle funzioni standard del
linguaggio. Questo aspetto è particolarmente rilevante nei linguaggi storicamente
non memory safe, come C e C++, dove l'allocazione manuale e la gestione esplicita
della memoria espongono facilmente a vulnerabilità come buffer overflow, use-after-free
o double free.

Nel corso degli anni sono state sviluppate numerose librerie che affrontano
direttamente queste problematiche. Alcune di esse, come \textit{xmalloc}\footnote{Repository
GitHub: \url{https://github.com/rosingh/xmalloc}}, fungono da wrapper per le funzioni
di allocazione (\texttt{malloc()}, \texttt{free()}, \texttt{realloc()}), implementando
controlli interni per verificare la correttezza delle operazioni e rilevare anomalie
durante l'utilizzo della heap. Altre, come \textit{safestringlib}\footnote{Repository
GitHub: \url{https://github.com/intel/safestringlib}}, sostituiscono le funzioni
standard per la manipolazione delle stringhe con versioni più sicure (\texttt{strcpy\_s},
\texttt{memset\_s}, etc.), che includono verifiche esplicite sulla lunghezza dei
buffer per evitare scritture fuori dai limiti. La libreria \textit{Sailfish Pool}\footnote{Repository
GitHub: \url{https://github.com/dyne/sailfish-pool}} invece, introduce tecniche
come la preallocazione di pool di memoria e la zeroization dei dati prima della
deallocazione, particolarmente utili in contesti in cui la privacy e la protezione
dei dati sono essenziali.

La necessità di rafforzare la sicurezza della memoria non riguarda esclusivamente
i linguaggi più esposti: anche ambienti più moderni e progettati per prevenire
questi errori, come Rust, offrono librerie che fungono da meccanismi di
enforcement aggiuntivi. Un esempio significativo è rappresentato da \texttt{zeroize}\protect\footnote{Documentazione:
\url{https://docs.rs/zeroize/latest/zeroize/}}, una libreria che consente di azzerare
in modo automatico e sicuro i contenuti di una variabile prima che venga rilasciata
o vada fuori dallo scope, riducendo il rischio che dati sensibili permangano in
memoria dopo l'uso.

Un ulteriore aspetto rilevante da considerare nella selezione delle librerie è
la \textit{trasparenza del codice}. L'adozione di librerie open source consente
agli sviluppatori di ispezionare direttamente l'implementazione, aumentando la fiducia
nei meccanismi adottati e offrendo la possibilità di verificarne il comportamento
effettivo. Questo costituisce un vantaggio concreto rispetto a librerie closed
source o scarsamente documentate, in quanto riduce il rischio di introdurre codice
opaco o potenzialmente pericoloso all'interno del progetto.

In conclusione, l'utilizzo consapevole di librerie progettate per migliorare la
memory safety rappresenta una buona pratica, sia nei linguaggi tradizionalmente vulnerabili
sia in quelli più moderni.

\subsection{Analisi Statica}
\label{sec:analisi-statica}

Dopo aver trattato le tecniche per scrivere codice sicuro, è utile introdurre un'ulteriore
forma di mitigazione: l'\textit{analisi statica}. A prima vista, questa potrebbe
sembrare una tecnica più affine alla fase di \textit{testing}, poiché si occupa di
rilevare errori e vulnerabilità. Tuttavia, l'evoluzione degli strumenti di sviluppo
e l'affermazione del principio di \textit{shift-left} hanno portato l'analisi
statica a essere adottata sempre più precocemente nel ciclo di vita del software,
al punto da diventare parte integrante della fase di sviluppo.

Con ``shift-left'' si intende una strategia che promuove l'anticipo delle attività
di verifica e validazione, allo scopo di intercettare i problemi il prima
possibile. In quest'ottica, l'analisi statica è oggi spesso integrata
direttamente negli ambienti di sviluppo (IDE) e nei sistemi di Continuous
Integration/Continuous Deployment (CI/CD), fornendo un feedback immediato agli sviluppatori
durante la scrittura del codice stesso.

L'analisi statica consiste nell'esaminare il codice sorgente (o, in alcuni casi,
il bytecode o il codice compilato) alla ricerca di difetti, vulnerabilità di
sicurezza e violazioni degli standard di codifica, il tutto senza eseguire il programma.
Il vantaggio principale è la possibilità di individuare problemi in una fase
precoce dello sviluppo, riducendo così i costi e la complessità delle successive
correzioni.

\begin{quote}
  \textbf{\textit{Nota:}} L'analisi statica rileva anche altri problemi di
  qualità del codice oltre alle vulnerabilità di memory safety, che tuttavia non
  sono trattati in questo elaborato.
\end{quote}

\noindent
Nonostante la sua capacità di rilevazione precoce degli errori, l'analisi statica
presenta alcune limitazioni: una delle sfide principali è la gestione dei
\textbf{falsi positivi}, ovvero segnalazioni di problemi che in realtà non
costituiscono errori o vulnerabilità nel contesto specifico dell'applicazione. Questo
è dovuto al fatto che gli analizzatori non riescono sempre a comprendere il contesto
completo in cui il codice viene eseguito, portando a segnalazioni errate.

Un esempio pratico può essere quello del~\autoref{lst:copy_string}: in questo caso,
alcuni analizzatori statici potrebbero segnalare un potenziale buffer overflow, nonostante
il controllo esplicito della lunghezza della stringa di input. Questa situazione
si verifica, ad esempio, quando l'analizzatore non tiene conto del legame
semantico tra la condizione \texttt{if} e l'istruzione \texttt{strcpy}. Altri tool
invece potrebbero adottare strategie conservative che segnalano sempre funzioni come
\texttt{strcpy} come potenzialmente pericolose.

Allo stesso modo, l'analisi statica può anche portare a \textbf{falsi negativi},
ossia la mancata rilevazione di problemi reali, specialmente per vulnerabilità
complesse che dipendono da configurazioni di runtime o interazioni specifiche
con input esterni non prevedibili staticamente.

In sintesi, integrare l'analisi statica nel processo di sviluppo è una pratica altamente
raccomandata, che può rilevare dimenticanze o falle introdotte dallo
sviluppatore e dalle librerie usate, contribuendo al rafforzamento della sicurezza.
Tuttavia, è fondamentale considerare i suoi limiti e continuare a migliorare il software
anche nelle successive fasi del ciclo di vita, ovvero durante il testing, la distribuzione
e la manutenzione.

\bigskip
\begin{lstlisting}[language=C, caption={Copia sicura di una stringa con \texttt{strcpy()}}, label={lst:copy_string}]
#include <stdlib.h>
#include <string.h>
void copy_string(const char *input) {
  char buffer[64];
  if (strlen(input) < sizeof(buffer)){
    strcpy(buffer, input); // Sicuro: controllato prima
  }
}
\end{lstlisting}
\section{Testing}
\label{sec:testing}

Pur avendo trattato varie mitigazioni applicabili nella fase dello sviluppo del software
(\autoref{sec:development}), queste si basano sul contributo dello sviluppatore
che può commettere errori (best practices e uso di librerie), o su strumenti che
non segnalano tutti i problemi possibili o che producono falsi positivi (analisi
statica). Vista l'inclinazione agli errori delle metodologie proposte, queste potrebbero
non essere sufficienti a garantire la sicurezza della memoria ed è quindi fondamentale
soffermarsi anche sulla parte di testing.

Il testing è l'altra fase principale del ciclo di vita del software dopo lo sviluppo.
In generale, il fine dei test è quello di rilevare \textit{unexpected behavior}
generici, che possono variare da errori di formattazione presenti negli output fino
a crash o vulnerabilità di sicurezza del software. Nel contesto di questo
documento, l'attenzione è rivolta alla memory safety, e quindi saranno
presentati strumenti e tecniche sviluppati appositamente per rilevare
problematiche legate alla memory safety.

\subsection{Analisi Dinamica}
\label{sec:analisi-dinamica}

Una delle tecniche più comuni per il testing del software dal punto di vista della
memoria è la cosiddetta \textit{analisi dinamica}. A differenza dell'analisi statica,
che si basa sull'ispezione del codice sorgente senza eseguirlo, l'analisi dinamica
agisce a \textit{runtime}, monitorando il comportamento del programma durante l'esecuzione.
Questo approccio permette di rilevare errori e vulnerabilità che emergono solo
in presenza di determinati dati di input, condizioni ambientali o flussi di esecuzione
specifici, spesso difficili da prevedere a priori.

Nel contesto della memory safety, esistono due principali categorie di strumenti
di analisi dinamica, in base al momento e al livello in cui vengono applicati:

\paragraph{Strumenti basati su compilazione.}
Alcuni strumenti operano direttamente a livello di compilazione, modificando il
codice sorgente o il codice intermedio per inserire controlli di sicurezza
aggiuntivi. Due esempi noti sono:

\begin{itemize}
  \item \textbf{AddressSanitizer (ASan):} uno strumento integrato nei
    compilatori \texttt{clang} e \texttt{gcc}, in grado di rilevare errori comuni
    come buffer overflow, use-after-free, e heap/stack out-of-bounds. ASan
    aggiunge al codice binario delle istruzioni extra per controllare la
    validità degli accessi in memoria durante l'esecuzione.

  \item \textbf{MemorySanitizer (MSan):} simile ad ASan, ma focalizzato sull'individuazione
    dell'uso di variabili non inizializzate. Questo tipo di errore, pur essendo più
    difficile da individuare, può portare a comportamenti non deterministici e
    vulnerabilità.
\end{itemize}

Questi strumenti offrono una buona copertura in fase di testing, ma comportano un
significativo overhead in termini di prestazioni e dimensione del binario,
motivo per cui non sono adatti all'uso in ambienti di produzione.

\paragraph{Strumenti basati sul binario.}
Altri strumenti operano a livello del binario compilato, senza necessità di
modificare il sorgente. Tra i più noti in questo ambito troviamo \textbf{Valgrind},
un framework di strumentazione dinamica che esegue il programma su una CPU virtuale,
intercettando ogni accesso alla memoria. Il suo modulo \texttt{Memcheck} è in
grado di rilevare accessi non inizializzati, leak di memoria, e violazioni di
accesso. Poiché funziona a livello binario, è particolarmente utile in scenari
in cui il codice sorgente non è disponibile.

Gli strumenti binari tendono ad essere più generali ma anche più lenti, in quanto
emulano o virtualizzano il comportamento del programma, causando un
rallentamento significativo dell'esecuzione (fino a 10--50 volte più lento)\footnote{Manuale
Valgrind: \url{https://valgrind.org/docs/manual/manual-core.html\#manual-core.whatdoes}}.

\paragraph{Limiti e considerazioni pratiche.}
Nonostante la loro efficacia, gli strumenti di analisi dinamica presentano
alcune limitazioni:

\begin{itemize}
  \item Il loro funzionamento dipende fortemente dalla qualità e dalla copertura
    dei test: eseguono controlli solo sul codice realmente percorso durante i
    test. Infatti, se il codice contiene ramificazioni condizionali, queste potrebbero
    non essere raggiunte durante l'esecuzione e quindi non essere segnalate se contengono
    bug o vulnerabilità.

  \item Introdurre strumentazione dinamica può alterare il comportamento temporale
    del programma, rendendo alcuni bug meno riproducibili.

  \item Per via dell'elevato overhead computazionale, questi strumenti sono
    generalmente utilizzati esclusivamente in fase di \textit{testing} e non in ambienti
    di produzione.
\end{itemize}

Nonostante ciò, l'analisi dinamica rappresenta un complemento fondamentale all'analisi
statica, permettendo di coprire una gamma più ampia di possibili vulnerabilità legate
alla memoria. Nell'ambito di un approccio difensivo multilivello, l'impiego congiunto
di più tecniche di testing aumenta significativamente la probabilità di intercettare
errori prima del rilascio del software.

\subsection{Fuzzing}
\label{sec:fuzzing}

Il \textit{fuzzing} (o fuzz testing) è una tecnica di testing automatizzato che consiste
nel generare una grande quantità di input, spesso casuali o leggermente mutati a
partire da casi validi, con l'obiettivo di far emergere comportamenti anomali,
crash o vulnerabilità nel programma in esecuzione. Sebbene non sia stato concepito
esclusivamente per individuare problemi di gestione della memoria, il fuzzing è
particolarmente efficace nel rivelare condizioni di \textit{undefined behavior},
spesso sintomatiche di bug memory-unsafe.

\smallskip
Ci sono diversi tipi di fuzzing, ma in generale si possono distinguere due categorie
principali:

\begin{itemize}
  \item \textbf{Fuzzing basato su generazione:} in questo approccio, gli input vengono
    generati in modo casuale o semi-casuale, spesso a partire da un insieme di
    casi di test validi. Questo metodo è utile per esplorare ampie porzioni dello
    spazio degli input, ma può risultare inefficace nel rilevare vulnerabilità
    specifiche.

  \item \textbf{Fuzzing basato su mutazione:} qui gli input validi vengono
    modificati (mutati) per creare nuovi casi di test. Questo approccio è particolarmente
    efficace per individuare vulnerabilità legate a formati di file o dati strutturati,
    poiché sfrutta la conoscenza preesistente di un input valido.
\end{itemize}

Gran parte dei tool di fuzzing non identificano direttamente gli errori di
memoria, ma si limitano a segnalare i crash o le anomalie comportamentali,
associandoli agli input che li hanno provocati. Per rendere l'analisi più mirata
ed efficace, i cosiddetti \textit{fuzzers} vengono comunemente usati in combinazione
con strumenti di analisi dinamica come i \textit{sanitizers}, in grado di diagnosticare
in modo più preciso gli errori sottostanti.

Questa mancanza ha portato a sviluppare anche strumenti di fuzzing \textit{multi-purpose},
che combinano le funzionalità di fuzzing con altre tecniche di testing. Un
esempio è AFL++\cite{afl_plus_plus} che, tra le sue funzionalità, include anche un
AddressSanitizer (QASan\cite{qasan}) integrato, in grado di rilevare errori di memoria
durante il fuzzing.

Il fuzzing descritto finora è chiamato \textit{black-box fuzzing}, in quanto non
richiede alcuna conoscenza del codice sorgente o della logica interna del programma
sottoposto a test. Esistono anche approcci \textit{white-box} che, invece, analizzano
il codice sorgente per generare input più mirati e specifici, che riescono ad
esplorare \textit{"l'albero di esecuzione"} del programma in modo più efficiente.
Un esempio è la cosiddetta \textit{Dynamic Symbolic Execution} (DSE), che
consiste nell'eseguire il programma con input simbolici, riuscendo ad esplorare tutte
le possibili esecuzioni e a generare input che portano a percorsi specifici.

\noindent

\paragraph{\textit{Nota:}}
Le tecniche e i tool di analisi dinamica e fuzzing, sono spesso associati al
codice C o C++, ma sono applicabili anche ad altri linguaggi: infatti Rust, pur essendo
uno dei linguaggi più sicuri dal punto di vista della memory safety, supporta il
cosiddetto \textit{Unsafe Rust}, che permette un maggiore controllo a basso
livello, ma introduce anche la possibilità di errori di memoria. Per questo
motivo, tool come Valgrind e AddressSanitizer sono stati adattati per essere utilizzati
anche con Rust.\cite{valgrind_rust}\cite{rust_manual_san}
\section{Distribuzione}
\label{sec:deployment}

In questa sezione verranno trattate diverse tecniche sviluppate per proteggere i
software durante la loro esecuzione. Queste mitigazioni possono essere applicate
in fase di compilazione, a runtime o essere integrate direttamente a livello di sistema
operativo o hardware.

Il \textit{deployment} rappresenta una delle fasi più critiche dal punto di
vista della sicurezza, in quanto è il momento in cui il software può essere esposto
agli attaccanti e, anche avendo adottato tutte le misure di sicurezza descritte
nei capitoli precedenti, è possibile che il software possa comunque essere vulnerabile
a exploit in fase di esecuzione.

È quindi consigliabile applicare le mitigazioni descritte in questa sezione,
soprattutto in modo combinato, per ridurre al minimo le possibilità di attacco.

\subsection{Protezioni a livello di memoria}
\label{sec:memory-protection}

Le protezioni a livello di memoria sono tecniche fondamentali per prevenire gli
attacchi che sfruttano vulnerabilità come buffer overflow, use-after-free e altre
forme di corruzione della memoria. Queste tecniche mirano a rendere più
difficile per un attaccante eseguire codice arbitrario o manipolare il flusso di
esecuzione di un programma attraverso la corruzione dei dati.

Queste mitigazioni, implementate sia a livello di compilatore che di sistema operativo,
rappresentano una prima linea di difesa essenziale contro molte classi di
attacchi. I paragrafi seguenti esplorano le principali mitigazioni implementate in
questo ambito.

\paragraph{Address Space Layout Randomization (ASLR)}
L'ASLR è una tecnica difensiva che mira a rendere imprevedibile la disposizione
in memoria delle principali aree di un processo, come stack, heap e librerie
condivise, al fine di ostacolare attacchi che si basano su indirizzi statici, e
quindi prevedibili.

Nonostante introduca entropia a ogni avvio del processo, la sua efficacia risulta
limitata sulle architetture a 32 bit. Lo studio di Shacham et al.\cite{aslr_effectiveness}
dimostra che in questi sistemi l'aleatorietà degli indirizzi per alcune regioni
di memoria può ridursi a soli 16 bit, rendendo possibili attacchi \textit{brute-force}
in grado di compromettere le difese in pochi minuti.

Per questo motivo, è cruciale valutare l'architettura target del software: l'ASLR
offre protezione più robusta sui sistemi a 64 bit, dove lo spazio d'indirizzamento
più ampio consente una dispersione significativa delle aree memorizzate. La
tecnica rimane comunque raccomandata come misura fondamentale, specialmente se integrata
con altre contromisure discusse in questa sezione, per aumentare sostanzialmente
la resilienza contro exploit noti.

\paragraph{Data Execution Prevention (DEP)}
La DEP è una misura di sicurezza che consiste nel contrassegnare specifiche aree
di memoria, come lo stack e l'heap, come non eseguibili. L'obiettivo è impedire
l'esecuzione di codice arbitrario iniettato in regioni di memoria destinate
esclusivamente alla conservazione di dati.

Questa protezione è particolarmente efficace contro attacchi come i buffer
overflow, nei quali un attaccante tenta di sovrascrivere l'indirizzo di ritorno di
una funzione per eseguire codice malevolo. Se la memoria che ospita tale codice
è marcata come non eseguibile, il tentativo di esecuzione fallisce, riducendo
significativamente il rischio di exploit.

Tuttavia, analogamente all'ASLR, anche la Data Execution Prevention è stata
aggirata nel tempo tramite tecniche avanzate come il Return Oriented Programming
(ROP) e il Jump Oriented Programming (JOP). Queste strategie non richiedono l'esecuzione
di codice iniettato, ma riutilizzano frammenti di codice legittimo già presenti
in memoria per comporre un payload malevolo. Di conseguenza, la DEP deve essere
considerata come una misura di sicurezza complementare, efficace solo se integrata
in una strategia difensiva più ampia.

\paragraph{Stack canary}
Gli stack canaries sono una tecnica di protezione software mirata a prevenire
gli attacchi basati sullo stack buffer overflow. Il nome deriva dall'analogia con
i canarini nelle miniere di carbone, utilizzati come sentinelle per rilevare gas
pericolosi: in questo caso, i canaries sono valori speciali inseriti tra le
variabili locali e l'indirizzo di ritorno nella struttura dello stack.

Durante l'esecuzione, prima di effettuare il ritorno da una funzione, il
programma verifica che il valore del canary non sia stato modificato. Se il valore
è stato alterato, indice di un probabile overflow dello stack, il processo viene
terminato o vengono attivate contromisure, impedendo così che un attaccante
possa modificare l'indirizzo di ritorno e quindi controllare il flusso di
esecuzione.

Questa protezione è efficace nel mitigare un'ampia classe di vulnerabilità
legate allo stack overflow, ma presenta alcune limitazioni: non protegge infatti
da overflow su heap o da altre forme di corruzione della memoria, né è efficace contro
attacchi che sfruttano tecniche più sofisticate come il Return Oriented Programming
(ROP), che non modificano direttamente il canary ma riutilizzano codice
legittimo.

Gli stack canaries sono generalmente implementati dal compilatore e possono
variare in complessità, dal semplice valore fisso a valori casuali generati all'avvio
del processo, aumentando così la difficoltà per l'attaccante di prevederli e
aggirarli.

\subsection{Controlli di integrità dell'esecuzione}
\label{sec:execution-integrity} Le tecniche presentate in questa sezione mirano
a garantire che il codice eseguito sia quello previsto e che non sia stato
manomesso o alterato in alcun modo.

\paragraph{Control Flow Integrity (CFI)}
è una tecnica avanzata di protezione che mira a impedire che un attaccante possa
alterare il flusso di controllo di un programma attraverso vulnerabilità come i buffer
overflow o gli use-after-free. L'idea alla base della CFI è quella di
assicurarsi che durante l'esecuzione, un programma possa compiere solo i salti di
controllo (es. chiamate indirette, ritorni di funzione) che sono stati previsti e
autorizzati in fase di compilazione.

Per ottenere questo obiettivo, la CFI è implementata principalmente a livello di
compilatore, che effettua le seguenti operazioni:
\begin{itemize}
  \item Analizza il grafo di controllo del programma

  \item Genera metadati che definiscono i salti validi e consentiti

  \item Inserisce controlli a runtime nel codice per verificare, prima
    di ogni salto, che la destinazione sia legittima
\end{itemize}

In questo modo, anche vettori di attacco come ROP e JOP, che si basano sull'alterazione
del flusso di controllo, possono essere mitigati e resi inefficaci.

In generale esistono due tipi principali di CFI:
\begin{itemize}
  \item \textbf{Fine-grained}: impone vincoli più specifici e dettagliati sui salti
    consentiti. Questo approccio migliora senz'altro la sicurezza, ma richiede un
    costo computazionale maggiore

  \item \textbf{Coarse-grained}: consente salti più generali e meno restrittivi.
    Questo approccio è più veloce e meno costoso, ma offre una protezione inferiore
    rispetto al fine-grained
\end{itemize}

Anche per la Control Flow Integrity gli svantaggi non mancano: la CFI può introdurre
un overhead significativo in termini di prestazioni e potrebbe essere aggirata se
i controlli sono troppo generici o se l'attaccante riesce a manipolare i metadati
di controllo. Tuttavia, se implementata correttamente, la CFI rappresenta una
misura di sicurezza efficace contro molte classi di attacchi.

\paragraph{Memory Tagging Extension (MTE)}
Questa è la prima delle due funzionalità di sicurezza \textit{"architecture-dependent"}
presenti in questa sezione. La MTE infatti, è una feature hardware introdotta da
Google e ARM (v8.5+) che mira a rilevare e prevenire errori di accesso alla
memoria, come buffer overflow e use-after-free, attraverso l'uso di tag di
memoria.

Parlando ad alto livello, la Memory Tagging Extension funziona associando un tag
a ogni area di memoria allocata. Questi tag sono utilizzati per identificare la
proprietà e la validità dell'area di memoria. Quando un programma tenta di accedere
a un'area di memoria, il processore verifica se il tag associato all'area
corrisponde al tag previsto per quell'operazione. Se i tag non corrispondono, il
processore genera un'eccezione, impedendo l'accesso non autorizzato e riducendo il
rischio di exploit.

La MTE può lavorare in 3 modalità differenti:
\begin{itemize}
  \item \textbf{Synchronous}: il processore genera un'eccezione quando viene rilevato
    un accesso non autorizzato e termina il programma. Questa modalità è
    ottimizzata per la ricerca di bug ma introduce molto overhead. Utile per la fase
    di testing, ma non raccomandata per l'uso in produzione.

  \item \textbf{Asynchronous}: il processore continua l'esecuzione quando viene rilevato
    un accesso non autorizzato, rimandando la generazione dell'eccezione fino al
    più vicino ingresso nel kernel e solo a quel punto, il processo viene terminato.
    Questa modalità è ottimizzata per le prestazioni piuttosto che per l'accuratezza
    dei report di bug, ed è raccomandata per l'uso in produzione su codice ben testato.

  \item \textbf{Asymmetric}(v8.7-A+): questa modalità è una combinazione delle
    due precedenti. Se l'accesso non autorizzato è causato da una lettura, il
    processore agisce in maniera "Synchronous", mentre se l'accesso non autorizzato
    è causato da una scrittura, il processore agisce in maniera "Asynchronous".
\end{itemize}

\paragraph{Pointer Authentication Code (PAC)}
La seconda funzionalità di sicurezza \textit{"architecture-dependent"} è il PAC.
Anche essa sviluppata da ARM (v8.3-A+), PAC è una tecnologia nata per prevenire in
modo efficace attacchi di tipo ROP o buffer overflow.

PAC si basa sull'idea di associare una firma crittografica a ogni puntatore utilizzato
nel programma. Questa firma viene calcolata in modo tale che qualsiasi modifica
al puntatore stesso o alla memoria a cui punta, renda impossibile verificare la firma.
In questo modo, se un attaccante tenta di manipolare un puntatore per eseguire codice
malevolo, la firma non corrisponderà più e il processore genererà un'eccezione, impedendo
l'esecuzione del codice non autorizzato.

PAC presenta due fasi principali:
\begin{itemize}
  \item \textbf{Firma}: quando un puntatore viene creato, il processore calcola
    una firma basata sul valore del puntatore, su una chiave segreta mantenuta nel
    processore e su un contesto opzionale(es. il valore dello stack pointer). La
    firma viene poi salvata nei bit inutilizzati del puntatore stesso per
    ridurre i tempi di verifica.

  \item \textbf{Verifica}: quando un puntatore viene utilizzato, il processore
    verifica la firma associata al puntatore. Se la firma è valida, l'accesso
    alla memoria è consentito; in caso contrario, viene generata un'eccezione.
\end{itemize}

\bigskip
\noindent
Entrambe le tecnologie di ARM, MTE e PAC, sono hardware-assisted e l'hoverhead introdotto
è molto basso, rendendole adatte per l'uso in produzione. Tuttavia, è importante
notare che queste tecnologie sono specifiche per l'architettura ARM e quindi non
sempre disponibili come mitigazioni per rafforzare la memory safety.

\subsection{Isolamento del software}
\label{sec:isolation}

Il concetto di \textit{isolamento} si riferisce alla separazione tra componenti
software con l'obiettivo di limitare l'impatto di eventuali compromissioni. L'idea
alla base è quella di confinare il danno in un'area ben definita, impedendo che
un exploit possa propagarsi ad altri componenti del sistema.

In ambito memory safety, l'isolamento rappresenta una misura efficace per mitigare
vulnerabilità che, pur essendo presenti, non riescono a compromettere l'intero sistema
grazie all'introduzione di barriere architetturali o software.

Le principali tecniche di isolamento includono:
\begin{itemize}
  \item \textbf{Sandboxing}: consiste nell'eseguire il software all'interno di un
    ambiente controllato e isolato, con accesso limitato a risorse di sistema e
    dati sensibili. Le sandbox possono essere implementate a livello di sistema operativo
    (es. \texttt{seccomp}, \texttt{AppArmor}) o tramite tecnologie di virtualizzazione
    leggera.

  \item \textbf{Containerizzazione}: fornisce isolamento a livello del sistema operativo,
    separando le applicazioni e le loro dipendenze in container indipendenti. Sebbene
    più leggeri rispetto alle macchine virtuali, i container condividono il
    kernel dell'host e richiedono un'attenta configurazione per evitare
    escalation di privilegi.

  \item \textbf{Virtualizzazione}: permette la creazione di macchine virtuali (VM)
    che emulano completamente l'hardware, consentendo l'esecuzione di più
    sistemi operativi isolati su un singolo host fisico. Le VM offrono un isolamento
    più forte rispetto ai container, ma introducono un maggiore overhead in
    termini di risorse e prestazioni.
\end{itemize}

Un esempio concreto dell'efficacia dell'isolamento lo troviamo nel caso reale di
WebAudio in Google Chrome, già menzionato nella
\autoref{sec:vulnerability_types}. L'uso della sandbox del browser impediva all'attaccante
di ottenere l'accesso completo al sistema, confinando l'esecuzione di codice
arbitrario all'interno dell'ambiente "sandboxed".\cite{webaudio_uaf}

È importante sottolineare che queste tecnologie non sono state progettate specificamente
per garantire la memory safety, ma la loro architettura consente di limitare l'impatto
di eventuali vulnerabilità isolando i processi e le risorse. Ad esempio, la
containerizzazione è nata principalmente per esigenze di portabilità e
scalabilità delle applicazioni, ma fornisce anche un utile livello di confinamento.

Tuttavia, l'isolamento non elimina completamente le vulnerabilità, e introduce un
costo aggiuntivo in termini di complessità gestionale e prestazioni. Pertanto, la
scelta della tecnica più appropriata deve basarsi su una valutazione del contesto
applicativo e dei requisiti di sicurezza specifici.
\section{Mantenimento}
\label{sec:maintenance}

La fase di mantenimento rappresenta l'ultima, ma non meno importante, tappa del ciclo
di vita del software. In questo stadio, l'obiettivo primario è garantire che il sistema
continui a operare in modo corretto e sicuro nel tempo, rispondendo a nuove
esigenze, aggiornamenti tecnologici e, soprattutto, alla scoperta di nuove vulnerabilità.

\subsection{Tracciamento delle Dipendenze}
\label{sec:tracciamento-dipendenze}

Un aspetto critico per la sicurezza della memoria nel lungo periodo è la
gestione delle dipendenze software, specialmente quando si utilizzano librerie esterne,
spesso sviluppate in linguaggi memory-unsafe come C e C++.

Sistemi moderni possono contare su strumenti di dependency tracking
automatizzati, in grado di:
\begin{itemize}
  \item identificare vulnerabilità note tramite l'integrazione con database CVE,

  \item notificare la presenza di versioni obsolete o insicure,

  \item suggerire aggiornamenti sicuri o alternative memory-safe.
\end{itemize}

Framework come OWASP Dependency-Check\footnotemark o strumenti integrati nei sistemi di CI/CD
(es. GitHub Dependabot) automatizzano parte di questo processo, facilitando la gestione
reattiva delle vulnerabilità. In contesti critici è inoltre consigliabile:
\footnotetext{\url{https://owasp.org/www-project-dependency-check/}}
\begin{itemize}
  \item mantenere un \textit{Software Bill of Materials} (SBOM), per avere un inventario
    trasparente delle dipendenze

  \item ridurre il numero di dipendenze esterne inutili (minimizzazione
    della superficie di attacco)

  \item preferire librerie attivamente manutenute e con audit pubblici di sicurezza.
\end{itemize}

In linguaggi come Rust, che enfatizzano la sicurezza di tipo e l'ownership, è
comunque fondamentale monitorare i crate esterni, specialmente quelli che usano blocchi
\texttt{unsafe} o effettuano Foreign Function Interface (FFI) verso codice legacy. Il tracciamento deve essere
quindi parte integrante del workflow di mantenimento.

\subsection{Monitoraggio e Logging}
\label{sec:monitoraggio-logging}

Oltre al tracciamento delle dipendenze, una strategia efficace di mantenimento prevede
l'adozione di sistemi di monitoraggio e logging, con l'obiettivo di individuare
comportamenti anomali potenzialmente legati a vulnerabilità di memoria.

In particolare, i meccanismi di logging possono aiutare a:
\begin{itemize}
  \item rilevare accessi fuori dai limiti o crash ripetuti

  \item diagnosticare memory leak persistenti

  \item correlare anomalie a specifici aggiornamenti di codice o librerie.
\end{itemize}

In definitiva, l'integrazione di tecniche di osservabilità avanzata rappresenta un
fattore abilitante per una manutenzione proattiva, capace di prevenire l'escalation
di vulnerabilità latenti e migliorare la resilienza complessiva del sistema
software.