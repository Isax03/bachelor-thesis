\chapter{Memory Safety nel SDLC}
\label{cha:sdlc}

Dopo aver introdotto i concetti fondamentali legati alla memory safety e aver
visto l'impatto che una gestione errata della memoria può avere sulla sicurezza,
sull'affidabilità e sulla stabilità del software, questo capitolo si concentra
su come questi principi possano essere concretamente integrati all'interno del
ciclo di vita dello sviluppo del software \textit{(Software Development Life Cycle,
SDLC)}.

La memory safety non è un obiettivo che può essere raggiunto in un singolo momento
del processo di sviluppo, ma deve essere perseguito lungo tutte le sue fasi. Una
falla memory-unsafe, infatti, può essere introdotta in qualsiasi momento, in
maniera più o meno diretta: dal linguaggio di programmazione utilizzato,
passando per la qualità del codice, fino ad arrivare all'ambiente di esecuzione.
Per questo motivo, ogni fase dello SDLC rappresenta un'opportunità per prevenire,
individuare e mitigare vulnerabilità legate all'uso insicuro della memoria. Comprendere
il contributo specifico di ciascuna fase alla sicurezza della memoria consente ai
team di sviluppo di adottare una strategia integrata e preventiva, riducendo
significativamente il rischio di introdurre vulnerabilità critiche nel sistema.

Va sottolineato fin da subito che le tecniche, gli strumenti e i linguaggi eventualmente
citati \textit{non} vanno intesi come universali o necessariamente ottimali per
ogni contesto. L'obiettivo di questo capitolo non è infatti quello di raccomandare
soluzioni specifiche, ma piuttosto fornire un \textit{quadro strutturato di
buone pratiche e approcci} che possono essere adattati in base ai requisiti del progetto,
alle risorse disponibili e al contesto applicativo.

\section{Fasi preliminari}
\label{sec:preliminary_stages}

\subsection{Scelta del Linguaggio}
\label{sec:linguaggio}
\section{Sviluppo}
\label{sec:development}

La fase di sviluppo del software rappresenta il cuore del ciclo di vita del software.
In questa sezione vengono analizzate le best practice di scrittura del codice
consigliate da enti specializzati come OWASP e SEI CERT, che si occupano di sicurezza
e qualità del software.

Viene inoltre esaminata l'importanza di utilizzare librerie che implementano funzioni
e metodi per rafforzare la memory safety anche in linguaggi considerati generalmente
non memory safe.

Infine, viene trattata l'analisi statica, una tecnica che permette di
rilevare vulnerabilità e problemi di qualità del codice senza la necessità di eseguire
il programma.

\subsection{Best practice nel codice}
\label{sec:best-practices-codice}

La creazione di software con elevati standard di memory safety inizia dalle
decisioni prese durante la fase di implementazione del codice. Per quanto si possano
utilizzare linguaggi moderni e progettati con meccanismi di sicurezza intrinseci,
la qualità del software dipende inevitabilmente dalle decisioni adottate dallo
sviluppatore. Per questo motivo esistono linee guida consolidate, che aiutano a
prevenire vulnerabilità legate alla gestione della memoria. L'applicazione di queste
buone pratiche consente di ridurre drasticamente il rischio di exploit e aumentare
la robustezza del software fin dalle sue fondamenta, costituendo un primo
livello di difesa essenziale, specialmente in linguaggi notoriamente non memory safe.

Una delle principali fonti di riferimento è la \textbf{OWASP Secure Coding
Practices Quick Reference Guide}~\cite{owasp_best_practices}, che fornisce una checklist
focalizzata su vari aspetti della sicurezza del software, tra cui la gestione
della memoria.

Di seguito si riportano alcune delle principali best practice relative alla
memory safety:

\begin{itemize}
  \item \textbf{Validare input e output da fonti non affidabili}: tutti i dati esterni
    devono essere sottoposti a controlli per evitare che valori malformati causino
    overflow o corruzione di memoria.

  \item \textbf{Verificare le dimensioni dei buffer}: prima di accedere o scrivere
    su un buffer, è necessario assicurarsi che sia sufficientemente grande da
    contenere i dati previsti.

  \item \textbf{Assicurare la corretta terminazione delle stringhe}: durante l'utilizzo
    di funzioni che richiedono una dimensione in byte, è essenziale garantire
    che il carattere terminatore \texttt{NULL} sia sempre presente e correttamente
    posizionato.

  \item \textbf{Controllare i limiti dei buffer nei cicli}: è importante assicurarsi
    che ogni iterazione non superi i limiti di memoria del buffer.

  \item \textbf{Troncare le stringhe in input}: limitare la lunghezza delle stringhe
    in input prima di passarle ad altre funzioni riduce i rischi di overflow.

  \item \textbf{Chiudere esplicitamente le risorse}: è buona pratica liberare esplicitamente
    memoria, file e altri handle di sistema, senza fare affidamento sul garbage collector
    (se presente).

  \item \textbf{Evitare l'uso di funzioni note come vulnerabili}: funzioni standard
    come \texttt{gets}, \texttt{strcpy} e \texttt{sprintf} sono intrinsecamente pericolose
    e dovrebbero essere sostituite con alternative più sicure (a tal proposito
    si veda la~\autoref{sec:librerie} nel seguito).

  \item \textbf{Liberare correttamente la memoria}: la memoria dinamica va liberata
    in tutti i punti di uscita del programma, compresi quelli dovuti a
    condizioni di errore (evita memory leak).

  \item \textbf{Sovrascrivere i dati sensibili prima della deallocazione}: quando
    si gestiscono dati come password o chiavi crittografiche, è consigliabile
    sovrascrivere la memoria per evitare che rimangano accessibili in chiaro (zeroization).
\end{itemize}

Oltre alle regole già formalizzate, risulta opportuno considerare anche altre
buone pratiche non sempre trattate esplicitamente, ma che possono contribuire a prevenire
i bug descritti nella~\autoref{sec:vulnerability_types}. Un esempio
significativo è il controllo degli \textbf{integer overflow} durante il calcolo degli
indici di un array: un valore che supera il massimo rappresentabile da un intero
può produrre un indice errato, portando a un accesso fuori dai limiti della memoria.

A complemento della guida proposta da \textit{OWASP}, i \textbf{SEI CERT Coding
Standard}~\cite{cert_coding_standard} offrono una raccolta di regole dettagliate,
spesso accompagnate da esempi di codice non conforme (\textit{non-compliant}) e
dalle relative soluzioni corrette (\textit{compliant}). Le regole di SEI CERT
sono organizzate in base al linguaggio di programmazione e risultano molto più
numerose rispetto a quelle di OWASP, rendendo difficile presentarle tutte in questo
documento. Tuttavia, per fornire un'idea della loro struttura e approccio, viene
riportato nell'\autoref{appendix:sei_cert} un esempio di regola SEI CERT
relativa alla gestione sicura degli array in C.

\subsection{Librerie}
\label{sec:librerie}

Per migliorare la memory safety nei progetti software, risulta spesso
vantaggioso affidarsi a librerie progettate per offrire un livello di controllo
e protezione superiore rispetto a quello garantito dalle funzioni standard del
linguaggio. Questo aspetto è particolarmente rilevante nei linguaggi storicamente
non memory safe, come C e C++, dove l'allocazione manuale e la gestione esplicita
della memoria espongono facilmente a vulnerabilità come buffer overflow, use
after free o double free.

Nel corso degli anni sono state sviluppate numerose librerie che affrontano
direttamente queste problematiche. Alcune di esse, come \textit{xmalloc}\footnote{Repository
GitHub: \url{https://github.com/rosingh/xmalloc}}, fungono da wrapper per le funzioni
di allocazione (\texttt{malloc()}, \texttt{free()}, \texttt{realloc()}), implementando
controlli interni per verificare la correttezza delle operazioni e rilevare anomalie
durante l'utilizzo della heap. Altre, come \textit{safestringlib}\footnote{Repository
GitHub: \url{https://github.com/intel/safestringlib}}, sostituiscono le funzioni
standard per la manipolazione delle stringhe con versioni più sicure (\texttt{strcpy\_s},
\texttt{memset\_s}, etc.), che includono verifiche esplicite sulla lunghezza dei
buffer per evitare scritture fuori dai limiti. La libreria \textit{Sailfish Pool}\footnote{Repository
GitHub: \url{https://github.com/dyne/sailfish-pool}} invece, introduce tecniche
come la preallocazione di pool di memoria e la zeroization dei dati prima della
deallocazione, particolarmente utili in contesti in cui la privacy e la protezione
dei dati sono essenziali.

La necessità di rafforzare la sicurezza della memoria non riguarda esclusivamente
i linguaggi più esposti: anche ambienti più moderni e progettati per prevenire
questi errori, come Rust, offrono librerie che fungono da meccanismi di
enforcement aggiuntivi. Un esempio significativo è rappresentato da \texttt{zeroize}\protect\footnote{Documentazione:
\url{https://docs.rs/zeroize/latest/zeroize/}}, una libreria che consente di azzerare
in modo automatico e sicuro i contenuti di una variabile prima che venga rilasciata
o vada fuori dallo scope, riducendo il rischio che dati sensibili permangano in
memoria dopo l'uso.

Un ulteriore aspetto rilevante da considerare nella selezione delle librerie è
la \textit{trasparenza del codice}. L'adozione di librerie open source consente
agli sviluppatori di ispezionare direttamente l'implementazione, aumentando la fiducia
nei meccanismi adottati e offrendo la possibilità di verificarne il comportamento
effettivo. Questo costituisce un vantaggio concreto rispetto a librerie closed
source o scarsamente documentate, in quanto riduce il rischio di introdurre codice
opaco o potenzialmente pericoloso all'interno del progetto.

In conclusione, l'utilizzo consapevole di librerie progettate per migliorare la
memory safety rappresenta una buona pratica, sia nei linguaggi tradizionalmente vulnerabili
sia in quelli più moderni.

\subsection{Analisi statica}
\label{sec:analisi-statica}

Dopo aver trattato le tecniche per scrivere codice sicuro, è utile introdurre un'ulteriore
forma di mitigazione: l'\textit{analisi statica}. A prima vista, questa potrebbe
sembrare una tecnica più affine alla fase di \textit{testing}, poiché si occupa di
rilevare errori e vulnerabilità. Tuttavia, l'evoluzione degli strumenti di sviluppo
e l'affermazione del principio di \textit{shift-left} hanno portato l'analisi
statica a essere adottata sempre più precocemente nel ciclo di vita del software,
al punto da diventare parte integrante della fase di sviluppo.

Con ``shift-left'' si intende una strategia che promuove l'anticipo delle attività
di verifica e validazione, allo scopo di intercettare i problemi il prima
possibile. In quest'ottica, l'analisi statica è oggi spesso integrata
direttamente negli ambienti di sviluppo (IDE) e nei sistemi di Continuous
Integration/Continuous Deployment (CI/CD), fornendo un feedback immediato agli sviluppatori
durante la scrittura del codice stesso.

L'analisi statica consiste nell'esaminare il codice sorgente (o, in alcuni casi,
il bytecode o il codice compilato) \textbf{senza eseguirlo}, alla ricerca di difetti
e problemi che spaziano dalle vulnerabilità legate alla memory safety fino ad altri
tipi di errori come divisioni per zero, codice morto o violazioni delle
convenzioni di stile. Il vantaggio principale è la possibilità di individuare problemi
in una fase precoce dello sviluppo, riducendo così i costi e la complessità
delle successive correzioni.

Nonostante la sua capacità di rilevazione precoce degli errori, l'analisi statica
presenta alcune limitazioni: una delle sfide principali è la gestione dei
\textbf{falsi positivi}, ovvero segnalazioni di problemi che in realtà non
costituiscono errori o vulnerabilità nel contesto specifico dell'applicazione. Questo
è dovuto al fatto che gli analizzatori non riescono sempre a comprendere il contesto
completo in cui il codice viene eseguito, portando a segnalazioni errate.

Un esempio pratico può essere quello del~\autoref{lst:copy_string}: in questo caso,
alcuni analizzatori statici, come Frama-C, Clang o cppcheck, approfonditi nella~\autoref{sec:initial_analysis},
potrebbero segnalare un potenziale buffer overflow, nonostante il controllo esplicito
della lunghezza della stringa di input. Questa situazione si verifica, ad esempio,
quando l'analizzatore non tiene conto del legame semantico tra la condizione
\texttt{if} e l'istruzione \texttt{strcpy}. Altri tool invece potrebbero adottare
strategie conservative che segnalano sempre funzioni come \texttt{strcpy} come
potenzialmente pericolose.

Allo stesso modo, l'analisi statica può anche portare a \textbf{falsi negativi},
ossia la mancata rilevazione di problemi reali, specialmente per vulnerabilità
complesse che dipendono da configurazioni di runtime o interazioni specifiche
con input esterni non prevedibili staticamente.

In sintesi, integrare l'analisi statica nel processo di sviluppo è una pratica altamente
raccomandata, che può rilevare dimenticanze o falle introdotte dallo
sviluppatore e dalle librerie usate, contribuendo al rafforzamento della sicurezza.
Tuttavia, è fondamentale considerare i suoi limiti e continuare a migliorare il software
anche nelle successive fasi del ciclo di vita, ovvero durante il testing, la distribuzione
e la manutenzione.

\bigskip
\begin{lstlisting}[language=C, caption={Copia sicura di una stringa con \texttt{strcpy()}}, label={lst:copy_string}]
#include <stdlib.h>
#include <string.h>
void copy_string(const char *input) {
  char buffer[64];
  if (strlen(input) < sizeof(buffer)-1){
    strcpy(buffer, input); // Sicuro: controllato prima
  }
}
\end{lstlisting}
\section{Testing}
\label{sec:testing}

Pur avendo trattato varie mitigazioni applicabili nella fase dello sviluppo del software
(\autoref{sec:development}), queste si basano sul contributo dello sviluppatore
che può commettere errori (best practices e uso di librerie) o su strumenti che
possono produrre falsi positivi o falsi negativi (analisi statica). Vista l'inclinazione
agli errori, le metodologie proposte potrebbero non essere sufficienti a
garantire la sicurezza della memoria ed è quindi fondamentale soffermarsi anche
sulla parte di testing.

Il testing è l'altra fase principale del ciclo di vita del software dopo lo
sviluppo. In generale, il fine dei test è quello di rilevare \textit{unexpected
behavior} generici, che possono variare da errori di formattazione presenti
negli output fino a crash o vulnerabilità di sicurezza del software. Nel contesto
di questo documento, l'attenzione è rivolta alla memory safety, e quindi saranno
presentati strumenti e tecniche sviluppati appositamente per rilevare problematiche
legate alla memory safety.

\subsection{Analisi Dinamica}
\label{sec:analisi-dinamica}

\subsection{Fuzzing}
\label{sec:fuzzing}
\section{Distribuzione ed Esecuzione}
\label{sec:deployment}

In questa sezione vengono trattate diverse tecniche sviluppate per proteggere i software
durante la loro esecuzione. Queste mitigazioni possono essere applicate in fase
di compilazione, a runtime o essere integrate direttamente a livello di sistema operativo
o hardware.

Il \textit{deployment} rappresenta una delle fasi più critiche dal punto di
vista della sicurezza, in quanto è il momento in cui il software può essere esposto
agli attaccanti e, anche avendo adottato tutte le misure di sicurezza descritte
nei capitoli precedenti, è possibile che il software possa comunque essere vulnerabile
a exploit in fase di esecuzione.

È quindi consigliabile applicare le mitigazioni descritte in questa sezione,
soprattutto in modo combinato, per ridurre al minimo le possibilità di attacco.

\subsection{Protezioni a livello di memoria}
\label{sec:memory-protection}

Il primo tipo di protezione trattato in questa sezione riguarda i meccanismi di
difesa a livello di memoria. Queste tecniche mirano a ostacolare l'abuso di vulnerabilità
come buffer overflow, use-after-free o corruzione dello stack, agendo sulla disposizione,
i permessi e l'integrità delle aree di memoria durante l'esecuzione.

Queste mitigazioni, implementate sia a livello di compilatore che di sistema operativo,
rappresentano una prima linea di difesa essenziale contro molte classi di
attacchi.

I paragrafi seguenti illustrano le principali tecniche implementate in questo
ambito: ASLR, DEP e Stack Canary. Viene fornita una panoramica del funzionamento
e dell'efficacia, senza tuttavia entrare nei dettagli implementativi relativi
alla loro attivazione nei diversi ambienti di sviluppo.

\paragraph{Address Space Layout Randomization (ASLR)}
L'ASLR è una tecnica difensiva che mira a rendere imprevedibile la disposizione in
memoria delle principali aree di un processo, come stack, heap e librerie condivise,
al fine di ostacolare attacchi che si basano su indirizzi statici, e quindi prevedibili.

Nonostante introduca entropia a ogni avvio del processo, la sua efficacia
risulta limitata sulle architetture a 32 bit. Lo studio di Shacham et al.\cite{aslr_effectiveness}
dimostra che in questi sistemi l'aleatorietà degli indirizzi per alcune regioni di
memoria può ridursi a soli 16 bit, rendendo possibili attacchi \textit{brute-force}
in grado di compromettere le difese in pochi minuti.

Per questo motivo, è cruciale valutare l'architettura target del software: l'ASLR
offre protezione più robusta sui sistemi a 64 bit, dove lo spazio d'indirizzamento
più ampio consente una dispersione significativa delle aree memorizzate. La tecnica
rimane comunque raccomandata come misura fondamentale, specialmente se integrata
con altre contromisure discusse in questa sezione, per aumentare sostanzialmente
la resilienza contro exploit noti.

\paragraph{Data Execution Prevention (DEP)\protect\footnote{Fonte: \url{https://www.datasunrise.com/knowledge-center/dep-data-execution-prevention/},
ultimo accesso: 27 maggio 2025}}
La DEP è una misura di sicurezza che consiste nel contrassegnare specifiche aree
di memoria, come non eseguibili. L'obiettivo è impedire l'esecuzione di codice
arbitrario iniettato in regioni di memoria destinate esclusivamente alla
conservazione di dati.

Questa protezione è particolarmente efficace contro attacchi come i buffer
overflow, nei quali un attaccante tenta di sovrascrivere l'indirizzo di ritorno di
una funzione per eseguire codice malevolo. Se la memoria che ospita tale codice
è marcata come non eseguibile, il tentativo di esecuzione fallisce, riducendo
significativamente il rischio di exploit.

Tuttavia, analogamente all'ASLR, anche la Data Execution Prevention è stata
aggirata nel tempo tramite tecniche avanzate come il Return Oriented Programming
(ROP) e il Jump Oriented Programming (JOP). Queste strategie non richiedono l'esecuzione
di codice iniettato, ma riutilizzano frammenti di codice legittimo già presenti
in memoria per comporre un payload malevolo. Di conseguenza, la DEP deve essere
considerata come una misura di sicurezza complementare, efficace solo se integrata
in una strategia difensiva più ampia.

\paragraph{Stack Canary}
Gli stack canary sono una tecnica di protezione software mirata a prevenire gli
attacchi basati sullo stack buffer overflow. Il nome deriva dall'analogia con i canarini
nelle miniere di carbone, utilizzati come sentinelle per rilevare gas pericolosi:
in questo caso, i canary sono valori speciali (con valore fisso o casuale)
inseriti tra le variabili locali e l'indirizzo di ritorno nella struttura dello
stack.

Durante l'esecuzione, prima di effettuare il ritorno da una funzione, il
programma verifica che il valore del canary non sia stato modificato. Se il valore
è stato alterato, indice di un probabile overflow dello stack, il processo viene
terminato o vengono attivate contromisure, impedendo così che un attaccante
possa modificare l'indirizzo di ritorno e quindi controllare il flusso di
esecuzione.

Questa protezione è efficace nel mitigare un'ampia classe di vulnerabilità
legate allo stack overflow, ma presenta alcune limitazioni: non protegge infatti
da overflow su heap o da altre forme di corruzione della memoria, né è efficace contro
attacchi che sfruttano tecniche più sofisticate come il ROP, che non modificano
direttamente il canary ma riutilizzano codice legittimo.\cite{stack_canaries}

\subsection{Controlli di integrità dell'esecuzione}
\label{sec:execution-integrity} Le tecniche presentate in questa sezione mirano a
garantire che il codice eseguito sia quello previsto e che non sia stato manomesso
o alterato in alcun modo.

\paragraph{Control Flow Integrity (CFI)}
È una tecnica avanzata di protezione che mira a impedire che un attaccante possa
alterare il flusso di controllo di un programma attraverso vulnerabilità come i
buffer overflow o gli use-after-free. L'idea alla base della CFI è quella di assicurarsi
che durante l'esecuzione, un programma possa compiere solo i salti di controllo
(es. chiamate indirette, ritorni di funzione) che sono stati previsti e
autorizzati in fase di compilazione.

Per raggiungere questo obiettivo, la CFI è implementata principalmente a livello
di compilatore, che analizza il grafo di controllo del programma, per generare i
metadati che definiscono i salti validi. Una volta identificate le chiamate
consentite, inserisce controlli a runtime nel codice per verificare che la destinazione
sia legittima, prima di ogni salto.

In questo modo, anche vettori di attacco come ROP e JOP, che si basano sull'alterazione
del flusso di controllo, possono essere mitigati e resi inefficaci.

In generale esistono due tipi principali di CFI:
\begin{itemize}
  \item \textbf{Fine-grained}: impone vincoli più specifici e dettagliati sui salti
    consentiti. Questo approccio migliora senz'altro la sicurezza, ma richiede un
    costo computazionale maggiore.

  \item \textbf{Coarse-grained}: consente salti più generali e meno restrittivi.
    Questo approccio è più veloce e meno costoso, ma offre una protezione inferiore
    rispetto al fine-grained.
\end{itemize}

Anche per la CFI gli svantaggi non mancano: essa può introdurre un overhead
significativo in termini di prestazioni e potrebbe essere aggirata se i
controlli sono troppo generici o se l'attaccante riesce a manipolare i metadati
di controllo. Tuttavia, se implementata correttamente, la CFI rappresenta una misura
di sicurezza efficace contro molte classi di attacchi.\cite{control_flow_integrity}

\paragraph{Memory Tagging Extension (MTE)\protect\footnote{https://developer.arm.com/documentation/108035/0100/Introduction-to-the-Memory-Tagging-Extension}}
Questa è la prima delle due funzionalità di sicurezza \textit{``architecture-dependent''}
presenti in questa sezione. La MTE infatti, è una feature hardware introdotta da
Google e ARM (v8.5+) che mira a rilevare e prevenire errori di accesso alla
memoria, come buffer overflow e use-after-free, attraverso l'uso di \textit{tag
di memoria}.

A livello concettuale, la MTE associa un tag (etichetta) a ogni area di memoria
allocata. Questi tag sono utilizzati per identificare la proprietà e la validità
dell'area di memoria. Quando un programma tenta di accedere a un'area di memoria,
il processore verifica se il tag associato all'area corrisponde al tag previsto per
quell'operazione. Se i tag non corrispondono, il processore genera un'eccezione,
impedendo l'accesso non autorizzato e riducendo il rischio di exploit.

\bigskip
La MTE può lavorare in tre modalità differenti:
\begin{itemize}
  \item \textbf{Synchronous}: il processore genera un'eccezione quando viene rilevato
    un accesso non autorizzato e termina il programma. Questa modalità è
    ottimizzata per la ricerca di bug ma introduce molto overhead. Utile per la fase
    di testing, ma non raccomandata per l'uso in produzione.

  \item \textbf{Asynchronous}: il processore continua l'esecuzione quando viene rilevato
    un accesso non autorizzato, rimandando la generazione dell'eccezione fino al
    più vicino ingresso nel kernel, e solo a quel punto il processo viene terminato.
    Questa modalità è ottimizzata per le prestazioni piuttosto che per l'accuratezza
    dei report di bug, ed è raccomandata per l'uso in produzione su codice ben testato.

  \item \textbf{Asymmetric}(v8.7-A+): questa modalità è una combinazione delle
    due precedenti. Se l'accesso non autorizzato è causato da una lettura, il
    processore agisce in maniera ``Synchronous'', mentre se l'accesso non autorizzato
    è causato da una scrittura, il processore agisce in maniera ``Asynchronous''.
\end{itemize}

\paragraph{Pointer Authentication Code (PAC)\protect\footnote{https://developer.arm.com/documentation/109576/0100/Pointer-Authentication-Code/Introduction-to-PAC}}
L'altra funzionalità di sicurezza \textit{``architecture-dependent''} presente
in questa sezione è il PAC. Anche essa sviluppata da ARM (v8.3-A+), PAC è una
tecnologia nata per prevenire in modo efficace attacchi di tipo ROP o buffer
overflow.

PAC si basa sull'idea di associare una firma crittografica a ogni puntatore
utilizzato nel programma. Questa firma viene calcolata in modo tale che qualsiasi
modifica al puntatore stesso o alla memoria a cui punta, renda impossibile
verificare la firma. In questo modo, se un attaccante tenta di manipolare un
puntatore per eseguire codice malevolo, la firma non corrisponderà più e il processore
genererà un'eccezione, impedendo l'esecuzione del codice non autorizzato.

PAC presenta due fasi principali:
\begin{itemize}
  \item \textbf{Firma}: quando un puntatore viene creato, il processore calcola
    una firma basata sul valore del puntatore, su una chiave segreta mantenuta nel
    processore e su un contesto opzionale (es. il valore dello stack pointer). La
    firma viene poi salvata nei bit inutilizzati del puntatore stesso per ridurre
    i tempi di verifica.

  \item \textbf{Verifica}: quando un puntatore viene utilizzato, il processore
    verifica la firma associata al puntatore. Se la firma è valida, l'accesso
    alla memoria è consentito; in caso contrario, viene generata un'eccezione.
\end{itemize}

\smallskip
\noindent
Entrambe le tecnologie sviluppate da ARM (MTE e PAC), sono hardware-assisted e l'hoverhead
introdotto è molto basso, rendendole adatte per l'uso in produzione. Tuttavia, è
importante notare che queste tecnologie sono specifiche per l'architettura ARM e
quindi non sempre disponibili come mitigazioni per rafforzare la memory safety.

\subsection{Isolamento del software}
\label{sec:isolation}

Il concetto di \textbf{isolamento} si riferisce alla separazione tra componenti
software con l'obiettivo di limitare l'impatto di eventuali compromissioni. L'idea
alla base è quella di confinare il danno in un'area ben definita, impedendo che
un exploit possa propagarsi ad altri componenti del sistema.

In ambito memory safety, l'isolamento rappresenta una misura efficace per mitigare
vulnerabilità che, pur essendo presenti, non riescono a compromettere l'intero sistema
grazie all'introduzione di barriere architetturali o software.

Le principali tecniche di isolamento includono:
\begin{itemize}
  \item \textbf{Sandboxing}: consiste nell'eseguire il software all'interno di un
    ambiente controllato e isolato, con accesso limitato a risorse di sistema e
    dati sensibili. Le sandbox possono essere implementate a livello di sistema operativo
    (es. \texttt{seccomp}, \texttt{AppArmor}) o tramite tecnologie di virtualizzazione
    leggera.

  \item \textbf{Containerizzazione}: fornisce isolamento a livello del sistema operativo,
    separando le applicazioni e le loro dipendenze in container indipendenti. Sebbene
    più leggeri rispetto alle macchine virtuali, i container condividono il
    kernel dell'host e richiedono un'attenta configurazione per evitare
    escalation di privilegi.

  \item \textbf{Virtualizzazione}: permette la creazione di macchine virtuali (VM)
    che emulano completamente l'hardware, consentendo l'esecuzione di più
    sistemi operativi isolati su un singolo host fisico. Le VM offrono un isolamento
    più forte rispetto ai container, ma introducono un maggiore overhead in
    termini di risorse e prestazioni.
\end{itemize}

Un esempio concreto dell'efficacia dell'isolamento lo troviamo nel caso reale di
WebAudio in Google Chrome, in cui uno use after free premetteva l'esecuzione
remota di codice. Tuttavia, l'uso della sandbox del browser impediva all'attaccante
di ottenere l'accesso completo al sistema, confinando l'esecuzione di codice arbitrario
all'interno dell'ambiente ``sandboxed''.\cite{webaudio_uaf}

È importante sottolineare che queste tecnologie non sono state progettate
specificamente per garantire la memory safety, ma la loro architettura consente di
limitare l'impatto di eventuali vulnerabilità isolando i processi e le risorse.
Ad esempio, la containerizzazione è nata principalmente per esigenze di portabilità
e scalabilità delle applicazioni, ma fornisce anche un utile livello di
confinamento.

Tuttavia, l'isolamento non elimina completamente le vulnerabilità, e introduce
un costo aggiuntivo in termini di complessità gestionale e prestazioni. Pertanto,
la scelta della tecnica più appropriata deve basarsi su una valutazione del
contesto applicativo e dei requisiti di sicurezza specifici.
\section{Mantenimento}
\label{sec:maintenance}

\subsection{Tracciamento delle Dipendenze}
\label{sec:tracciamento-dipendenze}