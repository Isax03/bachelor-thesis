\section{Pianificazione, Analisi dei Requisiti e Design}
\label{sec:planning_requirements_design}

Le fasi preliminari del Software Development Life Cycle (SDLC) comprendono la pianificazione,
l'analisi dei requisiti e il design del software. Sebbene queste attività non prevedano
ancora la scrittura effettiva del codice, rivestono comunque un ruolo importante
nella definizione delle proprietà di sicurezza dell'intero sistema, incluse le garanzie
di memory safety.

Una corretta gestione in queste fasi consente di identificare precocemente i rischi
legati all'uso della memoria e di predisporre adeguate strategie di mitigazione.
Ad esempio, nello sviluppo di applicazioni che gestiscono dati sensibili degli utenti,
come sistemi di autenticazione o gestione di informazioni personali, è importante
valutare fin dall'inizio quali requisiti di sicurezza siano necessari per prevenire
vulnerabilità legate alla memoria. Una pianificazione attenta consente di
selezionare strumenti, linguaggi e librerie che offrano protezioni adeguate,
riducendo il rischio di errori che potrebbero compromettere la riservatezza o l'integrità
dei dati.

In particolare, già in fase di pianificazione è possibile (e raccomandabile) condurre
un'analisi dei requisiti di sicurezza, esplicitando obiettivi orientati alla
memory safety. Analogamente, durante la fase di progettazione, è utile prevedere
misure difensive trasversali all'intero ciclo di vita, come l'uso di ambienti
\textit{sandbox} durante il deployment per limitare i danni derivanti da eventuali
vulnerabilità di memoria. Questi requisiti e decisioni guideranno poi tutte le
fasi successive dello sviluppo.

\subsection{Definizione di roadmap}
\label{sec:roadmap} Un'attività essenziale ma spesso trascurata è la definizione
di \textit{roadmap} da parte delle organizzazioni. Delineare una chiara
strategia di sviluppo e di adozione delle tecnologie è uno dei primi passi per
entrare nell'ottica di un approccio memory safe.

In quest'ottica, la CyberSecurity and Infrastructure Security Agency (CISA), in
collaborazione con agenzie internazionali, ha pubblicato un documento\cite{memory_safe_roadmaps}
che fornisce linee guida per i produttori di software su come creare e
pubblicare roadmap per la transizione verso linguaggi di programmazione che
garantiscano la sicurezza della memoria. Inoltre, viene fornita una serie di
mitigazioni applicabili a linguaggi non memory safe.

In particolare, secondo l'articolo, una roadmap efficace per la memory safety
dovrebbe includere:

\begin{itemize}
  \item \textbf{Fasi definite con date e risultati attesi}: decomporre il processo
    di transizione in fasi chiare e più piccole come:
    \begin{itemize}
      \item la valutazione dei linguaggi memory safe;

      \item la scrittura di un nuovo componente utilizzando un linguaggio memory
        safe ``by default'' per testarne l'efficacia;

      \item fare \textit{threat modeling} per identificare il codice memory
        unsafe più esposto a rischi;

      \item il refactoring del codice memory unsafe.
    \end{itemize}

  \item \textbf{Scadenza per l'uso esclusivo di linguaggi memory safe nei nuovi
    sistemi}: stabilire una data a partire dalla quale tutto il nuovo codice sarà
    scritto in un linguaggio memory safe.

  \item \textbf{Piano di formazione interna e integrazione nei workflow}: prevedere
    tempo e risorse per formare i team su linguaggi memory safe, debugging, integrazione
    nei processi di build e controllo della qualità.

  \item \textbf{Gestione delle dipendenze esterne (es. librerie C/C++)}: documentare
    una strategia per gestire le dipendenze legacy, in particolare librerie open
    source scritte in C/C++.

  \item \textbf{Piano di trasparenza}: aggiornamenti periodici (es. trimestrali/semestrali)
    per comunicare progressi, sfide e miglioramenti al SDLC, e ispirare altri ad
    adottare approcci simili.

  \item \textbf{Piano per il supporto ai CVE}: impegno a fornire CWE per il 100\%
    delle vulnerabilità pubblicate (CVE) e a offrire contesto tecnico utile per
    distinguere i difetti memory-unsafe da altri tipi di bug.
\end{itemize}

L'adozione di tali roadmap rappresenta un primo passo verso la realizzazione di sistemi
software più sicuri e resilienti, riducendo significativamente il rischio di
vulnerabilità legate alla gestione della memoria.

\subsection{Scelta del Linguaggio}
\label{sec:linguaggio}

La scelta del linguaggio di programmazione rappresenta una delle decisioni primarie
nelle fasi preliminari dello sviluppo software. Essa influenza in modo diretto
la qualità, la sicurezza e la manutenibilità del codice prodotto, ed è
particolarmente rilevante quando si vogliono prevenire vulnerabilità legate alla
memory safety.

Sebbene linguaggi come Rust siano stati progettati con l'obiettivo esplicito di
garantire la memory safety, non esiste una soluzione unica valida per tutti i contesti.
Ogni linguaggio presenta compromessi progettuali che ne influenzano l'idoneità
in scenari specifici.

Per esempio, C e C++ restano la scelta dominante nello sviluppo di firmware,
driver o sistemi embedded, dove è richiesto un controllo a basso livello dell'hardware
e delle risorse. Al contrario, Java e Kotlin trovano impiego in applicazioni mobili
e enterprise, grazie alla loro portabilità e al supporto dell'ecosistema JVM. Le
possibilità sono numerose, e ogni progetto potrebbe richiedere una combinazione
diversa di criteri funzionali e non funzionali.

Oltre alle caratteristiche tecniche, la scelta del linguaggio dipende spesso anche
da fattori pragmatici come il know-how del team, la disponibilità di tool e
librerie, i vincoli di tempo, o l'integrazione con sistemi esistenti e/o legacy.
Infatti, vale la pena sottolineare che, sebbene la scelta del linguaggio sia cruciale,
non è sempre possibile effettuarla. In alcuni casi, il linguaggio è vincolato
dalla codebase esistente, da requisiti di interoperabilità o da standard di settore.
Per tali situazioni, questa sezione della tesi può essere poco utile nella pratica,
ma può comunque fornire spunti per migliorare la sicurezza della memoria e dare
una panoramica dei linguaggi memory safe più diffusi.

\paragraph{Criteri di confronto}
Per offrire una panoramica concreta sulla relazione tra linguaggi di programmazione
e memory safety, è stata sviluppata la~\autoref{tab:linguaggi_memory_safety}, in
collaborazione con i ricercatori del Centro per la CyberSecurity di FBK\footnote{\url{https://cs.fbk.eu/}},
che riassume alcune proprietà rilevanti per la memory safety. I criteri
selezionati includono:
\begin{itemize}
  \item \textbf{Controllo dei limiti} (Bounds Check), per prevenire accessi fuori
    dai confini di array.

  \item \textbf{Gestione dei riferimenti nulli} (Null Safety), spesso causa di malfunzionamenti
    o blocchi.

  \item \textbf{Modello di proprietà e prestito} (Ownership/Borrowing), centrale
    in Rust per evitare accessi concorrenti e use-after-free.

  \item \textbf{Sicurezza di tipo} (Type Safety), per ridurre i rischi di casting
    improprio o accessi illegali alla memoria.

  \item \textbf{\textit{Zeroization} della memoria}, ovvero la capacità del linguaggio
    o delle librerie di pulire automaticamente dati sensibili dopo l'uso; utile
    in contesti crittografici o per la gestione di credenziali.

  \item \textbf{Integrità della memoria}, legata alla protezione da accessi fuori
    contesto o non autorizzati.

  \item \textbf{Curva di apprendimento} (Difficoltà), indicativa della facilità di
    adozione; linguaggi più accessibili tendono a ridurre il rischio di bug nei
    primi cicli di sviluppo; non legata alla sicurezza in modo diretto, ma utile
    per valutare l'impatto della scelta del linguaggio sul processo di sviluppo.

  \item \textbf{Ecosistema e librerie}, riferito alla disponibilità di librerie esterne
    mature e ben testate, che riducono la necessità di implementazioni custom
    potenzialmente vulnerabili.
\end{itemize}

Naturalmente esistono molti altri criteri di valutazione applicabili, sia di performance
che di sicurezza, che tuttavia esulano dallo scopo di questa tesi.

\setlength{\tabcolsep}{4pt}
\begin{table}[H]
  \small
  \centering
  \begin{threeparttable}
    \begin{tabular}{l|c|c|c|c|c|c|}
      \multicolumn{1}{l}{}           & \textbf{C/C++}                & \textbf{Java}                 & \textbf{Kotlin}               & \textbf{Python}               & \textbf{Rust}                    & \textbf{Go}                      \\
      \hline
      \textbf{Bounds Check}          & \cellcolor{red!20}No          & \cellcolor{green!20}Sì        & \cellcolor{green!20}Sì        & \cellcolor{green!20}Sì        & \cellcolor{green!20}Sì           & \cellcolor{green!20}Sì           \\
      \textbf{Null Safety}           & \cellcolor{red!20}No          & \cellcolor{red!20}No          & \cellcolor{green!20}Sì        & \cellcolor{yellow!20}Parziale & \cellcolor{green!20}Sì           & \cellcolor{green!20}Sì           \\
      \textbf{Ownership/Borrowing}   & \cellcolor{red!20}No          & \cellcolor{red!20}No          & \cellcolor{red!20}No          & \cellcolor{red!20}No          & \cellcolor{green!20}Sì           & \cellcolor{red!20}No             \\
      \textbf{Type Safety}           & \cellcolor{yellow!20}Parziale & \cellcolor{yellow!20}Parziale & \cellcolor{yellow!20}Parziale & \cellcolor{green!20}Sì        & \cellcolor{green!20}Sì           & \cellcolor{green!20}Sì           \\
      \textbf{Memory Zeroization}    & \cellcolor{red!20}No          & \cellcolor{red!20}No          & \cellcolor{red!20}No          & \cellcolor{red!20}No          & \cellcolor{green!20}Sì\tnote{a}  & \cellcolor{red!20}No             \\
      \textbf{Memory Integrity}      & \cellcolor{red!20}No          & \cellcolor{yellow!20}Parziale & \cellcolor{yellow!20}Parziale & \cellcolor{red!20}No          & \cellcolor{green!20}Sì           & \cellcolor{green!20}Sì           \\
      \hline
      \textbf{Difficoltà}\tnote{b}   & \cellcolor{red!20}Alta        & \cellcolor{yellow!20}Media    & \cellcolor{yellow!20}Media    & \cellcolor{green!20}Bassa     & \cellcolor{red!20}Alta           & \cellcolor{green!20}Bassa        \\
      \textbf{Ecosistema e Librerie} & \cellcolor{green!20}Vasto     & \cellcolor{green!20}Vasto     & \cellcolor{green!20}Vasto     & \cellcolor{green!20}Vasto     & \cellcolor{yellow!20}In crescita & \cellcolor{yellow!20}In crescita \\
      \hline
    \end{tabular}
    \begin{tablenotes}
      \footnotesize \item[a] Supportata tramite il crate \texttt{zeroize}, mantenuto
      dal team \texttt{RustCrypto} sensibili. \item[b] Fonte:
      \url{https://www.learningcardano.com/...}, ultimo accesso: 13 maggio 2025.
    \end{tablenotes}
  \end{threeparttable}
  \caption{Confronto tra linguaggi delle caratteristiche legate alla memory
  safety}
  \label{tab:linguaggi_memory_safety}
\end{table}

\footnotetext[2]{Supportata tramite il crate \texttt{zeroize}, mantenuto dal
team \texttt{RustCrypto} e ampiamente usato per la gestione sicura di dati sensibili}

\footnotetext[3]{Fonte: \url{https://www.learningcardano.com/how-long-does-it-take-to-learn-the-most-popular-programming-languages/},
ultimo accesso: 13 maggio 2025}

\noindent

Come si può osservare dalla tabella, emergono pattern interessanti che riflettono
le diverse filosofie di design dei linguaggi analizzati.

\begin{itemize}
  \item \textbf{Rust e Go} si distinguono come i linguaggi più orientati alla
    memory safety, offrendo protezioni integrate estensive. Rust, in particolare,
    rappresenta l'approccio più radicale con il suo sistema di ownership/borrowing
    unico, che elimina alla radice molte categorie di vulnerabilità, e il
    supporto per la zeroization della memoria. Go, pur non implementando
    ownership/borrowing, compensa con un garbage collector efficace e controlli runtime
    robusti. Tuttavia, entrambi presentano ecosistemi ancora in fase di maturazione
    rispetto ai linguaggi più consolidati.

  \item \textbf{Java, Kotlin e Python} occupano una posizione intermedia, beneficiando
    di ecosistemi maturi e ricchi di librerie, ma con approcci diversificati
    alla sicurezza. Kotlin emerge come il più avanzato del gruppo grazie alla null
    safety nativa, che elimina una fonte significativa di errori runtime. Java,
    nonostante la sua maturità, mantiene alcuni compromessi storici come la possibilità
    di \texttt{NullPointerException}. Python offre un buon bilanciamento generale,
    ma la sua natura dinamica può introdurre vulnerabilità in contesti specifici.

  \item \textbf{C e C++} rappresentano l'estremo opposto, con protezioni minime integrate
    ma controllo completo delle risorse di sistema. Questo li rende
    indispensabili in contesti dove performance e controllo hardware sono
    prioritari, ma richiede competenze specifiche e processi di sviluppo rigorosi
    per mitigare i rischi intrinseci.
\end{itemize}

La scelta finale del linguaggio dovrebbe quindi considerare il \textit{trade-off}
tra controllo, performance, sicurezza e produttività, valutando attentamente il
contesto specifico del progetto, le competenze del team e i requisiti di sicurezza.
In molti casi, un approccio ibrido che combini linguaggi diversi per componenti specifiche
può rappresentare la soluzione ottimale.