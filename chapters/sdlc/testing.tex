\section{Testing}
\label{sec:testing}

Pur avendo trattato varie mitigazioni applicabili nella fase dello sviluppo del software
(\autoref{sec:development}), queste si basano sul contributo dello sviluppatore
che può commettere errori (best practices e uso di librerie), o su strumenti che
non segnalano tutti i problemi possibili o che producono falsi positivi (analisi
statica). Vista l'inclinazione agli errori delle metodologie proposte, queste potrebbero
non essere sufficienti a garantire la sicurezza della memoria ed è quindi fondamentale
soffermarsi anche sulla parte di testing.

Il testing è l'altra fase principale del ciclo di vita del software dopo lo sviluppo.
In generale, il fine dei test è quello di rilevare \textit{unexpected behavior}
generici, che possono variare da errori di formattazione presenti negli output fino
a crash o vulnerabilità di sicurezza del software. Nel contesto di questo
documento, l'attenzione è rivolta alla memory safety, e quindi saranno
presentati strumenti e tecniche sviluppati appositamente per rilevare
problematiche legate alla memory safety.

\subsection{Analisi Dinamica}
\label{sec:analisi-dinamica}

Una delle tecniche più comuni per il testing del software dal punto di vista della
memoria è la cosiddetta \textit{analisi dinamica}. A differenza dell'analisi statica,
che si basa sull'ispezione del codice sorgente senza eseguirlo, l'analisi dinamica
agisce a \textit{runtime}, monitorando il comportamento del programma durante l'esecuzione.
Questo approccio permette di rilevare errori e vulnerabilità che emergono solo
in presenza di determinati dati di input, condizioni ambientali o flussi di esecuzione
specifici, spesso difficili da prevedere a priori.

Nel contesto della memory safety, esistono due principali categorie di strumenti
di analisi dinamica, in base al momento e al livello in cui vengono applicati:

\paragraph{Strumenti basati su compilazione.}
Alcuni strumenti operano direttamente a livello di compilazione, modificando il
codice sorgente o il codice intermedio per inserire controlli di sicurezza
aggiuntivi. Due esempi noti sono:

\begin{itemize}
  \item \textbf{AddressSanitizer (ASan):} uno strumento integrato nei
    compilatori \texttt{clang} e \texttt{gcc}, in grado di rilevare errori comuni
    come buffer overflow, use-after-free, e heap/stack out-of-bounds. ASan
    aggiunge al codice binario delle istruzioni extra per controllare la
    validità degli accessi in memoria durante l'esecuzione.

  \item \textbf{MemorySanitizer (MSan):} simile ad ASan, ma focalizzato sull'individuazione
    dell'uso di variabili non inizializzate. Questo tipo di errore, pur essendo più
    difficile da individuare, può portare a comportamenti non deterministici e
    vulnerabilità.
\end{itemize}

Questi strumenti offrono una buona copertura in fase di testing, ma comportano un
significativo overhead in termini di prestazioni e dimensione del binario,
motivo per cui non sono adatti all'uso in ambienti di produzione.

\paragraph{Strumenti basati sul binario.}
Altri strumenti operano a livello del binario compilato, senza necessità di
modificare il sorgente. Tra i più noti in questo ambito troviamo \textbf{Valgrind},
un framework di strumentazione dinamica che esegue il programma su una CPU virtuale,
intercettando ogni accesso alla memoria. Il suo modulo \texttt{Memcheck} è in
grado di rilevare accessi non inizializzati, leak di memoria, e violazioni di
accesso. Poiché funziona a livello binario, è particolarmente utile in scenari
in cui il codice sorgente non è disponibile.

Gli strumenti binari tendono ad essere più generali ma anche più lenti, in quanto
emulano o virtualizzano il comportamento del programma, causando un
rallentamento significativo dell'esecuzione (fino a 10--50 volte più lento)\footnote{Manuale
Valgrind: \url{https://valgrind.org/docs/manual/manual-core.html\#manual-core.whatdoes}}.

\paragraph{Limiti e considerazioni pratiche.}
Nonostante la loro efficacia, gli strumenti di analisi dinamica presentano
alcune limitazioni:

\begin{itemize}
  \item Il loro funzionamento dipende fortemente dalla qualità e dalla copertura
    dei test: eseguono controlli solo sul codice realmente percorso durante i
    test. Infatti, se il codice contiene ramificazioni condizionali, queste potrebbero
    non essere raggiunte durante l'esecuzione e quindi non essere segnalate se contengono
    bug o vulnerabilità.

  \item Introdurre strumentazione dinamica può alterare il comportamento temporale
    del programma, rendendo alcuni bug meno riproducibili.

  \item Per via dell'elevato overhead computazionale, questi strumenti sono
    generalmente utilizzati esclusivamente in fase di \textit{testing} e non in ambienti
    di produzione.

  \item Il comportamento del software potrebbe essere influenzato dall'ambiente in
    cui viene eseguito (es. architettura HW, OS, versioni del runtime, ecc.). Per questo motivo, è
    importante testare il software in ambienti quanto più simili a quello di produzione.
\end{itemize}

Nonostante ciò, l'analisi dinamica rappresenta un complemento fondamentale all'analisi
statica, permettendo di coprire una gamma più ampia di possibili vulnerabilità legate
alla memoria. Nell'ambito di un approccio difensivo multilivello, l'impiego congiunto
di più tecniche di testing aumenta significativamente la probabilità di intercettare
errori prima del rilascio del software.

\subsection{Fuzzing}
\label{sec:fuzzing}

Il \textit{fuzzing} (o fuzz testing) è una tecnica di testing automatizzato che consiste
nel generare una grande quantità di input, spesso casuali o leggermente mutati a
partire da casi validi, con l'obiettivo di far emergere comportamenti anomali,
crash o vulnerabilità nel programma in esecuzione. Sebbene non sia stato concepito
esclusivamente per individuare problemi di gestione della memoria, il fuzzing è
particolarmente efficace nel rivelare condizioni di \textit{undefined behavior},
spesso sintomatiche di bug memory-unsafe.

\smallskip
Ci sono diversi tipi di fuzzing, ma in generale si possono distinguere due categorie
principali:

\begin{itemize}
  \item \textbf{Fuzzing basato su generazione:} in questo approccio, gli input vengono
    generati in modo casuale o semi-casuale, spesso a partire da un insieme di
    casi di test validi. Questo metodo è utile per esplorare ampie porzioni dello
    spazio degli input, ma può risultare inefficace nel rilevare vulnerabilità
    specifiche.

  \item \textbf{Fuzzing basato su mutazione:} qui gli input validi vengono
    modificati (mutati) per creare nuovi casi di test. Questo approccio è particolarmente
    efficace per individuare vulnerabilità legate a formati di file o dati strutturati,
    poiché sfrutta la conoscenza preesistente di un input valido.
\end{itemize}

Gran parte dei tool di fuzzing non identificano direttamente gli errori di
memoria, ma si limitano a segnalare i crash o le anomalie comportamentali,
associandoli agli input che li hanno provocati. Per rendere l'analisi più mirata
ed efficace, i cosiddetti \textit{fuzzers} vengono comunemente usati in combinazione
con strumenti di analisi dinamica come i \textit{sanitizers}, in grado di diagnosticare
in modo più preciso gli errori sottostanti.

Questa mancanza ha portato a sviluppare anche strumenti di fuzzing \textit{multi-purpose},
che combinano le funzionalità di fuzzing con altre tecniche di testing. Un
esempio è AFL++\cite{afl_plus_plus} che, tra le sue funzionalità, include anche un
AddressSanitizer (QASan\cite{qasan}) integrato, in grado di rilevare errori di memoria
durante il fuzzing.

Il fuzzing descritto finora è chiamato \textit{black-box fuzzing}, in quanto non
richiede alcuna conoscenza del codice sorgente o della logica interna del programma
sottoposto a test. Esistono anche approcci \textit{white-box} che, invece, analizzano
il codice sorgente per generare input più mirati e specifici, che riescono ad
esplorare \textit{"l'albero di esecuzione"} del programma in modo più efficiente.
Un esempio è la cosiddetta \textit{Dynamic Symbolic Execution} (DSE), che
consiste nell'eseguire il programma con input simbolici, riuscendo ad esplorare tutte
le possibili esecuzioni e a generare input che portano a percorsi specifici.

\paragraph{Limiti del fuzzing.}
Nonostante l'efficacia del fuzzing nel rilevare vulnerabilità, questa tecnica presenta
alcune limitazioni significative:

\begin{itemize}
  \item \textbf{Configurazione complessa:} L'impostazione corretta degli
    strumenti di fuzzing richiede competenze specifiche, soprattutto quando si tratta
    di definire formati di input validi o vincoli per applicazioni complesse.

  \item \textbf{Consumo di risorse:} Il fuzzing efficace richiede l'esecuzione
    di milioni di test, con conseguente elevato consumo di tempo e risorse computazionali.
    Questo può rendere il processo proibitivo per progetti con risorse limitate.

  \item \textbf{Esplosione combinatoria:} Negli approcci più sofisticati come la
    DSE, l'esplorazione di tutti i possibili percorsi di esecuzione può crescere
    esponenzialmente, rendendo impossibile una copertura completa per
    applicazioni di medie o grandi dimensioni.

  \item \textbf{Coverage limitata:} Nonostante gli avanzamenti nelle tecniche di
    generazione degli input, il fuzzing potrebbe non raggiungere porzioni di codice
    che richiedono condizioni di ingresso molto specifiche.
\end{itemize}

Per mitigare questi problemi, un approccio pragmatico consiste nell'integrare il
fuzzing in pipeline di CI/CD continuative, focalizzandosi inizialmente sui
componenti più critici o storicamente vulnerabili del software.

\noindent

\paragraph{\textit{Nota:}}
Le tecniche e i tool di analisi dinamica e fuzzing, sono spesso associati al codice
C o C++, ma sono applicabili anche ad altri linguaggi: infatti Rust, pur essendo
uno dei linguaggi più sicuri dal punto di vista della memory safety, supporta il
cosiddetto \textit{Unsafe Rust}, che permette un maggiore controllo a basso livello,
ma introduce anche la possibilità di errori di memoria. Per questo motivo, tool
come Valgrind e AddressSanitizer sono stati adattati per essere utilizzati anche
con Rust.\cite{valgrind_rust}\cite{rust_manual_san}