\section{Testing}
\label{sec:testing}

Pur avendo trattato varie mitigazioni applicabili nella fase dello sviluppo del software
(\autoref{sec:development}), queste si basano sul contributo dello sviluppatore
che può commettere errori (best practices e uso di librerie) o su strumenti che
possono produrre falsi positivi o falsi negativi (analisi statica). Vista l'inclinazione
agli errori, le metodologie proposte potrebbero non essere sufficienti a
garantire la sicurezza della memoria ed è quindi fondamentale soffermarsi anche
sulla parte di testing.

Il testing è l'altra fase principale del ciclo di vita del software dopo lo
sviluppo. In generale, il fine dei test è quello di rilevare \textit{unexpected
behavior} generici, che possono variare da errori di formattazione presenti
negli output fino a crash o vulnerabilità di sicurezza del software. Nel contesto
di questo documento, l'attenzione è rivolta alla memory safety, e quindi saranno
presentati strumenti e tecniche sviluppati appositamente per rilevare problematiche
legate alla memory safety.

\subsection{Analisi Dinamica}
\label{sec:analisi-dinamica}

\subsection{Fuzzing}
\label{sec:fuzzing}