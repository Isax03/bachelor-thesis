\section{Distribuzione}
\label{sec:deployment}

In questa sezione verranno trattate diverse tecniche sviluppate per proteggere i
software durante la loro esecuzione. Queste mitigazioni possono essere applicate
in fase di compilazione, a runtime o essere integrate direttamente a livello di sistema
operativo o hardware.

Il \textit{deployment} rappresenta una delle fasi più critiche dal punto di
vista della sicurezza, in quanto è il momento in cui il software può essere esposto
agli attaccanti e, anche avendo adottato tutte le misure di sicurezza descritte
nei capitoli precedenti, è possibile che il software possa comunque essere vulnerabile
a exploit in fase di esecuzione.

È quindi consigliabile applicare le mitigazioni descritte in questa sezione,
soprattutto in modo combinato, per ridurre al minimo le possibilità di attacco.

\subsection{Protezioni a livello di memoria}
\label{sec:memory-protection}

Le protezioni a livello di memoria sono tecniche fondamentali per prevenire gli
attacchi che sfruttano vulnerabilità come buffer overflow, use-after-free e altre
forme di corruzione della memoria. Queste tecniche mirano a rendere più
difficile per un attaccante eseguire codice arbitrario o manipolare il flusso di
esecuzione di un programma attraverso la corruzione dei dati.

Queste mitigazioni, implementate sia a livello di compilatore che di sistema operativo,
rappresentano una prima linea di difesa essenziale contro molte classi di
attacchi. I paragrafi seguenti esplorano le principali mitigazioni implementate in
questo ambito.

\paragraph{Address Space Layout Randomization (ASLR)}
L'ASLR è una tecnica difensiva che mira a rendere imprevedibile la disposizione
in memoria delle principali aree di un processo, come stack, heap e librerie
condivise, al fine di ostacolare attacchi che si basano su indirizzi statici, e
quindi prevedibili.

Nonostante introduca entropia a ogni avvio del processo, la sua efficacia risulta
limitata sulle architetture a 32 bit. Lo studio di Shacham et al.\cite{aslr_effectiveness}
dimostra che in questi sistemi l'aleatorietà degli indirizzi per alcune regioni
di memoria può ridursi a soli 16 bit, rendendo possibili attacchi \textit{brute-force}
in grado di compromettere le difese in pochi minuti.

Per questo motivo, è cruciale valutare l'architettura target del software: l'ASLR
offre protezione più robusta sui sistemi a 64 bit, dove lo spazio d'indirizzamento
più ampio consente una dispersione significativa delle aree memorizzate. La
tecnica rimane comunque raccomandata come misura fondamentale, specialmente se integrata
con altre contromisure discusse in questa sezione, per aumentare sostanzialmente
la resilienza contro exploit noti.

\paragraph{Data Execution Prevention (DEP)}
La DEP è una misura di sicurezza che consiste nel contrassegnare specifiche aree
di memoria, come lo stack e l'heap, come non eseguibili. L'obiettivo è impedire
l'esecuzione di codice arbitrario iniettato in regioni di memoria destinate
esclusivamente alla conservazione di dati.

Questa protezione è particolarmente efficace contro attacchi come i buffer
overflow, nei quali un attaccante tenta di sovrascrivere l'indirizzo di ritorno di
una funzione per eseguire codice malevolo. Se la memoria che ospita tale codice
è marcata come non eseguibile, il tentativo di esecuzione fallisce, riducendo
significativamente il rischio di exploit.

Tuttavia, analogamente all'ASLR, anche la Data Execution Prevention è stata
aggirata nel tempo tramite tecniche avanzate come il Return Oriented Programming
(ROP) e il Jump Oriented Programming (JOP). Queste strategie non richiedono l'esecuzione
di codice iniettato, ma riutilizzano frammenti di codice legittimo già presenti
in memoria per comporre un payload malevolo. Di conseguenza, la DEP deve essere
considerata come una misura di sicurezza complementare, efficace solo se integrata
in una strategia difensiva più ampia.

\paragraph{Stack Canary}
Gli stack canaries sono una tecnica di protezione software mirata a prevenire
gli attacchi basati sullo stack buffer overflow. Il nome deriva dall'analogia con
i canarini nelle miniere di carbone, utilizzati come sentinelle per rilevare gas
pericolosi: in questo caso, i canaries sono valori speciali inseriti tra le
variabili locali e l'indirizzo di ritorno nella struttura dello stack.

Durante l'esecuzione, prima di effettuare il ritorno da una funzione, il
programma verifica che il valore del canary non sia stato modificato. Se il valore
è stato alterato, indice di un probabile overflow dello stack, il processo viene
terminato o vengono attivate contromisure, impedendo così che un attaccante
possa modificare l'indirizzo di ritorno e quindi controllare il flusso di
esecuzione.

Questa protezione è efficace nel mitigare un'ampia classe di vulnerabilità
legate allo stack overflow, ma presenta alcune limitazioni: non protegge infatti
da overflow su heap o da altre forme di corruzione della memoria, né è efficace contro
attacchi che sfruttano tecniche più sofisticate come il Return Oriented Programming
(ROP), che non modificano direttamente il canary ma riutilizzano codice
legittimo.

Gli stack canaries sono generalmente implementati dal compilatore e possono
variare in complessità, dal semplice valore fisso a valori casuali generati all'avvio
del processo, aumentando così la difficoltà per l'attaccante di prevederli e
aggirarli.

\subsection{Controlli di integrità dell'esecuzione}
\label{sec:execution-integrity} Le tecniche presentate in questa sezione mirano
a garantire che il codice eseguito sia quello previsto e che non sia stato
manomesso o alterato in alcun modo.

\paragraph{Control Flow Integrity (CFI)}
è una tecnica avanzata di protezione che mira a impedire che un attaccante possa
alterare il flusso di controllo di un programma attraverso vulnerabilità come i buffer
overflow o gli use-after-free. L'idea alla base della CFI è quella di
assicurarsi che durante l'esecuzione, un programma possa compiere solo i salti di
controllo (es. chiamate indirette, ritorni di funzione) che sono stati previsti e
autorizzati in fase di compilazione.

Per ottenere questo obiettivo, la CFI è implementata principalmente a livello di
compilatore, che effettua le seguenti operazioni:
\begin{itemize}
  \item Analizza il grafo di controllo del programma

  \item Genera metadati che definiscono i salti validi e consentiti

  \item Inserisce controlli a runtime nel codice per verificare, prima di ogni
    salto, che la destinazione sia legittima
\end{itemize}

In questo modo, anche vettori di attacco come ROP e JOP, che si basano sull'alterazione
del flusso di controllo, possono essere mitigati e resi inefficaci.

In generale esistono due tipi principali di CFI:
\begin{itemize}
  \item \textbf{Fine-grained}: impone vincoli più specifici e dettagliati sui salti
    consentiti. Questo approccio migliora senz'altro la sicurezza, ma richiede un
    costo computazionale maggiore

  \item \textbf{Coarse-grained}: consente salti più generali e meno restrittivi.
    Questo approccio è più veloce e meno costoso, ma offre una protezione inferiore
    rispetto al fine-grained
\end{itemize}

Anche per la Control Flow Integrity gli svantaggi non mancano: la CFI può introdurre
un overhead significativo in termini di prestazioni e potrebbe essere aggirata se
i controlli sono troppo generici o se l'attaccante riesce a manipolare i metadati
di controllo. Tuttavia, se implementata correttamente, la CFI rappresenta una
misura di sicurezza efficace contro molte classi di attacchi.

\paragraph{Memory Tagging Extension (MTE)\protect\footnote{https://developer.arm.com/documentation/108035/0100/Introduction-to-the-Memory-Tagging-Extension}}
Questa è la prima delle due funzionalità di sicurezza \textit{"architecture-dependent"}
presenti in questa sezione. La MTE infatti, è una feature hardware introdotta da
Google e ARM (v8.5+) che mira a rilevare e prevenire errori di accesso alla
memoria, come buffer overflow e use-after-free, attraverso l'uso di tag di
memoria.

Parlando ad alto livello, la Memory Tagging Extension funziona associando un tag
a ogni area di memoria allocata. Questi tag sono utilizzati per identificare la
proprietà e la validità dell'area di memoria. Quando un programma tenta di accedere
a un'area di memoria, il processore verifica se il tag associato all'area
corrisponde al tag previsto per quell'operazione. Se i tag non corrispondono, il
processore genera un'eccezione, impedendo l'accesso non autorizzato e riducendo il
rischio di exploit.

La MTE può lavorare in 3 modalità differenti:
\begin{itemize}
  \item \textbf{Synchronous}: il processore genera un'eccezione quando viene rilevato
    un accesso non autorizzato e termina il programma. Questa modalità è
    ottimizzata per la ricerca di bug ma introduce molto overhead. Utile per la fase
    di testing, ma non raccomandata per l'uso in produzione.

  \item \textbf{Asynchronous}: il processore continua l'esecuzione quando viene rilevato
    un accesso non autorizzato, rimandando la generazione dell'eccezione fino al
    più vicino ingresso nel kernel e solo a quel punto, il processo viene terminato.
    Questa modalità è ottimizzata per le prestazioni piuttosto che per l'accuratezza
    dei report di bug, ed è raccomandata per l'uso in produzione su codice ben testato.

  \item \textbf{Asymmetric}(v8.7-A+): questa modalità è una combinazione delle
    due precedenti. Se l'accesso non autorizzato è causato da una lettura, il
    processore agisce in maniera "Synchronous", mentre se l'accesso non autorizzato
    è causato da una scrittura, il processore agisce in maniera "Asynchronous".
\end{itemize}

\paragraph{Pointer Authentication Code (PAC)\protect\footnote{https://developer.arm.com/documentation/109576/0100/Pointer-Authentication-Code/Introduction-to-PAC}}
La seconda funzionalità di sicurezza \textit{"architecture-dependent"} è il PAC.
Anche essa sviluppata da ARM (v8.3-A+), PAC è una tecnologia nata per prevenire in
modo efficace attacchi di tipo ROP o buffer overflow.

PAC si basa sull'idea di associare una firma crittografica a ogni puntatore utilizzato
nel programma. Questa firma viene calcolata in modo tale che qualsiasi modifica
al puntatore stesso o alla memoria a cui punta, renda impossibile verificare la firma.
In questo modo, se un attaccante tenta di manipolare un puntatore per eseguire codice
malevolo, la firma non corrisponderà più e il processore genererà un'eccezione, impedendo
l'esecuzione del codice non autorizzato.

PAC presenta due fasi principali:
\begin{itemize}
  \item \textbf{Firma}: quando un puntatore viene creato, il processore calcola
    una firma basata sul valore del puntatore, su una chiave segreta mantenuta nel
    processore e su un contesto opzionale(es. il valore dello stack pointer). La
    firma viene poi salvata nei bit inutilizzati del puntatore stesso per
    ridurre i tempi di verifica.

  \item \textbf{Verifica}: quando un puntatore viene utilizzato, il processore
    verifica la firma associata al puntatore. Se la firma è valida, l'accesso
    alla memoria è consentito; in caso contrario, viene generata un'eccezione.
\end{itemize}

\bigskip
\noindent
Entrambe le tecnologie di ARM, MTE e PAC, sono hardware-assisted e l'hoverhead introdotto
è molto basso, rendendole adatte per l'uso in produzione. Tuttavia, è importante
notare che queste tecnologie sono specifiche per l'architettura ARM e quindi non
sempre disponibili come mitigazioni per rafforzare la memory safety.

\subsection{Isolamento del software}
\label{sec:isolation}

Il concetto di \textit{isolamento} si riferisce alla separazione tra componenti
software con l'obiettivo di limitare l'impatto di eventuali compromissioni. L'idea
alla base è quella di confinare il danno in un'area ben definita, impedendo che
un exploit possa propagarsi ad altri componenti del sistema.

In ambito memory safety, l'isolamento rappresenta una misura efficace per mitigare
vulnerabilità che, pur essendo presenti, non riescono a compromettere l'intero sistema
grazie all'introduzione di barriere architetturali o software.

Le principali tecniche di isolamento includono:
\begin{itemize}
  \item \textbf{Sandboxing}: consiste nell'eseguire il software all'interno di un
    ambiente controllato e isolato, con accesso limitato a risorse di sistema e
    dati sensibili. Le sandbox possono essere implementate a livello di sistema operativo
    (es. \texttt{seccomp}, \texttt{AppArmor}) o tramite tecnologie di virtualizzazione
    leggera.

  \item \textbf{Containerizzazione}: fornisce isolamento a livello del sistema operativo,
    separando le applicazioni e le loro dipendenze in container indipendenti. Sebbene
    più leggeri rispetto alle macchine virtuali, i container condividono il
    kernel dell'host e richiedono un'attenta configurazione per evitare
    escalation di privilegi.

  \item \textbf{Virtualizzazione}: permette la creazione di macchine virtuali (VM)
    che emulano completamente l'hardware, consentendo l'esecuzione di più
    sistemi operativi isolati su un singolo host fisico. Le VM offrono un isolamento
    più forte rispetto ai container, ma introducono un maggiore overhead in
    termini di risorse e prestazioni.
\end{itemize}

Un esempio concreto dell'efficacia dell'isolamento lo troviamo nel caso reale di
WebAudio in Google Chrome, già menzionato nella
\autoref{sec:vulnerability_types}. L'uso della sandbox del browser impediva all'attaccante
di ottenere l'accesso completo al sistema, confinando l'esecuzione di codice
arbitrario all'interno dell'ambiente "sandboxed".\cite{webaudio_uaf}

È importante sottolineare che queste tecnologie non sono state progettate specificamente
per garantire la memory safety, ma la loro architettura consente di limitare l'impatto
di eventuali vulnerabilità isolando i processi e le risorse. Ad esempio, la
containerizzazione è nata principalmente per esigenze di portabilità e
scalabilità delle applicazioni, ma fornisce anche un utile livello di confinamento.

Tuttavia, l'isolamento non elimina completamente le vulnerabilità, e introduce un
costo aggiuntivo in termini di complessità gestionale e prestazioni. Pertanto, la
scelta della tecnica più appropriata deve basarsi su una valutazione del contesto
applicativo e dei requisiti di sicurezza specifici.