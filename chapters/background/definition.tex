\section{Definizione di memory safety}
\label{sec:memory_safety}

Il concetto di memory safety non dispone di una definizione formale univoca,
generalmente riconosciuta da tutta la comunità scientifica. In letteratura si trovano,
per la maggior parte, definizioni di carattere più pratico, che si concentrano
sull'assenza di determinati tipi di bug o comportamenti indesiderati legati alla
gestione della memoria.

Di seguito, vengono presentate sia una definizione formale di memory safety che una
definizione intuitiva derivante da aspetti più pratici, al fine di fornire una
visione completa del concetto e delle sue implicazioni.

\paragraph{Definizione formale}

Una delle definizioni formali di memory safety più accreditate è quella proposta
da A. Azevedo de Amorim et al.~\cite{meaning_memory_safety}, che si basa su una formalizzazione
rigorosa del concetto, attraverso un framework matematico basato su semantica
operazionale e strutture formali per ragionare sulla sicurezza della memoria in
modo verificabile.

Nel modello proposto, un programma è considerato \textbf{memory safe} se, durante
la sua esecuzione, preserva le seguenti proprietà:
\begin{enumerate}
  \item \textbf{Isolamento spaziale e temporale}:
    \begin{itemize}
      \item Le operazioni su una regione di memoria non influenzano dati al di fuori
        del proprio ``footprint'' (porzione accessibile).

        \textit{Per esempio, gli accessi out-of-bound violano questa proprietà,
        poiché le porzioni di memoria fuori dai limiti di un array non fanno
        parte del suo footprint}

      \item I \textit{Frame Theorems}, sviluppati e dimostrati in~\cite{meaning_memory_safety},
        garantiscono che l'estensione dello heap iniziale con blocchi non
        raggiungibili non alteri il comportamento del programma.
    \end{itemize}

  \item \textbf{Non-interferenza}:
    \begin{itemize}
      \item Nel modello ideale proposto, la memoria non raggiungibile non può
        essere osservata né alterata dall'esecuzione del programma: ciò garantisce
        sia \textit{integrità} (impossibilità di modificare dati isolati), sia \textit{segretezza}
        (impossibilità di dedurne l'esistenza).
    \end{itemize}

  \item \textbf{Contenimento degli errori}:
    \begin{itemize}
      \item Accessi illegali (es. dereferenziazione di puntatori invalidi)
        terminano l'esecuzione in modo prevedibile con un errore esplicito, evitando
        comportamenti indefiniti.
    \end{itemize}
\end{enumerate}

Questo modello consente di dimostrare proprietà come la protezione tra moduli e
l'assenza di comportamenti indefiniti. Conseguentemente, esso fornisce anche un
solido framework matematico per specificare e dimostrare le garanzie di
sicurezza implementate da un sistema (come proprietà del \textit{type system} o
del \textit{memory model}). Tale formalizzazione guida inoltre lo sviluppo di strumenti
verificati per l'enforcement della memory safety, come i monitor hardware/software
descritti in~\cite{meaning_memory_safety}.

Tuttavia, questa definizione si basa su ipotesi idealizzate, come memoria
illimitata, assenza di side-channel\footnote{Canale d'informazione non
intenzionale che permette di estrarre dati sensibili attraverso l'osservazione
di caratteristiche fisiche o temporali del sistema (es. consumo energetico, timing
delle operazioni, emissioni elettromagnetiche) piuttosto che attraverso l'accesso
diretto ai dati.} sfruttabili dagli attaccanti e assenza di aritmetica sui
puntatori o di cast a interi (che permettono la manipolazione diretta degli indirizzi),
che lo rendono solo parzialmente allineato ai linguaggi e agli ambienti reali.
Per questo motivo una definizione più intuitiva e pratica può essere utile per
comprendere il concetto di memory safety in contesti reali.

\paragraph{Definizione pratica}

Dal punto di vista pratico, la memory safety viene comunemente intesa come la capacità
di un programma di evitare vulnerabilità o comportamenti errati legati alla gestione
della memoria.

Una definizione operativa in questo senso è fornita da D. Dhurjati et al.~\cite{memory_safety_without_runtime_checks},
che descrivono un programma memory safe come un'entità software che:
\begin{itemize}
  \item non accede a locazioni di memoria al di fuori dello spazio di indirizzamento
    a esso allocato;

  \item non esegue istruzioni al di fuori del proprio codice compilato e ``linkato''
    (ovvero del codice risultante dal collegamento con le librerie statiche e
    dinamiche)

  \item rispetta restrizioni di tipo sulle operazioni, vincoli sulle conversioni
    di tipo e accessi nei limiti degli array.
\end{itemize}

Sulla base di questi principi, la memory safety può essere vista come la capacità
di garantire che tutte le operazioni di accesso alla memoria (lettura, scrittura,
esecuzione) avvengano in modo controllato e conforme alle intenzioni del programma.
In altre parole, essa protegge il programma da errori come \textit{buffer
overflow}, \textit{dangling pointers} e \textit{use after free}, che possono
causare violazioni di integrità o problemi di sicurezza.

Questa visione, pur derivando da un'analisi formale, si allinea con un'intuizione
più diffusa: un programma è memory safe se è protetto da una serie di bug comuni
e pericolosi relativi agli accessi in memoria. Questa prospettiva è spesso
adottata in contesti divulgativi, dove la memory safety è intesa come l'assenza di
specifiche classi di vulnerabilità anziché come una proprietà rigorosamente formale.

% Memory Safety Continuum %
\paragraph{Memory safety come spettro continuo}

Il divario tra modelli teorici e implementazioni reali suggerisce che la memory safety
non sia una proprietà assoluta o binaria, ma una condizione che può essere
raggiunta in misura variabile in contesti di applicazione reali.

Mentre la definizione formale fornisce garanzie assolute all'interno del proprio
modello matematico, le limitazioni e le idealizzazioni di tale modello rendono difficile
il raggiungimento di una memory safety completa nei sistemi pratici.

Questa idea viene esplicitamente sviluppata dalla \textit{Open Source Security
Foundation} (OpenSSF)~\cite{memory_safety_continuum}, che propone di interpretare
la memory safety come uno spettro continuo. In tale visione, sistemi e linguaggi
possono offrire livelli differenti di protezione, a seconda delle garanzie
fornite e delle mitigazioni adottate.

Questo approccio sottolinea come la memory safety possa essere rafforzata
progressivamente, piuttosto che garantita in modo assoluto.