\section{Tipi di vulnerabilità}
\label{sec:vulnerability_types}

Sebbene esistano diverse tassonomie e classificazioni per le vulnerabilità legate
alla memoria, come quelle proposte da enti come OpenSSF\cite{memory_safety_continuum_definition}
o MITRE (CWE-1399\cite{cwe_1399}), la struttura illustrata di seguito è stata sviluppata
personalmente. Di seguito infatti, vengono presentate le principali classi di debolezze
secondo una visione concettuale che le organizza in base alla natura dell'accesso
errato alla memoria. Questa categorizzazione, pur basandosi su definizioni consolidate,
privilegia una prospettiva operativa che distingue gli errori spaziali, temporali
e di inizializzazione, oltre ai memory leak.

\paragraph{Accessi spaziali: violazione dei confini}
\label{sec:spatial}

Comprendono tutte le operazioni che oltrepassano i limiti di strutture dati
allocate:
\begin{itemize}
  \item \textbf{Buffer overflow}: scrittura oltre la capacità di un buffer.

  \item \textbf{Buffer over-read}: lettura oltre i limiti di un buffer.

  \item \textbf{Out-of-bounds access}: violazione più generica dei limiti di array
    o strutture.
\end{itemize}
Un caso emblematico è rappresentato dalla vulnerabilità \textit{HeartBleed}\cite{heartbleed},
scoperta nel 2014 in OpenSSL, che ha causato una massiccia fuga di dati da server
web in tutto il mondo nel 2014.

\paragraph{Accessi temporali: violazione del ciclo di vita}
\label{sec:temporal}

Riguardano l'uso improprio di memoria dopo la sua deallocazione:
\begin{itemize}
  \item \textbf{Dangling pointer}: puntatori riferiti ad aree non più valide.

  \item \textbf{Use After Free}: accesso a blocchi di memoria dopo la loro deallocazione.

  \item \textbf{Double free}: doppia liberazione della stessa area di memoria.
\end{itemize}
Un esempio rilevante è la vulnerabilità rilevata nel 2020 nel motore \textit{WebAudio}
di Google Chrome\cite{webaudio_uaf}, che permetteva a siti web malevoli di
eseguire codice arbitrario nel browser.

\paragraph{Accessi a memoria non inizializzata}
\label{sec:initialization}

Comprendono l'uso di riferimenti a memoria non correttamente preparata:
\begin{itemize}
  \item \textbf{Null pointer dereference}: dereferenziazione di puntatori nulli.

  \item \textbf{Uninitialized memory access}: uso di variabili o strutture prima
    della loro inizializzazione.

  \item \textbf{Wild pointer}: puntatori con valori casuali o invalidi.
\end{itemize}
Un caso significativo è stato trovato nel kernel Linux\cite{null_pointer_dereference_linux},
in cui una dereferenziazione di un puntatore nullo permetteva escalation di privilegi
e Denial of Service.

\paragraph{Memory leaks}
\label{sec:memory_leaks} Pur non essendo tecnicamente vulnerabilità di memory
safety (poiché non comportano accessi invalidi), i memory leak causano progressivo
esaurimento della memoria quando aree allocate non vengono rilasciate. Questo
può compromettere stabilità e sicurezza attraverso denial-of-service.