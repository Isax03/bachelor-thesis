\section{Tipi di vulnerabilità}
\label{sec:vulnerability_types}

Esistono diverse classificazioni per le vulnerabilità legate alla memoria, che
esplorano le debolezze in modo molto specifico relazionandole tra loro, come quella
proposta da MITRE (CWE-1399~\cite{cwe_1399}). Tuttavia, per fornire una visione d'insieme
delle vulnerabilità più comuni, questa sezione propone la tassonomia sviluppata da
OpenSSF~\cite{memory_safety_continuum_definition}.

La classificazione in questione organizza le vulnerabilità di memory safety in
quattro categorie principali: errori di accesso, variabili non inizializzate, memory
leak e race condition.

\begin{quote}
  \textbf{Nota:} La nomenclatura utilizzata in questa sezione può differire da
  quella impiegata nell'introduzione (come "buffer overflow" e "out-of-bounds
  write"). Queste variazioni riflettono semplicemente modi diversi di denominare
  le stesse vulnerabilità, dipendendo dal livello di specificità desiderato e
  dalle fonti di riferimento adottate. Per coerenza, sono stati mantenuti i
  termini presenti nelle rispettive fonti citate; per una nomenclatura più precisa
  e sistematica si rimanda alla classificazione CWE-1399~\cite{cwe_1399}.
\end{quote}

\paragraph{Errori di accesso (Access errors)}
\label{sec:access_errors}

Comprendono tutte le operazioni di lettura o scrittura invalide di un puntatore
che violano i confini o lo stato di validità della memoria:
\begin{itemize}
  \item \textbf{Buffer overflow} (CWE-787): scrittura oltre la capacità di un buffer.

  \item \textbf{Buffer over-read} (CWE-125): lettura oltre i limiti di un buffer.

  \item \textbf{Invalid page fault}: accesso a pagine di memoria non valide.

  \item \textbf{Use after free} (CWE-416): accesso a blocchi di memoria dopo la loro
    deallocazione.
\end{itemize}
Un caso emblematico è la vulnerabilità \textit{Heartbleed}~\cite{heartbleed} scoperta
nel 2014 in OpenSSL che, attraverso un buffer over-read, causò una massiccia fuga
di dati da server web. Anche i casi di WannaCry e CrowdStrike (introdotti precedentemente)
sono legati a vulnerabilità di accesso.

\paragraph{Variabili non inizializzate (Uninitialized variables)}
\label{sec:uninitialized}

Riguardano l'uso di variabili che non sono state assegnate con un valore valido:
\begin{itemize}
  \item \textbf{Null pointer dereference} (CWE-476): dereferenziazione di puntatori
    nulli.

  \item \textbf{Wild pointers}: puntatori con valori casuali o invalidi.

  \item \textbf{Uninitialized variables}: uso di variabili prima della loro inizializzazione.
\end{itemize}
Un caso significativo per tali bug è quello del kernel Linux~\cite{null_pointer_dereference_linux},
che ha mostrato come una dereferenziazione di puntatore nullo possa permettere escalation
di privilegi e Denial of Service.

\paragraph{Memory leak}
\label{sec:memory_leaks}

Comprendono situazioni in cui l'uso della memoria non è tracciato correttamente o
viene tracciato in modo errato:
\begin{itemize}
  \item \textbf{Stack exhaustion}: esaurimento dello spazio disponibile nello stack.

  \item \textbf{Heap exhaustion}: esaurimento dello spazio disponibile nell'heap.

  \item \textbf{Double free}: doppia liberazione della stessa area di memoria.

  \item \textbf{Invalid free}: tentativo di liberare memoria non allocata.

  \item \textbf{Mismatched free}: uso di funzioni di deallocazione incompatibili.

  \item \textbf{Unwanted aliasing}: riferimenti multipli non intenzionali alla stessa
    area.
\end{itemize}
Per questa categoria, si può fare riferimento al double free presente in WhatsApp
scoperto nel 2019~\cite{whatsapp_double_free}, che permetteva l'esecuzione di codice
arbitrario tramite l'invio di una GIF malformata.

\paragraph{Race condition}
\label{sec:race_conditions}

Sebbene le race condition riguardino accessi concorrenti a memoria condivisa e possano
causare comportamenti non deterministici che compromettono la memory safety,
esse spaziano oltre il focus specifico di questa tesi. Vengono pertanto menzionate
per completezza, ma non vengono trattate in dettaglio nei capitoli successivi.