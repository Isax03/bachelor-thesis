\section{Tipi di vulnerabilità}
\label{sec:vulnerability_types}

I bug che compromettono la memory safety si possono classificare in tre macro-categorie,
secondo la tassonomia proposta da OpenSSF\cite{memory_safety_continuum} e coerente
con le definizioni più comuni\cite{wikipedia_definition}. Ciascuna di esse descrive
una diversa modalità con cui un programma può accedere in modo errato alla memoria.

\paragraph{Accessi fuori dai limiti (Out-of-bounds)}
\label{sec:oob}

Comprendono tutte le situazioni in cui un programma legge o scrive al di fuori
dei limiti di un buffer, array o struttura dati. Questi bug violano l'isolamento
spaziale della memoria e possono corrompere dati o consentire esecuzione
arbitraria di codice.

\begin{itemize}
  \item \textbf{Buffer overflow}: scrittura oltre la fine di un buffer.

  \item \textbf{Buffer underflow}: scrittura prima dell'inizio del buffer.

  \item \textbf{Index out-of-bounds}: accesso a un indice non valido in array o vettori.
\end{itemize}

\paragraph{Accessi all'heap dopo il rilascio (Use-after-free)}
\label{sec:uaf}

Si verificano quando un programma accede a una regione di memoria
precedentemente deallocata. Questa memoria può essere riutilizzata da altri oggetti,
portando a comportamenti imprevedibili e vulnerabilità gravi.

\begin{itemize}
  \item \textbf{Dangling pointer}: puntatore che fa riferimento a un'area già liberata.

  \item \textbf{Double free}: tentativo di liberare due volte la stessa area di memoria,
    spesso sfruttabile per corrompere lo heap.
\end{itemize}

\paragraph{Accessi alla memoria non inizializzata o non allocata}
\label{sec:invalid_access}

Questa categoria comprende l'uso di puntatori non inizializzati, nulli o che puntano
a zone mai allocate. Tali accessi portano a crash o, in alcuni casi, a
esecuzione di codice non controllato.

\begin{itemize}
  \item \textbf{NULL dereference}: dereferenziazione di un puntatore nullo.

  \item \textbf{Uninitialized memory access}: utilizzo di variabili o strutture prima
    della loro inizializzazione.

  \item \textbf{Wild pointer}: puntatori con valore casuale o invalido, spesso
    derivanti da mancata inizializzazione.
\end{itemize}