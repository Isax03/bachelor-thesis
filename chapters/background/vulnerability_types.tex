\section{Tipi di vulnerabilità}
\label{sec:vulnerability_types}

I bug che compromettono la memory safety si possono classificare in tre macro-categorie,
secondo la tassonomia proposta da OpenSSF\cite{memory_safety_continuum_definition}
e coerente con le definizioni più comuni\cite{wikipedia_definition}. Ciascuna di
esse descrive una diversa modalità con cui un programma può accedere in modo errato
alla memoria.

\paragraph{Accessi fuori dai limiti (Out-of-bounds)}
\label{sec:oob}

Comprendono tutte le situazioni in cui un programma legge o scrive al di fuori dei
limiti di un buffer, array o struttura dati. Questi bug violano l'isolamento spaziale
della memoria e possono corrompere dati o consentire esecuzione arbitraria di codice.

\begin{itemize}
  \item \textbf{Buffer overflow}: scrittura oltre la fine di un buffer.

  \item \textbf{Buffer underflow}: scrittura prima dell'inizio del buffer.

  \item \textbf{Index out-of-bounds}: accesso a un indice non valido in array o vettori.
\end{itemize}

\paragraph{Accessi all'heap dopo il rilascio (Use-after-free)}
\label{sec:uaf}

Si verificano quando un programma accede a una regione di memoria
precedentemente deallocata. Questa memoria può essere riutilizzata da altri oggetti,
portando a comportamenti imprevedibili come crash e alla lettura di dati
sensibili o di norma non accessibili.

\begin{itemize}
  \item \textbf{Dangling pointer}: puntatore che fa riferimento a un'area già liberata
    e non più valida.

  \item \textbf{Double free}: tentativo di liberare due volte la stessa area di memoria,
    spesso sfruttabile per corrompere lo heap.
\end{itemize}

\paragraph{Accessi alla memoria non inizializzata o non allocata}
\label{sec:invalid_access}

Questa categoria comprende l'uso di puntatori non inizializzati, nulli o che
puntano a zone mai allocate. Tali accessi portano a crash o, in alcuni casi, a esecuzione
di codice non controllato.

\begin{itemize}
  \item \textbf{NULL dereference}: dereferenziazione di un puntatore nullo.

  \item \textbf{Uninitialized memory access}: utilizzo di variabili o strutture prima
    della loro inizializzazione.

  \item \textbf{Wild pointer}: puntatori con valore casuale o invalido, spesso
    derivanti da mancata inizializzazione.
\end{itemize}

\paragraph{Memory Leaks}
\label{sec:memory_leaks} Si verificano quando un programma alloca memoria ma non
la rilascia correttamente, portando a un graduale esaurimento delle risorse
disponibili. Sebbene meno critici di altre vulnerabilità, i memory leak possono
causare denial-of-service o degradazione delle prestazioni in sistemi long-running.

\paragraph{Classificazione CWE}
Oltre alla tassonomia di OpenSSF, la Common Weakness Enumeration (CWE) fornisce
una panoramica più ampia delle vulnerabilità software. La CWE è un catalogo sviluppato
dalla comunità informatica per identificare e classificare le debolezze nei sistemi
software. Il sito offre diversi elenchi di vulnerabilità raggruppate per
contesto, tra cui la categoria CWE-1399\footnote{Comprehensive Categorization: Memory
Safety, \url{https://cwe.mitre.org/data/definitions/1399.html}} che raccoglie le
vulnerabilità legate alla memoria. Per ogni debolezza è fornita una descrizione,
terminologie alternative, relazioni con altre debolezze e collegamenti a best practices
dettate da enti come OWASP e SEI CERT, menzionati più avanti nella
\autoref{sec:best-practices-codice}. Alcune vulnerabilità sono poi associate anche
ai relativi vettori d'attacco a cui sono soggette.