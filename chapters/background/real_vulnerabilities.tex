\section{Casi reali di vulnerabilità memory-unsafe}
\label{sec:real_vulnerabilities}

Oltre alla definizione teorica e alla classificazione dei bug legati alla memory
safety, è fondamentale analizzare esempi concreti che ne dimostrino l'impatto
nel mondo reale. Le vulnerabilità memory-unsafe non sono solo un problema astratto
o relegato a software obsoleto: continuano a colpire sistemi moderni, infrastrutture
critiche e software di sicurezza stesso.

\paragraph{WannaCry}
\label{sec:wannacry} Uno dei casi più emblematici è WannaCry (2017), un ransomware
che ha causato danni su scala globale infettando centinaia di migliaia di dispositivi.
L'attacco sfruttava la vulnerabilità EternalBlue\cite{eternalblue}, un buffer
overflow nel protocollo SMBv1 di Windows, che permetteva l'esecuzione remota di codice.
Questo errore di memory safety ha consentito al malware di propagarsi
automaticamente in rete tramite il protocollo SMBv1, rendendolo uno degli attacchi
più rapidi ed efficaci nella storia del ransomware.

\paragraph{CrowdStrike Falcon Sensor}
\label{sec:crowdstrike} Nel luglio 2024, un aggiornamento distribuito da CrowdStrike
ha introdotto un bug in un driver del suo prodotto Falcon Sensor, causando crash
su larga scala nei sistemi Windows. L'errore, come evidenziato nella \textit{Root
Cause Analysis}\cite{crowdstrike_rca}, era legato a un accesso scorretto alla memoria
nel kernel mode, evidenziando come anche software considerato altamente
affidabile e soggetto a forti controlli possa introdurre fault critici se la
gestione della memoria non è corretta. A differenza di WannaCry, non si è
trattato di un exploit, ma l'incidente ha avuto impatti simili in termini di downtime
e disservizi.

\paragraph{Heartbleed}
\label{sec:heartbleed} Infine, il bug Heartbleed (2014) nella libreria OpenSSL ha
esposto una vasta quantità di dati sensibili attraverso una lettura fuori dai limiti
(out-of-bounds read) nella gestione dell'estensione TLS Heartbeat\cite{heartbleed}. Questa
vulnerabilità ha mostrato che anche i componenti di rete fondamentali per la
sicurezza delle comunicazioni su Internet possono essere affetti da gravi
problemi di memory safety. Heartbleed ha messo in luce quanto sia pericoloso un bug
che, pur non causando crash visibili, permette la compromissione silenziosa di chiavi
private, credenziali e informazioni personali.\\

Questi tre casi, pur diversi tra loro per dinamica e impatto, convergono su un punto
comune: la mancanza di memory safety rappresenta un rischio sistemico. Per questo
motivo, garantire un uso sicuro della memoria è oggi una priorità non solo per i
software critici, ma per l'intero ecosistema digitale.