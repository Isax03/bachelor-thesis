\chapter{Background}
\label{cha:background}

In questa sezione vengono fornite le nozioni fondamentali necessarie per
comprendere il concetto di \textit{memory safety} e il suo impatto nello sviluppo
software. La sicurezza della memoria è un aspetto cruciale per la stabilità e la
correttezza delle applicazioni moderne, in particolare in contesti dove la sicurezza
è una priorità.

Verranno quindi chiariti i concetti di base legati alla memory safety, le
tipologie più comuni di vulnerabilità dovute a una gestione scorretta della
memoria, e il modo in cui tali vulnerabilità possano essere sfruttate da attaccanti
per compromettere l'integrità di sistemi reali. Questa analisi fornisce il
contesto necessario per comprendere l'importanza delle tecniche e degli
strumenti di mitigazione che verranno presentati nei capitoli successivi.

% Definizione di memory safety
\section{Definizione di Memory Safety}
\label{sec:memory_safety}

Il concetto di memory safety non dispone di una definizione formale univoca,
generalmente riconosciuta da tutta la comunità scientifica. In letteratura si trovano,
per la maggior parte, definizioni di carattere più pratico, che si concentrano
sull'assenza di determinati tipi di bug o comportamenti indesiderati legati alla
gestione della memoria.

Un'eccezione significativa è rappresentata dall'articolo \textit{The Meaning of
Memory Safety}\cite{meaning_memory_safety}, che si pone l'obiettivo di formalizzare
rigorosamente il concetto, supportandolo con teoremi e dimostrazioni.

\vspace{0.5em}
Di seguito, verranno presentate sia la definizione formale che quella pratica e intuitiva
di memory safety, per fornire una visione completa del concetto e delle sue
implicazioni.

\subsection{Definizione formale}
\label{sec:formal_definition}

L'articolo citato\cite{meaning_memory_safety} propone una formalizzazione
rigorosa della memory safety attraverso un framework matematico basato su
semantica operazionale e strutture formali per ragionare sulla sicurezza della
memoria in modo verificabile.

Nel modello proposto, un programma è considerato \textit{memory safe} se durante
la sua esecuzione preserva le seguenti proprietà fondamentali:
\begin{enumerate}
  \item \textbf{Isolamento spaziale e temporale}:
    \begin{itemize}
      \item Le operazioni su una regione di memoria non influenzano dati al di fuori
        del proprio ``footprint'' (porzione accessibile).

      \item I \textit{Frame Theorems} (Sezione 3) garantiscono che l'estensione
        dello heap iniziale con blocchi non raggiungibili non alteri il
        comportamento del programma.
    \end{itemize}

  \item \textbf{Non-interferenza} (Corollario 1):
    \begin{itemize}
      \item La memoria non raggiungibile non può modificare né essere modificata
        dall'esecuzione. Ciò assicura sia \textit{integrità} (impossibilità di
        alterare dati isolati) sia \textit{segretezza} (impossibilità di dedurne
        l'esistenza).
    \end{itemize}

  \item \textbf{Contenimento degli errori}:
    \begin{itemize}
      \item Accessi illegali (es. dereferenziazione di puntatori invalidi)
        terminano l'esecuzione in modo prevedibile con un errore esplicito, evitando
        comportamenti indefiniti.
    \end{itemize}
\end{enumerate}

Questo modello consente di dimostrare proprietà come la protezione tra moduli e
l'assenza di comportamenti indefiniti. Conseguentemente, esso fornisce anche un
solido framework matematico per specificare e dimostrare le garanzie di
sicurezza implementate da un sistema (come proprietà del type system o del memory
model). Tale formalizzazione guida inoltre lo sviluppo di strumenti verificati per
l'enforcement della memory safety, come i monitor hardware/software descritti
nel paper (ad esempio PUMP, Sezione 5).

Tuttavia, questa definizione si basa su ipotesi idealizzate, come memoria illimitata,
assenza di aritmetica sui puntatori o di cast a interi, che lo rendono parzialmente
distante dai linguaggi e dagli ambienti reali (Sezione 4). Per questo motivo una
definizione più intuitiva e pratica è necessaria per comprendere il concetto di
memory safety in contesti reali.

\subsection{Definizione pratica}
\label{sec:practical_definition}

Dal punto di vista pratico, la memory safety viene comunemente intesa come la
capacità di un programma di evitare comportamenti errati o vulnerabilità legate
alla gestione della memoria dinamica.

In particolare, si riferisce alla prevenzione di errori come buffer overflows,
use-after-free e dangling pointers, che possono compromettere l'integrità o la sicurezza
del software. Una panoramica di queste vulnerabilità sarà fornita nella \autoref{sec:vulnerability_types}.

Sebbene non costituisca una fonte accademica formale, Wikipedia\cite{wikipedia_definition}
riflette questa visione diffusa, definendo la memory safety come \textit{"lo stato
di protezione da vari bug nel software e vulnerabilità nella sicurezza, quando
si ha a che fare con accessi in memoria"}. Questa formulazione, pur semplificata,
è in linea con numerosi articoli tecnici e documenti divulgativi che descrivono la
memory safety attraverso l'eliminazione di specifiche classi di bug piuttosto che
tramite una definizione formale.

% Memory Safety Continuum %
\subsection{Memory safety come spettro continuo}
\label{sec:continuum_definition}

Sebbene le definizioni formale e pratica adottino approcci diversi, entrambe sottintendono
che la memory safety non sia una proprietà assoluta o binaria, ma una condizione
che può essere raggiunta in misura variabile.

Questa idea viene esplicitamente sviluppata nell'articolo \textit{The Memory
Safety Continuum}\cite{memory_safety_continuum}, pubblicato dalla OpenSSF, che
propone di interpretare la memory safety come uno spettro continuo. In tale visione,
sistemi e linguaggi possono offrire livelli differenti di protezione, a seconda delle
garanzie fornite e delle mitigazioni adottate.

Questo approccio sottolinea come la memory safety possa essere rafforzata progressivamente,
piuttosto che garantita in modo assoluto.

% Tipi di vulnerabilità
  \section{Tipi di vulnerabilità e casi reali}
  \label{sec:vulnerability_types}

  Secondo la tassonomia proposta da OpenSSF~\cite{memory_safety_continuum_definition},
  che rispecchia le definizioni più comunemente accettate, i bug che compromettono
  la memory safety possono essere classificati in tre macro-categorie. Ognuna di
  queste categorie rappresenta una diversa modalità con cui un programma può
  accedere in modo errato alla memoria.

  Di seguito vengono illustrate le principali categorie di vulnerabilità relative
  alla memory safety, accompagnate da esempi concreti che ne evidenziano l'impatto
  sulla sicurezza dei sistemi.

  \paragraph{Accessi fuori dai limiti (Out-of-bounds)}
  \label{sec:oob}

  Si verificano quando un programma legge o scrive al di fuori dei limiti di un buffer,
  un array o una struttura dati. Questi bug violano l'isolamento spaziale della memoria
  e possono corrompere dati o consentire esecuzione arbitraria di codice.

  \begin{itemize}
    \item \textbf{Buffer overflow}: scrittura oltre la fine di un buffer.

    \item \textbf{Buffer underflow}: scrittura prima dell'inizio del buffer.

    \item \textbf{Index out-of-bounds}: accesso a un indice non valido in array o vettori.
  \end{itemize}

  \textbf{Caso reale: HeartBleed} -- Nel 2014, una vulnerabilità di tipo out-of-bounds
  read nella libreria OpenSSL ha causato una massiccia fuga di dati sensibili da server
  web in tutto il mondo. L'errore, situato nell'estensione TLS Heartbeat,
  permetteva a un client malintenzionato di inviare una richiesta in cui
  dichiarava una lunghezza del messaggio superiore a quella effettivamente inviata.
  Il server, invece di verificare che la lunghezza dichiarata corrispondesse ai
  dati ricevuti, rispondeva restituendo il proprio messaggio originale più dati aggiuntivi
  presi casualmente dalla propria memoria interna. Questi dati potevano includere
  informazioni sensibili come chiavi private, credenziali o altri dati critici.~\cite{heartbleed}.

  \paragraph{Accessi all'heap dopo il rilascio (Use-after-free)}
  \label{sec:uaf}

  Consistono nell'accesso a una regione di memoria precedentemente deallocata. Questa
  memoria può essere riutilizzata da altri oggetti, portando a comportamenti
  imprevedibili e alla lettura di dati sensibili.

  \begin{itemize}
    \item \textbf{Dangling pointer}: puntatore che fa riferimento a un'area già liberata
      e non più valida.

    \item \textbf{Double free}: tentativo di liberare due volte la stessa area di memoria,
      spesso legato ad esecuzioni multi-thread che portano a concorrenza di accessi.
  \end{itemize}

  \textbf{Caso reale: WebAudio in Chrome} -- Nel 2020 è stata scoperta una vulnerabilità
  use-after-free nel motore WebAudio di Google Chrome, legata alla gestione
  asincrona degli oggetti \texttt{AudioBufferSourceNode}. Un attaccante poteva
  manipolare il flusso di esecuzione creando condizioni di race tra thread audio e
  main thread, portando a un accesso a memoria già liberata. La vulnerabilità è
  stata sfruttata per ottenere l'esecuzione di codice arbitrario e dimostra come
  errori apparentemente semplici possano essere concatenati in exploit complessi e
  reali~\cite{webaudio_uaf}.

  \paragraph{Accessi alla memoria non inizializzata o non allocata}
  \label{sec:invalid_access}

  Questa categoria comprende l'uso di puntatori non inizializzati, nulli o che puntano
  a zone mai allocate. Tali accessi portano a crash o, in alcuni casi, a
  esecuzione di codice non controllato.

  \begin{itemize}
    \item \textbf{Null dereference}: dereferenziazione di un puntatore nullo.

    \item \textbf{Uninitialized memory access}: utilizzo di variabili o strutture prima
      della loro inizializzazione.

    \item \textbf{Wild pointer}: puntatori con valore casuale o invalido, spesso
      derivanti da mancata inizializzazione.
  \end{itemize}

  \textbf{Caso reale: Linux Kernel (versioni 2.6.x)} -- Nel 2009, è stato scoperto
  un bug di null pointer dereference nel kernel Linux. La vulnerabilità prevedeva la
  mancata inizializzazione dei puntatori a funzione in alcune strutture. Questo
  permetteva a un eventuale attaccante di scatenare una dereferenziazione di
  puntatori nulli, portando a esecuzione arbitraria di codice inserito nella pagina
  con indirizzo 0 (NULL), per raggiungere privilegi di root e avere accesso
  completo al sistema.~\cite{null_pointer_dereference_linux}

  \paragraph{Memory Leaks}
  \label{sec:memory_leaks} Si verificano quando un programma alloca dinamicamente memoria
  senza rilasciarla correttamente, causando un accumulo progressivo di aree
  inutilizzate che rimangono occupate per tutta la durata dell'esecuzione. Poiché i
  memory leak non comportano accessi non autorizzati o errati alla memoria, non
  sono generalmente considerati come bug relativi alla memory safety, ma possono comunque
  compromettere la stabilità e la sicurezza di un sistema.

  \subsection{Classificazione CWE}
  Oltre alla tassonomia di OpenSSF, la Common Weakness Enumeration (CWE) fornisce
  una panoramica più ampia delle vulnerabilità software.

  La CWE è un catalogo sviluppato dalla comunità informatica per identificare e
  classificare le debolezze nei sistemi software. Il sito offre diversi elenchi di
  vulnerabilità raggruppate per contesto, tra cui la categoria CWE-1399\footnote{Comprehensive
  Categorization: Memory Safety,~\url{https://cwe.mitre.org/data/definitions/1399.html}}
  che raccoglie le vulnerabilità legate alla memoria.

  Per ogni debolezza è fornita una descrizione, relazioni con altre debolezze e
  collegamenti a best practice dettate da enti come OWASP e SEI CERT, menzionati più
  avanti nella~\autoref{sec:best-practices-codice}. Alcune vulnerabilità sono poi
  associate anche ai relativi vettori d'attacco a cui sono soggette.

% Esempi di vulnerabilità reali
\section{Exploit reali}
\label{sec:real_exploits}