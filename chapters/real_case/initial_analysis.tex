\section{Analisi preliminare}
\label{sec:initial_analysis}

Dal momento che questo caso studio si basa su un'applicazione già scritta e funzionante,
la prima fase non sarà quella di riscrivere il codice seguendo le best practice
e usando librerie che introducono controlli di sicurezza, ma piuttosto quella di
analizzare il codice esistente, identificare le vulnerabilità e valutare l'efficacia
degli strumenti utilizzati per la loro individuazione.

\subsection*{Analisi statica}
Seguendo l'ordine in cui sono state introdotte le tecniche di mitigazione nel~\autoref{cha:sdlc},
la prima fase di analisi è quella statica, che analizza il codice sorgente senza
effettivamente eseguirlo.

\paragraph{Criteri di selezione degli strumenti}
I tool di analisi sono stati scelti a partire dalla lista proposta dal \textit{National
Institute of Standards and Technology} (NIST)~\cite{nist_sast_list} che propone
quasi 90 strumenti di analisi statica, in una tabella che comprende i linguaggi supportati,
le funzionalità offerte, la disponibilità (gratuita o meno) e la data dell'ultimo
aggiornamento. Grazie a queste informazioni, è stato possibile restringere la scelta
degli strumenti da utilizzare, usando i seguenti criteri:
\begin{itemize}
  \item \textbf{Linguaggio supportato}: rimuovendo dalla lista gli strumenti che
    non supportano il linguaggio C, si è ottenuta una lista di circa 50 tool;

  \item \textbf{Disponibilità gratuita}: dal momento che si tratta di un caso studio
    didattico, si è scelto di utilizzare solo strumenti gratuiti, eliminando dalla
    lista quelli a pagamento o con licenza commerciale. Questo ha ridotto
    ulteriormente la lista a 16 strumenti;

  \item \textbf{Data dell'ultimo aggiornamento}: per garantire l'affidabilità degli
    strumenti, si è scelto di considerare solo quelli aggiornati dal 2010 in poi.
    Questo ha dimezzato la lista a otto strumenti;

  \item \textbf{Focus sulla memory safety}: sono stati infine esclusi gli strumenti
    che non avevano la sicurezza della memoria tra i principali obiettivi e quelli
    troppo general purpose.
\end{itemize}

Questa scrematura ha portato a una lista finale di tre strumenti: \textbf{CppCheck},
un analizzatore statico di codice C/C++ focalizzato su errori di programmazione e
vulnerabilità di sicurezza; \textbf{Clang Static Analyzer}, parte del progetto
Clang che fornisce funzionalità avanzate per il rilevamento di errori e
vulnerabilità; e \textbf{Frama-C}, un analizzatore statico di codice C che si concentra
sulla sicurezza della memoria e sulla rilevazione di vulnerabilità comuni.

\subsubsection*{CppCheck}
L'analizzatore \texttt{cppcheck}\footnote{Sito: \url{https://cppcheck.sourceforge.io/}}
è il primo strumento che è stato utilizzato per l'analisi statica del codice sorgente.
Esso è in grado di rilevare errori di programmazione, vulnerabilità di sicurezza
e altri problemi nel codice C/C++.

Lo strumento è stato eseguito con il seguente comando:
\begin{lstlisting}[language=bash, numbers=none]
cppcheck --enable=warning --inconclusive --std=c11 *.c
\end{lstlisting}
Il comando specifica di abilitare i warning e di considerare anche i casi in cui
non è possibile determinare con certezza se un codice sia corretto o meno (\texttt{--inconclusive}),
e di utilizzare lo standard C11 per l'analisi.

L'analisi ha prodotto risultati limitati. CppCheck ha identificato esclusivamente
un potenziale memory leak, fallendo nel rilevare le vulnerabilità più critiche
come il buffer overflow e il double free presenti nel codice:

\begin{lstlisting}[language={}, numbers=none]
utils.c:17:9 error: Memory leak: info [memleak]
  return NULL;
  ^
\end{lstlisting}

Questo risultato è stato inaspettato, poiché ci si aspettava che lo strumento
rilevasse almeno parte delle vulnerabilità presenti nel codice, come ad esempio il
rischio di buffer overflow o l'evidente double free presente nel file \texttt{tracker\_core.c}.
Per questo motivo, è consigliabile utilizzare più strumenti di analisi statica
nei progetti reali, per ottenere risultati più completi e affidabili.

\subsubsection*{Clang Static Analyzer}
Un altro strumento di analisi statica utilizzato è \texttt{clang}\footnote{Sito:
\url{https://clang.llvm.org/}}, che è un compilatore con funzionalità di analisi
statica integrate. Esso, come \texttt{cppcheck}, è in grado di rilevare errori
di programmazione e vulnerabilità di sicurezza nel codice C/C++.

Anche in questo caso, lo strumento è stato eseguito dal terminale e l'analisi statica
del codice è possibile con il seguente comando:
\begin{lstlisting}[language=bash, numbers=none]
clang --analyze -Xanalyzer -analyzer-checker=core,security *.c
\end{lstlisting}
Il comando specifica di eseguire l'analisi statica del codice e di abilitare i
checker per la sicurezza e per i \textit{core issue}.

L'analisi ha prodotto risultati più interessanti rispetto a \texttt{cppcheck}, evidenziando
diversi problemi nel codice:
\begin{itemize}
  \item \textbf{Funzioni non sicure}: gran parte dell'output del comando evidenzia
    l'uso di funzioni non sicure, in particolare \texttt{strcpy}, \texttt{sprintf}
    e \texttt{fprintf}. Inoltre, \texttt{clang} suggerisce funzioni alternative più
    sicure, come \texttt{strlcpy}, \texttt{sprintf\_s} e \texttt{fprintf\_s}, che
    dovrebbero essere utilizzate per evitare vulnerabilità di buffer overflow;

  \item \textbf{Memory Leak}: viene rilevato lo stesso memory leak già evidenziato
    da \texttt{cppcheck};

  \item \textbf{Double Free}: al contrario di \texttt{cppcheck}, \texttt{clang}
    riesce a rilevare l'ovvio problema di double free presente nel file \texttt{tracker\_core.c}.
\end{itemize}

\subsubsection*{Frama-C}
L'ultimo strumento di analisi statica utilizzato sul progetto
File Tracker è \texttt{Frama-C}\footnote{Sito: \url{https://frama-c.com/}}, un
analizzatore che mira a combinare più tecniche di analisi, fornite tramite plugin,
per massimizzare l'eliminazione di bug e vulnerabilità nel codice C.

Frama-C è un analizzatore più avanzato rispetto ai precedenti, poiché prevede sia
uno strumento che lavora da linea di comando (come gli altri due), sia un'interfaccia
grafica che permette di visualizzare i risultati dell'analisi in modo più
interattivo. In entrambi i casi, l'output richiede più sforzo per essere
compreso, poiché i risultati sono presentati in modo più dettagliato e con terminologia
più tecnica.

L'interfaccia grafica di Frama-C è stata avviata con il comando:
\begin{lstlisting}[language=bash, numbers=none]
frama-c-gui -eva *.c
\end{lstlisting}
ottenendo una visualizzazione del codice sorgente con i risultati dell'analisi
evidenziati direttamente sul codice. In questo modo, è possibile navigare tra le
funzioni e i file del progetto, consentendo una visualizzazione interattiva dei problemi
rilevati.

L'analisi, in questo caso, ha prodotto risultati meno soddisfacenti rispetto agli
altri due strumenti, dal momento che la maggior parte del codice analizzato ha
segnalato risultati inconcludenti, ovvero che non è stato possibile determinare con
certezza se il codice sia corretto o meno. L'unico bug che Frama-C è riuscito a
segnalare è stato lo use-after-free nelle ultime righe del file \texttt{main.c}.

\subsubsection*{Riepilogo dell'analisi statica}
L'analisi statica del codice sorgente del progetto File Tracker, condotta
utilizzando tre strumenti distinti, ha evidenziato diverse criticità nel codice.
Come mostrato nella~\autoref{tab:static_analysis_results}, ciascun strumento ha dimostrato
punti di forza e limitazioni specifiche: CppCheck ha identificato principalmente
memory leak, Clang ha mostrato la migliore capacità di rilevamento generale segnalando
funzioni non sicure e double free, mentre Frama-C ha evidenziato use-after-free
non rilevati dagli altri strumenti. Complessivamente, nessun singolo strumento è
riuscito a identificare tutte le vulnerabilità presenti, confermando l'importanza
di utilizzare approcci complementari nell'analisi statica. I risultati ``Parziale''
nella tabella indicano i warning di Clang sull'utilizzo di funzioni non sicure
come \texttt{strcpy} e \texttt{sprintf}, dal momento che l'analizzatore non fornisce
informazioni specifiche sulla vulnerabilità in particolare, ma segnala che il
codice potrebbe essere a rischio.

\begin{table}[htbp]
  \centering
  \begin{tabular}{|l|l|l|l|l|}
    \hline
    \textbf{Bug/Vulnerabilità} & \textbf{Posizione}             & \textbf{CppCheck}              & \textbf{Clang}                 & \textbf{Frama-C}               \\
    \hline
    Out-of-Bounds Write        & \texttt{main.c:27,42}         & \cellcolor{red!20}Non Rilevato & \cellcolor{yellow!20}Parziale  & \cellcolor{red!20}Non Rilevato \\
    \hline
    Use After Free             & \texttt{main.c:127}            & \cellcolor{red!20}Non Rilevato & \cellcolor{red!20}Non Rilevato & \cellcolor{green!20}Rilevato   \\
    \hline
    Memory Leak                & \texttt{utils.c:14-18}         & \cellcolor{green!20}Rilevato   & \cellcolor{green!20}Rilevato   & \cellcolor{red!20}Non Rilevato \\
    \hline
    Out-of-Bounds Read         & \texttt{utils.c:55}            & \cellcolor{red!20}Non Rilevato & \cellcolor{red!20}Non Rilevato & \cellcolor{red!20}Non Rilevato \\
    \hline
    Dangling Pointer           & \texttt{utils.c:69-76}         & \cellcolor{red!20}Non Rilevato & \cellcolor{red!20}Non Rilevato & \cellcolor{red!20}Non Rilevato \\
    \hline
    Buffer Overflow            & \texttt{tracker\_core.c:14,18} & \cellcolor{red!20}Non Rilevato & \cellcolor{yellow!20}Parziale  & \cellcolor{red!20}Non Rilevato \\
    \hline
    Double Free                & \texttt{tracker\_core.c:45}    & \cellcolor{red!20}Non Rilevato & \cellcolor{green!20}Rilevato   & \cellcolor{red!20}Non Rilevato \\
    \hline
    Dangling Pointer           & \texttt{tracker\_core.c:56}    & \cellcolor{red!20}Non Rilevato & \cellcolor{red!20}Non Rilevato & \cellcolor{red!20}Non Rilevato \\
    \hline
    Out-of-Bounds Write        & \texttt{tracker\_core.c:81}    & \cellcolor{red!20}Non Rilevato & \cellcolor{yellow!20}Parziale  & \cellcolor{red!20}Non Rilevato \\
    \hline
    Use After Free             & \texttt{tracker\_core.c:161}   & \cellcolor{red!20}Non Rilevato & \cellcolor{red!20}Non Rilevato & \cellcolor{red!20}Non Rilevato \\
    \hline
    Out-of-Bounds Write        & \texttt{tracker\_core.c:194}   & \cellcolor{red!20}Non Rilevato & \cellcolor{yellow!20}Parziale  & \cellcolor{red!20}Non Rilevato \\
    \hline
    Out-of-Bounds Write        & \texttt{tracker\_core.c:216}   & \cellcolor{red!20}Non Rilevato & \cellcolor{yellow!20}Parziale  & \cellcolor{red!20}Non Rilevato \\
    \hline
  \end{tabular}
  \caption{Risultati dell'analisi statica}
  \label{tab:static_analysis_results}
\end{table}

\subsection*{Analisi dinamica}
Dopo aver analizzato il codice sorgente, la fase successiva è osservare il
comportamento del programma durante l'esecuzione tramite l'analisi dinamica.

Seguendo criteri simili a quelli utilizzati per la selezione degli strumenti di
analisi statica, sono stati scelti due strumenti di analisi dinamica, entrambi
già citati nella~\autoref{sec:analisi-dinamica}:
\begin{itemize}
  \item \textbf{AddressSanitizer (ASan)}, uno strumento di analisi dinamica integrato
    in Clang e GCC, progettato per rilevare errori di memoria come buffer
    overflow, use-after-free e memory leak;

  \item \textbf{Valgrind}, un framework di analisi dinamica che fornisce diversi
    strumenti per il rilevamento di errori di memoria, come memory leak, buffer overflow
    e use-after-free, grazie al suo strumento \texttt{memcheck}.
\end{itemize}

\noindent
Poiché entrambi gli strumenti di analisi dinamica dipendono strettamente dagli
input forniti al programma durante l'esecuzione e richiedono scenari specifici
per attivare le diverse vulnerabilità, sono stati testati solamente tre casi d'uso
mirati che permettono agli analizzatori di rilevare le seguenti vulnerabilità:
\begin{itemize}
  \item \textbf{Buffer Overflow} (\texttt{tracker\_core.c:14,18}), quando la
    lunghezza del percorso del file di log o della directory monitorata supera
    il limite di 255 caratteri;

  \item \textbf{Use After Free} (\texttt{tracker\_core.c:161}), quando il
    programma prova a stampare il nome di un file eliminato dalla directory
    monitorata dopo aver liberato la memoria allocata per il nome del file;

  \item \textbf{Double Free} (\texttt{tracker\_core.c:45}), quando il programma
    libera due volte la memoria allocata per la stringa del percorso della
    directory.
\end{itemize}

\subsubsection*{AddressSanitizer}
AddressSanitizer\footnote{Sito: \url{https://github.com/google/sanitizers/wiki/AddressSanitizer}},
come già menzionato nel~\autoref{cha:sdlc}, non si limita ad agire sul binario
compilato, ma richiede l'accesso al codice sorgente, che verrà arricchito con i
controlli necessari prima della compilazione. Infatti, il comando utilizzato per
compilare il progetto con AddressSanitizer è il seguente: \begin{lstlisting}[language=bash, numbers=none]
clang -fsanitize=address -g -O0 -std=c11 -o file_tracker_asan main.c tracker_core.c utils.c
\end{lstlisting}

Il comando specifica di abilitare AddressSanitizer (\texttt{-fsanitize=address}),
di includere le informazioni di debug (\texttt{-g}), di disabilitare le
ottimizzazioni (\texttt{-O0}) e di utilizzare lo standard C11 (\texttt{-std=c11}).

Dopo la compilazione, il programma può essere eseguito in cerca di vulnerabilità
o bug. Nello specifico, per testare le vulnerabilità selezionate, è necessaria
la creazione di contesti specifici. Di seguito sono riportati i comandi utilizzati
per l'impostazione di ciascun contesto, in base alla vulnerabilità da testare:
\begin{itemize}
  \item \textbf{Buffer Overflow}: per testare il buffer overflow, è sufficiente
    fornire un percorso di directory o di file di log che superi i 255 caratteri.
    Poiché il tracker verifica l'esistenza della directory monitorata prima dell'esecuzione,
    è necessario creare una struttura di directory annidate sufficientemente
    profonda da raggiungere il limite di 255 caratteri. A fini di riproducibilità,
    sono stati utilizzati i seguenti comandi: \begin{lstlisting}[language=bash, numbers=none]
# creazione della directory
mkdir -p $(printf 'asan/asan1/asan2/%.0s' {1..15}) # crea 45 directory annidate
# esecuzione del tracker con un percorso di directory lungo
./file_tracker_asan ./$(printf 'asan/asan1/asan2/%.0s' {1..15})
# oppure con un file di log nella directory appena creata
./file_tracker_asan ./ -l ./$(printf 'asan/asan1/asan2/%.0s' {1..15})logfile.log
    \end{lstlisting}

  \item \textbf{Use After Free}: per testare l'use-after-free, è sufficiente
    creare una directory e un file, eseguire il tracker e infine eliminare il file:
    \begin{lstlisting}[language=bash, numbers=none]
# creazione della directory e del file
mkdir asan && touch ./asan/asan.txt
# esecuzione del tracker per monitorare la directory ./asan/
./file_tracker_asan ./asan/
    \end{lstlisting}

  \item \textbf{Double Free}: per quanto riguarda il double free, il processo è
    ancora più semplice, poiché è sufficiente eseguire il tracker su una directory
    arbitraria e poi terminare il programma premendo \texttt{Ctrl+C} per forzare
    la chiusura: \begin{lstlisting}[language=bash, numbers=none]
# esecuzione del tracker su una directory arbitraria
./file_tracker_asan ./asan/
# terminazione del programma con Ctrl+C
    \end{lstlisting}
    In generale, il double free si verifica ogni volta che il programma termina
    a causa di un segnale \texttt{SIGINT} (come \texttt{Ctrl+C}).
\end{itemize}

Riproducendo questi tre scenari, AddressSanitizer è stato in grado di rilevare tutte
e tre le vulnerabilità, producendo output dettagliati e informativi per ciascun
caso. Un esempio rappresentativo dell'output generato è mostrato nell'~\autoref{appendix:asan_output}
(relativo al primo scenario di test).

Le caratteristiche distintive degli output prodotti da AddressSanitizer
includono:
\begin{itemize}
  \item \textbf{Localizzazione delle vulnerabilità}: AddressSanitizer fornisce informazioni
    dettagliate sulla posizione esatta della vulnerabilità nel codice, indicando
    il file e la riga specifica in cui si verifica il problema (grazie all'opzione
    \texttt{-g} utilizzata in fase di compilazione). Questo è molto utile per la
    correzione rapida e mirata;

  \item \textbf{Classificazione delle vulnerabilità}: per ogni vulnerabilità rilevata,
    ASan fornisce una nomenclatura chiara del tipo di problema (es. ``heap-buffer-overflow'',
    ``attempting double free'', ...) che aiuta a comprendere immediatamente la natura
    del bug;

  \item \textbf{Stack trace}: per ogni vulnerabilità rilevata, AddressSanitizer
    fornisce un traceback completo dello stack delle chiamate, mostrando la sequenza
    di funzioni che ha portato al problema;

  \item \textbf{Informazioni sulla memoria}: l'output include dettagli tecnici sulla
    regione di memoria coinvolta, come l'indirizzo specifico, la dimensione del blocco
    allocato e lo stato della memoria (freed, allocated, etc.).
\end{itemize}

Queste informazioni sono estremamente utili per gli sviluppatori, poiché permettono
di individuare rapidamente e risolvere le vulnerabilità nel codice, migliorando
la sicurezza e l'affidabilità dell'applicazione.

\subsubsection*{Valgrind}
Valgrind\footnote{Sito: \url{http://valgrind.org/}},
a differenza di AddressSanitizer, non agisce direttamente sul codice sorgente,
ma analizza l'esecuzione del programma compilato, monitorando l'uso della memoria
e rilevando errori come buffer overflow, use-after-free e memory leak.

Per utilizzare Valgrind, è necessario compilare il progetto senza AddressSanitizer,
ma preferibilmente con le informazioni di debug attivate, in modo da ottenere un
output più dettagliato e utile per la correzione dei bug (come per ASan). Il
comando usato per questo progetto è: \begin{lstlisting}[language=bash, numbers=none]
clang -g -O0 -std=c11 -o file_tracker_valgrind main.c tracker_core.c utils.c
\end{lstlisting}

Dopo la compilazione, il programma può essere eseguito con Valgrind aggiungendo il
seguente prefisso al comando di esecuzione del programma:
\begin{lstlisting}[language=bash, numbers=none]
valgrind --tool=memcheck --leak-check=full --show-leak-kinds=all
\end{lstlisting}

Il comando specifica di utilizzare lo strumento \texttt{memcheck} di Valgrind, che
è progettato per rilevare errori di memoria, e di abilitare il controllo
completo dei leak di memoria (\texttt{--leak-check=full}), mostrando tutti i
tipi di leak (\texttt{--show-leak-kinds=all}), anche se i tre scenari di test
richiesti non richiedono necessariamente l'analisi dei leak di memoria.

Anche nel caso di Valgrind, sono stati usati gli stessi tre scenari di test per verificare
le vulnerabilità selezionate, e i log prodotti sono simili a quelli di ASan: entrambi
gli strumenti forniscono informazioni dettagliate sulla posizione delle vulnerabilità,
con uno stack trace e informazioni sulla memoria coinvolta. Tuttavia, con ASan è
più intuitivo, soprattutto per i principianti, capire la natura del bug rilevato,
poiché Valgrind fornisce un output più tecnico e meno immediato.

Un'altra differenza importante tra i due strumenti è che Valgrind non blocca l'esecuzione
nel momento in cui viene rilevato un problema, ma continua a eseguire il
programma fino alla sua conclusione, producendo un report finale con tutti i problemi
rilevati. Infatti, in tutti e tre gli scenari di test, Valgrind ha rilevato
anche uno use-after-free (\texttt{main.c:127}), non facente parte dei tre casi
di test previsti, che si verifica quando il programma prova a stampare il numero
di file monitorati dopo aver liberato la memoria allocata per la struttura del tracker.
AddressSanitizer, al contrario, interrompe l'esecuzione del programma non appena
viene rilevato un problema, evitando ulteriori errori o comportamenti imprevisti.

\subsubsection*{Considerazioni sull'analisi dinamica}
L'analisi dinamica del progetto File Tracker ha dimostrato l'efficacia di AddressSanitizer
e Valgrind nel rilevare vulnerabilità di memory safety, confermando la loro
utilità nella fase di testing e debugging. Entrambi gli strumenti hanno identificato
le vulnerabilità previste, fornendo informazioni dettagliate sulla loro natura e
posizione nel codice.

Tuttavia, è importante precisare che l'analisi dinamica ha dei limiti intrinseci
come la sua dipendenza dai percorsi di esecuzione attivati durante i test. Le
vulnerabilità vengono rilevate solo se il codice che le contiene viene
effettivamente eseguito con input appropriati o in scenari specifici.

In particolare, alcuni problemi noti presenti nel codice non sono stati segnalati
da nessuno dei due strumenti durante i test, tra cui:
\begin{itemize}
  \item Out-of-Bounds Read(\texttt{utils.c:55}): lettura di un buffer oltre i
    limiti definiti tramite \texttt{strcmp} su un puntatore con offset negativo.

  \item Dangling Pointer(\texttt{utils.c:69-76}): ritorno di un puntatore di un
    buffer statico locale.

  \item Dangling Pointer(\texttt{tracker\_core.c:56}): deallocazione di un
    puntatore all'interno di una funzione, mentre il chiamante ne conserva ancora
    una copia dangling.
\end{itemize}

Questi esempi dimostrano che, oltre a richiedere tempi di esecuzione più lunghi e
a dipendere dagli input forniti, i tool DAST non sono in grado di rilevare tutte
le vulnerabilità potenziali, soprattutto in assenza di una copertura esaustiva del
codice o in presenza di bug più sottili.

In conclusione, AddressSanitizer e Valgrind si confermano strumenti fondamentali
per l'individuazione di bug legati alla gestione della memoria, ma non possono sostituire
completamente l'analisi statica o altre forme di verifica. È quindi
consigliabile integrare diverse tecniche di analisi per ottenere una copertura
più completa e robusta del codice.