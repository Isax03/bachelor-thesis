\section{Analisi preliminare}
\label{sec:initial_analysis}

Dal momento che questo caso studio si basa su un'applicazione già scritta e funzionante,
la prima fase non sarà quella di riscrivere il codice seguendo le best practice
e usando librerie che aggiungono controlli di sicurezza, ma piuttosto quella di analizzare
il codice presente nel progetto, identificare le vulnerabilità e valutare l'efficacia
degli strumenti utilizzati per la loro individuazione.

\subsection*{Analisi statica}
Seguendo l'ordine in cui sono state introdotte le tecniche di mitigazione nel ~\autoref{cha:sdlc},
la prima fase di analisi è quella statica, che analizza il codice sorgente senza
effettivamente eseguirlo.

\paragraph{Criteri di selezione degli strumenti}
I tool di analisi sono stati scelti a partire dalla lista proposta dal \textit{National
Institute of Standards and Technology} (NIST)\footnote{Source Code Security
Analyzers: \url{https://www.nist.gov/itl/ssd/software-quality-group/source-code-security-analyzers}}
che propone quasi 90 strumenti di analisi statica, in una tabella che comprende i
linguaggi supportati, le funzionalità offerte, la disponibilità (gratuita o meno)
e la data dell'ultimo aggiornamento. Grazie a queste informazioni, è stato possibile
restringere la scelta degli strumenti da utilizzare, usando i seguenti criteri:
\begin{itemize}
  \item \textbf{Linguaggio supportato}: rimuovendo dalla lista gli strumenti che
    non supportano il linguaggio C, si è ottenuta una lista di circa 50 tool

  \item \textbf{Disponibilità gratuita}: dal momento che si tratta di un caso studio
    didattico, si è scelto di utilizzare solo strumenti gratuiti, eliminando dalla
    lista quelli a pagamento o con licenza commerciale. Questo ha ridotto
    ulteriormente la lista a 16 strumenti.

  \item \textbf{Data dell'ultimo aggiornamento}: per garantire l'affidabilità degli
    strumenti, si è scelto di considerare solo quelli aggiornati dal 2010 in poi.
    Questo ha dimezzato la lista a otto strumenti.

  \item \textbf{Focus sulla memory safety}: sono stati infine esclusi gli strumenti
    che non avevano la sicurezza della memoria tra i principali obiettivi e quelli
    troppo general purpose.
\end{itemize}

Questa scrematura ha portato a una lista finale di tre strumenti: \textbf{CppCheck},
un analizzatore statico di codice C/C++ focalizzato su errori di programmazione e
vulnerabilità di sicurezza; \textbf{Clang Static Analyzer}, parte del progetto
Clang che fornisce funzionalità avanzate per il rilevamento di errori e
vulnerabilità; e \textbf{Frama-C}, un analizzatore statico di codice C che si concentra
sulla sicurezza della memoria e sulla rilevazione di vulnerabilità comuni.

\medskip
\subsubsection*{\small CppCheck\footnote{Sito: \url{https://cppcheck.sourceforge.io/}}}
L'analizzatore \texttt{cppcheck} è il primo strumento che è stato utilizzato per
l'analisi statica del codice sorgente. Esso è in grado di rilevare errori di programmazione,
vulnerabilità di sicurezza e altri problemi nel codice C/C++.

\noindent
Lo strumento è stato eseguito con il seguente comando: \begin{lstlisting}[language=bash, numbers=none]
cppcheck --enable=warning --inconclusive --std=c11 *.c
\end{lstlisting}
Il comando specifica di abilitare i warning e di considerare anche i casi in cui
non è possibile determinare con certezza se un codice sia corretto o meno (\texttt{--inconclusive}),
e di utilizzare lo standard C11 per l'analisi.

L'analisi ha prodotto risultati limitati. CppCheck ha identificato
esclusivamente un potenziale memory leak, fallendo nel rilevare le vulnerabilità
più critiche come il buffer overflow e il double free presenti nel codice.

\begin{lstlisting}[language={}, numbers=none]
utils.c:17:9 error: Memory leak: info [memleak]
  return NULL;
  ^
\end{lstlisting}

Questo risultato è stato inaspettato, poiché ci si aspettava che lo strumento rilevasse
almeno parte delle vulnerabilità presenti nel codice, come ad esempio il rischio
di buffer overflow o l'evidente double free presente nel file \texttt{tracker\_core.c}.
Per questo motivo, è consigliato utilizzare più strumenti di analisi statica nei
progetti reali, per ottenere risultati più completi e affidabili.

\medskip

\subsubsection*{\small Clang\footnote{Sito: \url{https://clang.llvm.org/}}}
Un altro strumento di analisi statica utilizzato è \texttt{clang}, che è un compilatore
con funzionalità di analisi statica integrate. Esso, come \texttt{cppcheck}, è in
grado di rilevare errori di programmazione e vulnerabilità di sicurezza nel codice
C/C++.

Anche in questo caso, lo strumento è stato eseguito da terminale e l'analisi
statica del codice è possibile con i seguente comando: \begin{lstlisting}[language={}, numbers=none]
clang --analyze -Xanalyzer -analyzer-checker=core,security *.c
\end{lstlisting}
Il comando specifica di eseguire l'analisi statica del codice e di abilitare i checker
per la sicurezza e per i \textit{core issue}.

L'analisi ha prodotto risultati più interessanti rispetto a \texttt{cppcheck},
evidenziando diversi problemi nel codice:
\begin{itemize}
  \item \textbf{Funzioni non sicure}: gran parte dell'output del comando evidenzia
    l'uso di funzioni non sicure, in particolare \texttt{strcpy}, \texttt{sprintf}
    e \texttt{fprintf}. In più, \texttt{clang} suggerisce funzioni alternative più
    sicure, come \texttt{strlcpy}, \texttt{sprintf\_s} e \texttt{fprintf\_s}, che
    dovrebbero essere utilizzate per evitare vulnerabilità di buffer overflow.

  \item \textbf{Memory Leak}: viene rilvato lo stesso memory leak già evidenziato
    da \texttt{cppcheck}

  \item \textbf{Double Free}: al contrario di \texttt{cppcheck}, \texttt{clang}
    riesce a rilevare l'ovvio problema di double free presente nel file \texttt{tracker\_core.c}
\end{itemize}

\medskip

\subsubsection*{\small Frama-C\footnote{Sito: \url{https://frama-c.com/}}}
L'ultimo strumento di analisi statica utilizzato sul progetto File Tracker è \texttt{Frama-C},
un analizzatore che mira a combinare più tecniche di analisi, fornite tramite
plugin, per massimizzare l'eliminazione di bug e vulnerabilità nel codice C.

Frama-C è un analizzatore più complesso rispetto ai precedenti, poiché prevede
sia uno strumento che lavora da linea di comando (come gli altri due), sia un'interfaccia
grafica che permette di visualizzare i risultati dell'analisi in modo più interattivo.
In entrambi i casi, l'output richiede più sforzo per essere compreso, poiché i
risultati sono presentati in modo più dettagliato e con terminologia più tecnica.

Avviando l'interfaccia grafica di Frama-C con il comando
\begin{lstlisting}[language=bash, numbers=none]
frama\-c\-gui \-eva *.c
\end{lstlisting}
si ottiene una visualizzazione del codice sorgente con i risultati dell'analisi
evidenziati direttamente sul codice. In questo modo, è possibile navigare tra le
funzioni e i file del progetto, visualizzando i problemi rilevati in modo interattivo.

L'analisi, in questo caso, ha prodotto risultati "peggiori" rispetto agli altri due
strumenti, dal momento che la maggior parte del codice analizzato ha segnalato
un risultati inconcludenti, ovvero che non è stato possibile determinare con certezza
se il codice sia corretto o meno. L'unico bug che Frama-C è riuscito a segnalare
è stato lo use-after-free nelle ultime righe del file \texttt{main.c}.

\bigskip
\noindent
\paragraph{Riepilogo dell'analisi statica}
L'analisi statica del codice sorgente del progetto File Tracker, condotta utilizzando
tre strumenti distinti, ha evidenziato diverse criticità nel codice. Come mostrato
nella ~\autoref{tab:static_analysis_results}, ciascun strumento ha dimostrato punti
di forza e limitazioni specifiche: CppCheck ha identificato principalmente
memory leak, Clang ha mostrato la migliore capacità di rilevamento generale segnalando
funzioni non sicure e double free, mentre Frama-C ha evidenziato use-after-free
non rilevati dagli altri strumenti. Complessivamente, nessun singolo strumento è
riuscito a identificare tutte le vulnerabilità presenti, confermando l'importanza
di utilizzare approcci complementari nell'analisi statica. I risultati "Parziale"
nella tabella indicano i warning di Clang sull'utilizzo di funzioni non sicure
come \texttt{strcpy} e \texttt{sprintf}, dal momento che l'analizzatore non fornisce
informazioni sulla vulnerabilità specifica, ma che comunque dà un'indicazione
sul fatto che il codice potrebbe essere vulnerabile.

\begin{table}[htbp]
  \centering
  \begin{tabular}{|l|l|l|l|l|}
    \hline
    \textbf{Bug/Vulnerabilità} & \textbf{Posizione}                     & \textbf{CppCheck}              & \textbf{Clang}                 & \textbf{Frama-C}               \\
    \hline
    Out-of-Bounds Write        & \texttt{main.c:27}, \texttt{main.c:42} & \cellcolor{red!20}Non Rilevato & \cellcolor{yellow!20}Parziale  & \cellcolor{red!20}Non Rilevato \\
    \hline
    Use After Free             & \texttt{main.c:127}                    & \cellcolor{red!20}Non Rilevato & \cellcolor{red!20}Non Rilevato & \cellcolor{green!20}Rilevato   \\
    \hline
    Memory Leak                & \texttt{utils.c:14-18}                 & \cellcolor{green!20}Rilevato   & \cellcolor{green!20}Rilevato   & \cellcolor{red!20}Non Rilevato \\
    \hline
    Out-of-Bounds Read         & \texttt{utils.c:56}                    & \cellcolor{red!20}Non Rilevato & \cellcolor{red!20}Non Rilevato & \cellcolor{red!20}Non Rilevato \\
    \hline
    Dangling Pointer           & \texttt{utils.c:72-79}                 & \cellcolor{red!20}Non Rilevato & \cellcolor{red!20}Non Rilevato & \cellcolor{red!20}Non Rilevato \\
    \hline
    Buffer Overflow            & \texttt{tracker\_core.c:14,18}         & \cellcolor{red!20}Non Rilevato & \cellcolor{yellow!20}Parziale  & \cellcolor{red!20}Non Rilevato \\
    \hline
    Double Free                & \texttt{tracker\_core.c:45}            & \cellcolor{red!20}Non Rilevato & \cellcolor{green!20}Rilevato   & \cellcolor{red!20}Non Rilevato \\
    \hline
    Dangling Pointer           & \texttt{tracker\_core.c:56}            & \cellcolor{red!20}Non Rilevato & \cellcolor{red!20}Non Rilevato & \cellcolor{red!20}Non Rilevato \\
    \hline
    Out-of-Bounds Write        & \texttt{tracker\_core.c:81}            & \cellcolor{red!20}Non Rilevato & \cellcolor{yellow!20}Parziale  & \cellcolor{red!20}Non Rilevato \\
    \hline
    Use After Free             & \texttt{tracker\_core.c:161}           & \cellcolor{red!20}Non Rilevato & \cellcolor{red!20}Non Rilevato & \cellcolor{red!20}Non Rilevato \\
    \hline
    Out-of-Bounds Write        & \texttt{tracker\_core.c:194}           & \cellcolor{red!20}Non Rilevato & \cellcolor{yellow!20}Parziale  & \cellcolor{red!20}Non Rilevato \\
    \hline
    Out-of-Bounds Write        & \texttt{tracker\_core.c:216}           & \cellcolor{red!20}Non Rilevato & \cellcolor{yellow!20}Parziale  & \cellcolor{red!20}Non Rilevato \\
    \hline
  \end{tabular}
  \caption{Risultati dell'Analisi Statica}
  \label{tab:static_analysis_results}
\end{table}

\subsection*{Analisi dinamica}