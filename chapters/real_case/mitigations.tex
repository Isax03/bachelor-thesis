\section{Applicazione delle mitigazioni}
\label{sec:mitigation_techniques}

Una volta identificate le vulnerabilità presenti nel progetto File Tracker, si
procede con l'applicazione delle tecniche di mitigazione illustrate nella~\autoref{sec:development}.

In questa fase vengono considerati due approcci distinti: da un lato, l'adozione
delle best practice per una gestione sicura della memoria (\autoref{sec:best-practices-codice});
dall'altro, l'impiego di librerie esterne progettate per rafforzare la sicurezza
delle operazioni su stringhe e memoria (\autoref{sec:librerie}). Entrambi gli approcci
saranno applicati al codice responsabile della vulnerabilità di buffer overflow identificata
nel primo scenario di test dell'analisi dinamica, situata nelle righe 14 e 18
del file \texttt{tracker\_core.c} (disponibile nella repository del progetto).

È importante notare che, poiché il progetto è di dimensioni contenute, molte delle
vulnerabilità potrebbero essere mitigate efficacemente tramite un refactoring mirato,
anche senza ricorrere a strumenti esterni. Tuttavia, per finalità didattiche, l'utilizzo
di librerie verrà comunque applicato e comparato con le best practice.

\subsection*{Best Practice}
\label{sec:best-practices-case-study}

Partendo dalla guida di OWASP~\cite{owasp_best_practices}, si può notare che il
codice preso in considerazione, non rispetta nessuna delle linee guida citate nella~\autoref{sec:best-practices-codice}.
In particolare:
\begin{itemize}
  \item non viene effettuato alcun controllo sulla lunghezza delle stringhe di input;

  \item le stringhe non vengono troncate correttamente nel caso in cui la lunghezza
    superi il limite previsto;

  \item non viene liberata la memoria allocata dinamicamente in caso di errore;

  \item vengono utilizzate funzioni di manipolazione delle stringhe comunemente considerate
    vulnerabili (\texttt{strcpy}).
\end{itemize}

Per mitigare la vulnerabilità di buffer overflow, si è proceduto a un refactoring
del codice, seguendo questi quattro punti, ottenendo il codice mostrato nel branch
git \texttt{mitigations/best-practices} della repository del progetto. In particolare
le modifiche consistono nell'introduzione di controlli sulla lunghezza delle stringhe
e nell'uso della variante più sicura di \texttt{strcpy}, ovvero \texttt{strncpy},
che permette di specificare il numero massimo di caratteri da copiare; è stato
inoltre forzato l'inserimento di un terminatore nullo alla fine della stringa, per
evitare che la stringa possa essere troncata senza un terminatore, nonostante la
lunghezza sia corretta; infine, sono stati scritti controlli per gestire
eventuali errori di allocazione della memoria, liberando le risorse allocate precedentemente.

\subsection*{Librerie esterne}
\label{sec:librerie-case-study}

Dal momento che i due problemi principali del codice unsafe sono legati alla
gestione della memoria e alla manipolazione delle stringhe, si è deciso di utilizzare
le seguenti librerie:
\begin{itemize}
  \item \texttt{xmalloc}\footnote{Repository: \url{https://github.com/rosingh/xmalloc}}
    per la gestione sicura della memoria, che fornisce funzioni di allocazione
    che gestiscono automaticamente gli errori e terminano il programma in caso
    di errori;

  \item \texttt{safestringlib}\footnote{Repository: https://github.com/intel/safestringlib}
    per la manipolazione sicura delle stringhe, che fornisce funzioni di copia e
    concatenazione che gestiscono automaticamente i controlli sui limiti e le
    terminazioni delle stringhe.
\end{itemize}

Il procedimento di mitigazione in questo caso, è meno immediato rispetto alle
best practice, poiché non consiste nella semplice riscrittura del codice, ma
richiede l'installazione delle librerie. Nel contesto di questo esempio, la
configurazione non occupa troppo tempo ma, in progetti reali e più complessi, l'integrazione
delle librerie potrebbe richiedere uno sforzo non trascurabile.

Per quanto riguarda la libreria \texttt{xmalloc}, è sufficiente scaricare un
file header \texttt{xmalloc.h} e un file di implementazione \texttt{xmalloc.c} e
includerli nel progetto. La libreria \texttt{safestringlib} invece, richiede l'installazione
tramite alcuni comandi disponibili nella repository ufficiale.

\medskip
Oltre all'installazione, un altro aspetto da considerare, per quanto riguarda le
librerie, è la compilazione del codice: infatti ci sono librerie esterne che richiedono
l'aggiunta di flag specifici per la compilazione. In questo caso specifico, è necessario
aggiungere il seguente testo ai comandi di compilazione:
\begin{lstlisting}[language=bash, numbers=none]
./lib/xmalloc/xmalloc.c -lsafestring-shared
\end{lstlisting}
Questa parte include infatti, il file \texttt{xmalloc.c} (salvato nella cartella
\texttt{lib/xmalloc/}) e collega la libreria \texttt{safestringlib}, con la flag
che indica al compilatore di cercare la libreria durante la fase di linking.

Dopo aver installato le librerie, il codice è stato modificato come mostrato nel
branch git \\ \texttt{mitigations/libraries} della repository del progetto.

Le modifiche principali consistono nell'utilizzo delle funzioni \texttt{xmalloc}
e \texttt{xfree} per la gestione della memoria, che gestiscono automaticamente
gli errori di allocazione e liberazione della memoria, e nell'uso della funzione
\texttt{strncpy\_s} della libreria \texttt{safestringlib}, che permette di
copiare le stringhe in modo simile a \texttt{strncpy}, ma con controlli
automatici sui limiti e con la terminazione esplicita della stringa destinazione,
restituendo codici specifici per ogni caso di errore.

\subsection*{Test delle mitigazioni}
\label{sec:test-mitigations}

Poiché che la porzione di codice modificata è già stata testata durante l'analisi
dinamica, si può procedere a riprodurre lo stesso scenario per verificare che la
funzione \texttt{tracker\_init} si comporti correttamente dopo le mitigazioni.

Eseguendo quindi gli stessi comandi utilizzati in precedenza, sia per AddressSanitizer
che per Valgrind, si ottiene un risultato simile per entrambi gli analizzatori: non
vengono rilevate anomalie nell'esecuzione del programma e, grazie ai controlli
implementati, il programma termina correttamente in modo controllato, senza
bloccarsi.

\subsection*{Confronto tra le due soluzioni}
\label{sec:comparison-case-study}

Dopo aver applicato le mitigazioni, è possibile confrontare le due soluzioni per
valutare i pro e i contro di ciascun approccio.

Partendo dalle best practice, un aspetto sicuramente positivo è il fatto che permette
di mantenere il codice sotto il pieno controllo dello sviluppatore. Inoltre, questo
approccio non richiedel'integrazione di componenti esterni, rendendo il progetto
più leggero, facilmente portabile e privo di dipendenze aggiuntive. D'altra
parte, richiede però una maggiore attenzione e competenza da parte dello sviluppatore,
poiché è facile introdurre errori anche banali. Un altro aspetto negativo riguarda
la manutenzione: essa può risultare più onerosa nel tempo, soprattutto se non esistono
linee guida interne chiare e consolidate.

Al contrario, l'utilizzo di librerie esterne introduce una forma di astrazione: molte
delle operazioni a rischio sono delegate a componenti progettati appositamente per
gestirle in sicurezza. Questo comporta diversi vantaggi, tra cui una maggiore robustezza
e un minor rischio di vulnerabilità dovute a errori umani. Di contro, può aumentare
la complessità, sia in fase di sviluppo (configurazione, compilazione) sia di manutenzione,
in quanto le librerie possono evolversi e richiedere aggiornamenti o modifiche
al codice.

In sintesi, l'approccio basato su best practice è più adatto per progetti piccoli
o per sviluppatori esperti che desiderano un controllo completo sul codice,
mentre l'utilizzo di librerie esterne è preferibile in contesti dove la
sicurezza, la manutenibilità e la rapidità di sviluppo sono prioritarie. In
altri ambienti invece, com'è stato evidenziato più volte in questa tesi,
potrebbe essere vantaggioso un approccio ibrido che combini le due strategie in
modo tale da massimizzare i benefici e minimizzare i rischi.