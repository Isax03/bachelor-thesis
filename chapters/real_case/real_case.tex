\chapter{Caso di studio}
\label{cha:real_case}

\section{Presentazione}
\label{sec:presentation}

Dopo aver illustrato nei capitoli precedenti le principali vulnerabilità legate alla
gestione della memoria e le relative tecniche di mitigazione, questo capitolo
presenta un caso di studio pratico, volto a dimostrare come i concetti teorici possano
essere applicati concretamente a un progetto reale.

\subsection*{Metodologia di selezione del progetto}

Per condurre un'analisi efficace delle vulnerabilità di memory safety, esistono due
alternative principali nella scelta del caso di studio: utilizzare un progetto
open source esistente oppure crearne uno appositamente per scopi dimostrativi.

L'analisi di progetti open source presenta tuttavia alcune problematiche
ricorrenti:
\begin{itemize}
  \item I progetti realmente utilizzati in produzione risultano già abbastanza robusti
    e testati, rendendo poco significativa l'analisi di vulnerabilità.

  \item I progetti ``educational'' sono spesso troppo semplificati e artificiali
    per rappresentare scenari realistici di sviluppo software.
\end{itemize}

In considerazione di queste limitazioni, si è optato per la seconda alternativa:
la creazione di un nuovo progetto realistico ma deliberatamente vulnerabile. A tal
fine, è stato impiegato \textit{GitHub Copilot}\footnote{\url{https://github.com/features/copilot}}
in modalità \textit{Agent}, richiedendo esplicitamente la generazione di codice
per un'applicazione plausibile, scritta in C, che includesse vulnerabilità
comuni di memory safety a fini educativi e dimostrativi. Per trasparenza, il
prompt utilizzato è disponibile in appendice (\autoref{appendix:prompt}).

\subsection*{Descrizione del progetto}

Il progetto risultante è un \textbf{file tracker}, ovvero un'applicazione a riga
di comando che monitora una directory specificata, rilevando in tempo reale la
creazione, modifica e cancellazione di file. Il programma supporta le seguenti
funzionalità:
\begin{itemize}
  \item Filtraggio dei file mediante pattern

  \item Configurazione dell'intervallo di polling

  \item Logging su file

  \item Modalità verbosa per debug
\end{itemize}

Nonostante la sua apparente semplicità, il codice è strutturato in più moduli e include
numerosi bug intenzionali che rappresentano vulnerabilità tipiche del linguaggio
C.

Una descrizione dettagliata di ciascuna vulnerabilità è disponibile nel file \texttt{README.md}
del progetto\footnote{Progetto GitHub: \url{https://github.com/Isax03/unsafe-file-tracker}}.

Nel corso del capitolo vengono illustrati:
\begin{itemize}
  \item I principali difetti rilevati tramite analisi statica

  \item Tre scenari di test che scatenano il rilevamento di errori con l'analisi
    dinamica

  \item Il confronto tra l'applicazione di best practices e l'uso di librerie per
    mitigare la vulnerabilità del primo scenario
\end{itemize}

Questo caso di studio rappresenta quindi una dimostrazione concreta dell'approccio
proposto nella tesi: identificare, analizzare e mitigare vulnerabilità di memoria
lungo il ciclo di vita del software, anche in assenza di exploit attivi.

\section{Analisi preliminare}
\label{sec:initial_analysis}

Dal momento che questo caso studio si basa su un'applicazione già scritta e funzionante,
la prima fase non sarà quella di riscrivere il codice seguendo le best practice
e usando librerie che introducono controlli di sicurezza, ma piuttosto quella di
analizzare il codice esistente, identificare le vulnerabilità e valutare l'efficacia
degli strumenti utilizzati per la loro individuazione.

\subsection*{Analisi statica}
Seguendo l'ordine in cui sono state introdotte le tecniche di mitigazione nel~\autoref{cha:sdlc},
la prima fase di analisi è quella statica, che analizza il codice sorgente senza
effettivamente eseguirlo.

\paragraph{Criteri di selezione degli strumenti}
I tool di analisi sono stati scelti a partire dalla lista proposta dal \textit{National
Institute of Standards and Technology} (NIST)\cite{nist_sast_list} che propone quasi
90 strumenti di analisi statica, in una tabella che comprende i linguaggi
supportati, le funzionalità offerte, la disponibilità (gratuita o meno) e la
data dell'ultimo aggiornamento. Grazie a queste informazioni, è stato possibile
restringere la scelta degli strumenti da utilizzare, usando i seguenti criteri:
\begin{itemize}
  \item \textbf{Linguaggio supportato}: rimuovendo dalla lista gli strumenti che
    non supportano il linguaggio C, si è ottenuta una lista di circa 50 tool.

  \item \textbf{Disponibilità gratuita}: dal momento che si tratta di un caso studio
    didattico, si è scelto di utilizzare solo strumenti gratuiti, eliminando dalla
    lista quelli a pagamento o con licenza commerciale. Questo ha ridotto
    ulteriormente la lista a 16 strumenti.

  \item \textbf{Data dell'ultimo aggiornamento}: per garantire l'affidabilità degli
    strumenti, si è scelto di considerare solo quelli aggiornati dal 2010 in poi.
    Questo ha dimezzato la lista a otto strumenti.

  \item \textbf{Focus sulla memory safety}: sono stati infine esclusi gli strumenti
    che non avevano la sicurezza della memoria tra i principali obiettivi e quelli
    troppo general purpose.
\end{itemize}

Questa scrematura ha portato a una lista finale di tre strumenti: \textbf{CppCheck},
un analizzatore statico di codice C/C++ focalizzato su errori di programmazione
e vulnerabilità di sicurezza; \textbf{Clang Static Analyzer}, parte del progetto
Clang che fornisce funzionalità avanzate per il rilevamento di errori e vulnerabilità;
e \textbf{Frama-C}, un analizzatore statico di codice C che si concentra sulla
sicurezza della memoria e sulla rilevazione di vulnerabilità comuni.

\subsubsection*{CppCheck}
L'analizzatore \texttt{cppcheck}\footnote{Sito: \url{https://cppcheck.sourceforge.io/}}
è il primo strumento che è stato utilizzato per l'analisi statica del codice
sorgente. Esso è in grado di rilevare errori di programmazione, vulnerabilità di
sicurezza e altri problemi nel codice C/C++.

Lo strumento è stato eseguito con il seguente comando: \begin{lstlisting}[language=bash, numbers=none]
cppcheck --enable=warning --inconclusive --std=c11 *.c
\end{lstlisting}
Il comando specifica di abilitare i warning e di considerare anche i casi in cui
non è possibile determinare con certezza se un codice sia corretto o meno (\texttt{--inconclusive}),
e di utilizzare lo standard C11 per l'analisi.

L'analisi ha prodotto risultati limitati. CppCheck ha identificato
esclusivamente un potenziale memory leak, fallendo nel rilevare le vulnerabilità
più critiche come il buffer overflow e il double free presenti nel codice:

\begin{lstlisting}[language={}, numbers=none]
utils.c:17:9 error: Memory leak: info [memleak]
  return NULL;
  ^
\end{lstlisting}

Questo risultato è stato inaspettato, poiché si pensava che lo strumento rilevasse
almeno parte delle vulnerabilità presenti nel codice, come ad esempio il rischio
di buffer overflow o l'evidente double free presente nel file \texttt{tracker\_core.c}.
Per questo motivo, è consigliabile utilizzare più strumenti di analisi statica nei
progetti reali, per ottenere risultati più completi e affidabili.

\subsubsection*{Clang Static Analyzer}
Un altro strumento utilizzato è \texttt{clang}\footnote{Sito: \url{https://clang.llvm.org/}},
che è un compilatore con funzionalità di analisi statica integrate. Esso, come \texttt{cppcheck},
è in grado di rilevare errori di programmazione e vulnerabilità di sicurezza nel
codice C e C++.

Anche in questo caso, lo strumento è stato eseguito dal terminale e l'analisi
statica del codice è possibile con il seguente comando: \begin{lstlisting}[language=bash, numbers=none]
clang --analyze -Xanalyzer -analyzer-checker=core,security *.c
\end{lstlisting}
Il comando specifica di eseguire l'analisi statica del codice e di abilitare i checker
per la sicurezza e per i \textit{core issue}.

L'analisi ha prodotto risultati più interessanti rispetto a \texttt{cppcheck},
evidenziando diversi problemi nel codice:
\begin{itemize}
  \item \textbf{Funzioni non sicure}: gran parte dell'output del comando evidenzia
    l'uso di funzioni non sicure, in particolare \texttt{strcpy}, \texttt{sprintf}
    e \texttt{fprintf}. Inoltre, \texttt{clang} suggerisce funzioni alternative più
    sicure, come \texttt{strlcpy}, \texttt{sprintf\_s} e \texttt{fprintf\_s}, che
    dovrebbero essere utilizzate per evitare vulnerabilità di buffer overflow.

  \item \textbf{Memory Leak}: viene rilevato lo stesso memory leak già evidenziato
    da \texttt{cppcheck}.

  \item \textbf{Double Free}: al contrario di \texttt{cppcheck}, \texttt{clang}
    riesce a rilevare l'ovvio problema di double free presente nel file \texttt{tracker\_core.c}.
\end{itemize}

\subsubsection*{Frama-C}
L'ultimo strumento di analisi statica utilizzato sul progetto File Tracker è
\texttt{Frama-C}\footnote{Sito: \url{https://frama-c.com/}}, un analizzatore che
mira a combinare più tecniche di analisi, fornite tramite plugin, per massimizzare
l'eliminazione di bug e vulnerabilità nel codice C.

Frama-C è un analizzatore più avanzato rispetto ai precedenti, poiché prevede
sia uno strumento che lavora da linea di comando (come gli altri due), sia un'interfaccia
grafica che permette di visualizzare i risultati dell'analisi in modo più interattivo.
In entrambi i casi, l'output richiede più sforzo per essere compreso, poiché i
risultati sono presentati in modo più dettagliato e con terminologia più tecnica.

L'interfaccia grafica di Frama-C è stata avviata con il comando: \begin{lstlisting}[language=bash, numbers=none]
frama-c-gui -eva *.c
\end{lstlisting}
ottenendo una visualizzazione del codice sorgente con i risultati dell'analisi evidenziati
direttamente sul codice. In questo modo, è possibile navigare tra le funzioni e i
file del progetto, consentendo una visualizzazione interattiva dei problemi
rilevati.

L'analisi, in questo caso, ha prodotto risultati meno soddisfacenti rispetto
agli altri due strumenti, dal momento che la maggior parte del codice analizzato
ha segnalato risultati inconcludenti, ovvero che non è stato possibile
determinare con certezza se il codice sia corretto o meno. L'unico bug che Frama-C
è riuscito a segnalare è stato lo use-after-free nelle ultime righe del file \texttt{main.c}.

\subsubsection*{Riepilogo dell'analisi statica}
L'analisi statica del codice sorgente del progetto File Tracker, condotta utilizzando
tre strumenti distinti, ha evidenziato diverse criticità nel codice. Come mostrato
nella~\autoref{tab:static_analysis_results}, ciascun strumento ha dimostrato
punti di forza e limitazioni specifiche: CppCheck ha identificato principalmente
memory leak, Clang ha mostrato la migliore capacità di rilevamento generale
segnalando funzioni non sicure e double free, mentre Frama-C ha evidenziato use-after-free
non rilevati dagli altri strumenti. Complessivamente, nessun singolo strumento è
riuscito a identificare tutte le vulnerabilità presenti, confermando l'importanza
di utilizzare approcci complementari nell'analisi statica. I risultati ``Parziale''
nella tabella indicano i warning di Clang sull'utilizzo di funzioni non sicure come
\texttt{strcpy} e \texttt{sprintf}, dal momento che l'analizzatore non fornisce
informazioni specifiche sulla vulnerabilità in particolare, ma segnala che il codice
potrebbe essere a rischio.

\begin{table}[htbp]
  \small
  \centering
  \begin{tabular}{|l|l|l|l|l|}
    \hline
    \textbf{Bug/Vulnerabilità} & \textbf{Posizione}             & \textbf{CppCheck}              & \textbf{Clang}                 & \textbf{Frama-C}               \\
    \hline
    Out-of-Bounds Write        & \texttt{main.c:27,42}          & \cellcolor{red!20}Non Rilevato & \cellcolor{yellow!20}Parziale  & \cellcolor{red!20}Non Rilevato \\
    \hline
    Use After Free             & \texttt{main.c:127}            & \cellcolor{red!20}Non Rilevato & \cellcolor{red!20}Non Rilevato & \cellcolor{green!20}Rilevato   \\
    \hline
    Memory Leak                & \texttt{utils.c:14-18}         & \cellcolor{green!20}Rilevato   & \cellcolor{green!20}Rilevato   & \cellcolor{red!20}Non Rilevato \\
    \hline
    Out-of-Bounds Read         & \texttt{utils.c:55}            & \cellcolor{red!20}Non Rilevato & \cellcolor{red!20}Non Rilevato & \cellcolor{red!20}Non Rilevato \\
    \hline
    Dangling Pointer           & \texttt{utils.c:69-76}         & \cellcolor{red!20}Non Rilevato & \cellcolor{red!20}Non Rilevato & \cellcolor{red!20}Non Rilevato \\
    \hline
    Buffer Overflow            & \texttt{tracker\_core.c:14,18} & \cellcolor{red!20}Non Rilevato & \cellcolor{yellow!20}Parziale  & \cellcolor{red!20}Non Rilevato \\
    \hline
    Double Free                & \texttt{tracker\_core.c:45}    & \cellcolor{red!20}Non Rilevato & \cellcolor{green!20}Rilevato   & \cellcolor{red!20}Non Rilevato \\
    \hline
    Dangling Pointer           & \texttt{tracker\_core.c:56}    & \cellcolor{red!20}Non Rilevato & \cellcolor{red!20}Non Rilevato & \cellcolor{red!20}Non Rilevato \\
    \hline
    Out-of-Bounds Write        & \texttt{tracker\_core.c:81}    & \cellcolor{red!20}Non Rilevato & \cellcolor{yellow!20}Parziale  & \cellcolor{red!20}Non Rilevato \\
    \hline
    Use After Free             & \texttt{tracker\_core.c:161}   & \cellcolor{red!20}Non Rilevato & \cellcolor{red!20}Non Rilevato & \cellcolor{red!20}Non Rilevato \\
    \hline
    Out-of-Bounds Write        & \texttt{tracker\_core.c:194}   & \cellcolor{red!20}Non Rilevato & \cellcolor{yellow!20}Parziale  & \cellcolor{red!20}Non Rilevato \\
    \hline
    Out-of-Bounds Write        & \texttt{tracker\_core.c:216}   & \cellcolor{red!20}Non Rilevato & \cellcolor{yellow!20}Parziale  & \cellcolor{red!20}Non Rilevato \\
    \hline
  \end{tabular}
  \caption{Risultati dell'analisi statica}
  \label{tab:static_analysis_results}
\end{table}

\subsection*{Analisi dinamica}
Dopo aver analizzato il codice sorgente, la fase successiva è osservare il comportamento
del programma durante l'esecuzione tramite l'analisi dinamica.

Seguendo criteri simili a quelli utilizzati per la selezione degli strumenti di analisi
statica, sono stati scelti due strumenti di analisi dinamica, entrambi già citati
nella~\autoref{sec:analisi-dinamica}:
\begin{itemize}
  \item \textbf{AddressSanitizer (ASan)}, uno strumento di analisi dinamica integrato
    in Clang e GCC, progettato per rilevare errori di memoria come buffer
    overflow, use-after-free e memory leak.

  \item \textbf{Valgrind}, un framework di analisi dinamica che fornisce diversi
    strumenti per il rilevamento di errori di memoria, come memory leak, buffer overflow
    e use-after-free, grazie al suo strumento \texttt{memcheck}.
\end{itemize}

\noindent
Poiché entrambi gli strumenti di analisi dinamica dipendono strettamente dagli input
forniti al programma durante l'esecuzione e richiedono scenari specifici per attivare
le diverse vulnerabilità, sono stati testati solamente tre casi d'uso mirati che
permettono agli analizzatori di rilevare le seguenti vulnerabilità:
\begin{itemize}
  \item \textbf{Buffer Overflow} (\texttt{tracker\_core.c:14,18}), quando la
    lunghezza del percorso del file di log o della directory monitorata supera
    il limite di 255 caratteri.

  \item \textbf{Use After Free} (\texttt{tracker\_core.c:161}), quando il
    programma prova a stampare il nome di un file eliminato dalla directory
    monitorata dopo aver liberato la memoria allocata per il nome del file.

  \item \textbf{Double Free} (\texttt{tracker\_core.c:45}), quando il programma
    libera due volte la memoria allocata per la stringa del percorso della
    directory.
\end{itemize}

\subsubsection*{AddressSanitizer}
AddressSanitizer\footnote{Sito: \url{https://github.com/google/sanitizers/wiki/AddressSanitizer}},
come già menzionato nel~\autoref{cha:sdlc}, non si limita ad agire sul binario compilato,
ma richiede l'accesso al codice sorgente, che verrà arricchito con i controlli necessari
a tempo di compilazione. Infatti, il comando utilizzato per compilare il progetto
con AddressSanitizer è il seguente:
\begin{lstlisting}[language=bash, numbers=none]
clang -fsanitize=address -g -O0 -std=c11 -o file_tracker_asan main.c tracker_core.c utils.c
\end{lstlisting}

Il comando specifica di abilitare AddressSanitizer (\texttt{-fsanitize=address}),
di includere le informazioni di debug (\texttt{-g}), di disabilitare le ottimizzazioni
(\texttt{-O0}) e di utilizzare lo standard C11 (\texttt{-std=c11}).

Dopo la compilazione, il programma può essere eseguito in cerca di vulnerabilità
o bug. Nello specifico, per testare le vulnerabilità selezionate, è necessaria la
creazione di contesti specifici. Di seguito sono riportati i comandi utilizzati
per l'impostazione di ciascun contesto, in base alla vulnerabilità da testare:
\begin{itemize}
  \item \textbf{Buffer Overflow}: per testare il buffer overflow, è sufficiente
    fornire un percorso di directory o di file di log che superi i 255 caratteri.
    Poiché il tracker verifica l'esistenza della directory monitorata, è
    necessario creare una struttura di directory annidate sufficientemente
    profonda da raggiungere il limite di 255 caratteri. A fini di riproducibilità,
    sono stati utilizzati i seguenti comandi: \begin{lstlisting}[language=bash, numbers=none]
# creazione della directory
mkdir -p $(printf 'asan/asan1/asan2/%.0s' {1..15}) # crea directory annidate
# esecuzione del tracker con un percorso di directory lungo
./file_tracker_asan ./$(printf 'asan/asan1/asan2/%.0s' {1..15})
# oppure con un file di log nella directory appena creata
./file_tracker_asan ./ -l ./$(printf 'asan/asan1/asan2/%.0s' {1..15})logfile.log
    \end{lstlisting}

  \item \textbf{Use After Free}: per testare l'use-after-free, è sufficiente
    creare una directory e un file, eseguire il tracker ed eliminare il file durante
    l'esecuzione:
    \begin{lstlisting}[language=bash, numbers=none]
# creazione della directory e del file
mkdir asan && touch ./asan/asan.txt
# esecuzione del tracker per monitorare la directory ./asan/
./file_tracker_asan ./asan/
    \end{lstlisting}

  \item \textbf{Double Free}: per quanto riguarda il double free, il processo è
    ancora più semplice, poiché è sufficiente eseguire il tracker su una directory
    arbitraria e poi terminare il programma premendo \texttt{Ctrl+C} per forzare
    la chiusura: \begin{lstlisting}[language=bash, numbers=none]
# esecuzione del tracker su una directory arbitraria
./file_tracker_asan ./asan/
# terminazione del programma con Ctrl+C
    \end{lstlisting}
    In generale, il double free si verifica ogni volta che il programma termina
    a causa di un segnale \texttt{SIGINT} (come \texttt{Ctrl+C}).
\end{itemize}

Riproducendo questi tre scenari, AddressSanitizer è stato in grado di rilevare
tutte e tre le vulnerabilità, restituendo output dettagliati e informativi per ciascun
caso. Un esempio rappresentativo dell'output generato è mostrato nell'\autoref{appendix:asan_output}
(relativo al primo scenario di test).

Le caratteristiche distintive degli output prodotti da AddressSanitizer
includono:
\begin{itemize}
  \item \textbf{Localizzazione delle vulnerabilità}: AddressSanitizer fornisce informazioni
    dettagliate sulla posizione esatta della vulnerabilità nel codice, indicando
    il file e la riga specifica in cui si verifica il problema (grazie all'opzione
    \texttt{-g} utilizzata in fase di compilazione). Questo è molto utile per la
    correzione rapida e mirata.

  \item \textbf{Classificazione delle vulnerabilità}: per ogni vulnerabilità rilevata,
    ASan fornisce una nomenclatura chiara del tipo di problema (es. ``heap-buffer-overflow'',
    ``attempting double free'', ...) che aiuta a comprendere immediatamente la natura
    del bug.

  \item \textbf{Stack trace}: per ogni vulnerabilità rilevata, AddressSanitizer
    fornisce un traceback completo dello stack delle chiamate, mostrando la sequenza
    di funzioni che ha portato al problema.

  \item \textbf{Informazioni sulla memoria}: l'output include dettagli tecnici sulla
    regione di memoria coinvolta, come l'indirizzo specifico, la dimensione del blocco
    allocato e lo stato della memoria (freed, allocated, etc.).
\end{itemize}

\subsubsection*{Valgrind}
Valgrind\footnote{Sito: \url{http://valgrind.org/}}, a differenza di
AddressSanitizer, non agisce direttamente sul codice sorgente, ma analizza l'esecuzione
del programma compilato, monitorando l'uso della memoria e rilevando errori come
buffer overflow, use-after-free e memory leak.

Per utilizzare Valgrind, è necessario compilare il progetto senza AddressSanitizer,
ma preferibilmente con le informazioni di debug attivate, in modo da ottenere un
output più dettagliato e utile per la correzione dei bug (come per ASan). Il
comando usato per questo progetto è: \begin{lstlisting}[language=bash, numbers=none]
clang -g -O0 -std=c11 -o file_tracker_valgrind main.c tracker_core.c utils.c
\end{lstlisting}

Dopo la compilazione, il programma può essere eseguito con Valgrind aggiungendo il
seguente prefisso al comando di esecuzione del programma:
\begin{lstlisting}[language=bash, numbers=none]
valgrind --tool=memcheck --leak-check=full --show-leak-kinds=all
\end{lstlisting}

Il comando specifica di utilizzare lo strumento \texttt{memcheck} di Valgrind, che
è progettato per rilevare errori di memoria, e di abilitare il controllo
completo dei leak di memoria (\texttt{--leak-check=full}), mostrando tutti i
tipi di leak (\texttt{--show-leak-kinds=all}), anche se i tre scenari di test
richiesti non richiedono necessariamente l'analisi dei leak di memoria.

Anche nel caso di Valgrind, sono stati usati gli stessi tre scenari di test per verificare
le vulnerabilità selezionate, e i log prodotti sono simili a quelli di ASan: entrambi
gli strumenti forniscono informazioni dettagliate sulla posizione delle vulnerabilità,
con uno stack trace e informazioni sulla memoria coinvolta. Tuttavia, con ASan è
più intuitivo, soprattutto per i principianti, capire la natura del bug rilevato,
poiché Valgrind fornisce un output più tecnico e meno immediato.

Un'altra differenza importante tra i due strumenti è che Valgrind non blocca l'esecuzione
nel momento in cui viene rilevato un problema, ma continua a eseguire il
programma fino alla sua conclusione, producendo un report finale con tutti i problemi
rilevati. Infatti, in tutti e tre gli scenari di test, Valgrind ha rilevato
anche uno use-after-free (\texttt{main.c:127}), non facente parte dei tre casi
di test previsti, che si verifica quando il programma prova a stampare il numero
di file monitorati dopo aver liberato la memoria allocata per la struttura del tracker.
AddressSanitizer, al contrario, interrompe l'esecuzione del programma non appena
viene rilevato un problema, evitando ulteriori errori o comportamenti imprevisti.

\subsubsection*{Considerazioni sull'analisi dinamica}
L'analisi dinamica del progetto File Tracker ha dimostrato l'efficacia di AddressSanitizer
e Valgrind nel rilevare vulnerabilità di memory safety, confermando la loro
utilità nella fase di testing e debugging. Entrambi gli strumenti hanno identificato
le falle previste, fornendo informazioni dettagliate sulla loro natura e
posizione nel codice.

Tuttavia, è importante precisare che l'analisi dinamica ha dei limiti intrinseci
come la sua dipendenza dai percorsi di esecuzione attivati durante i test. Le
vulnerabilità vengono rilevate solo se il codice che le contiene viene
effettivamente eseguito con input appropriati o in scenari specifici.

In particolare, alcuni problemi noti presenti nel codice non sono stati segnalati
da nessuno dei due strumenti durante i test, tra cui:
\begin{itemize}
  \item Out-of-Bounds Read(\texttt{utils.c:55}): lettura di un buffer oltre i
    limiti definiti tramite \texttt{strcmp} su un puntatore con offset negativo.

  \item Dangling Pointer(\texttt{utils.c:69-76}): ritorno di un puntatore di un
    buffer statico locale.

  \item Dangling Pointer(\texttt{tracker\_core.c:56}): deallocazione di un
    puntatore all'interno di una funzione, mentre il chiamante ne conserva ancora
    una copia dangling.
\end{itemize}

Questi esempi dimostrano che, oltre a richiedere tempi di esecuzione più lunghi e
a dipendere dagli input forniti, i tool DAST non sono in grado di rilevare tutte
le vulnerabilità potenziali, soprattutto in assenza di una copertura esaustiva del
codice o in presenza di bug più sottili.

In conclusione, AddressSanitizer e Valgrind si confermano strumenti fondamentali
per l'individuazione di bug legati alla gestione della memoria, ma non possono sostituire
completamente l'analisi statica o altre forme di verifica. È quindi
consigliabile integrare diverse tecniche di analisi per ottenere una copertura
più completa e robusta del codice.
\section{Applicazione delle mitigazioni}
\label{sec:mitigation_techniques}

Una volta identificate le vulnerabilità presenti nel progetto File Tracker, si
procede con l'applicazione delle tecniche di mitigazione illustrate nella~\autoref{sec:development}.

In questa fase vengono considerati due approcci distinti: da un lato, l'adozione
delle best practice per una gestione sicura della memoria (\autoref{sec:best-practices-codice});
dall'altro, l'impiego di librerie esterne progettate per rafforzare la sicurezza
delle operazioni su stringhe e memoria (\autoref{sec:librerie}). Entrambi gli approcci
saranno applicati al codice responsabile della vulnerabilità di buffer overflow identificata
nel primo scenario di test dell'analisi dinamica, situata nelle righe 14 e 18
del file \texttt{tracker\_core.c} (\autoref{lst:unsafe}).

È importante notare che, poiché il progetto è di dimensioni contenute, molte delle
vulnerabilità potrebbero essere mitigate efficacemente tramite un refactoring mirato,
anche senza ricorrere a strumenti esterni. Tuttavia, per finalità didattiche, l'utilizzo
di librerie verrà comunque applicato e comparato con le best practice.

\bigskip
\begin{lstlisting}[language=C, caption={Codice originale vulnerabile}, label={lst:unsafe}, style=changes_in_c]
file_tracker_t* tracker_init(const char* directory, const char* log_file) {
  file_tracker_t* tracker = malloc(sizeof(file_tracker_t));
  if (!tracker) {
    return NULL;
  }

  tracker->monitored_directory = malloc(256);
  strcpy(tracker->monitored_directory, directory);
  tracker->log_file_path = malloc(256);
  strcpy(tracker->log_file_path, log_file);

  // resto del codice
}
\end{lstlisting}

\subsection*{Best Practice}
\label{sec:best-practices-case-study}

Partendo dalla guida di OWASP~\cite{owasp_best_practices}, si può notare che il
codice in questione (\autoref{lst:unsafe}) non rispetta nessuna delle linee
guida citate nella~\autoref{sec:best-practices-codice}. In particolare:
\begin{itemize}
  \item non viene effettuato alcun controllo sulla lunghezza delle stringhe di input;

  \item le stringhe non vengono troncate correttamente nel caso in cui la lunghezza
    superi il limite previsto;

  \item non viene liberata la memoria allocata dinamicamente in caso di errore;

  \item vengono utilizzate funzioni di manipolazione delle stringhe comunemente considerate
    vulnerabili (\texttt{strcpy});
\end{itemize}

Per mitigare la vulnerabilità di buffer overflow, si è proceduto a un
refactoring del codice, seguendo questi quattro punti, come mostrato nel~\autoref{lst:safe-best-practices}.
In particolare le modifiche consistono nell'introduzione di controlli sulla lunghezza
delle stringhe e nell'uso della variante più sicura di \texttt{strcpy}, ovvero \texttt{strncpy},
che permette di specificare il numero massimo di caratteri da copiare; è stato
inoltre forzato l'inserimento di un terminatore nullo alla fine della stringa, per
evitare che la stringa possa essere troncata senza un terminatore, nonostante la
lunghezza sia corretta; infine, sono stati scritti controlli per gestire
eventuali errori di allocazione della memoria, liberando le risorse allocate precedentemente.

\begin{lstlisting}[language=C, caption={Codice mitigato (best practices)}, label={lst:safe-best-practices}, style=changes_in_c]
file_tracker_t* tracker_init(const char* directory, const char* log_file) {
  file_tracker_t* tracker = malloc(sizeof(file_tracker_t));
  if (!tracker) {
      return NULL;
  }
  if (strlen(directory) >= 256 || strlen(log_file) >= 256) {
      free(tracker);
      return NULL;
  }

  if((tracker->monitored_directory = malloc(256)) == NULL) {
      free(tracker);
      return NULL;
  }
  strncpy(tracker->monitored_directory, directory, 256);
  tracker->monitored_directory[255] = '\0';

  if((tracker->log_file_path = malloc(256)) == NULL) {
      free(tracker->monitored_directory);
      free(tracker);
      return NULL;
  }
  strncpy(tracker->log_file_path, log_file, 256);
  tracker->log_file_path[255] = '\0';

  // resto del codice
}
\end{lstlisting}

\subsection*{Librerie esterne}
\label{sec:librerie-case-study}

Dal momento che i due problemi principali del codice unsafe sono legati alla gestione
della memoria e alla manipolazione delle stringhe, si è deciso di utilizzare le
seguenti librerie:
\begin{itemize}
  \item \texttt{xmalloc}~\footnote{Repository: \url{https://github.com/rosingh/xmalloc}}
    per la gestione sicura della memoria, che fornisce funzioni di allocazione che
    gestiscono automaticamente gli errori e terminano il programma in caso di errori

  \item \texttt{safestringlib}~\footnote{Repository: https://github.com/intel/safestringlib}
    per la manipolazione sicura delle stringhe, che fornisce funzioni di copia e
    concatenazione che gestiscono automaticamente i controlli sui limiti e le terminazioni
    delle stringhe
\end{itemize}

Il procedimento di mitigazione in questo caso, è meno immediato rispetto alle best
practice, poiché non consiste nella semplice riscrittura del codice, ma richiede
l'installazione delle librerie. Nel contesto di questo esempio, la configurazione
non occupa troppo tempo ma, in progetti reali e più complessi, l'integrazione delle
librerie potrebbe richiedere uno sforzo non trascurabile.

Per quanto riguarda la libreria \texttt{xmalloc}, è sufficiente scaricare un file
header \texttt{xmalloc.h} e un file di implementazione \texttt{xmalloc.c} e includerli
nel progetto. La libreria \texttt{safestringlib} invece, richiede l'installazione
tramite alcuni comandi disponibili nel repository ufficiale.

\medskip
Oltre all'installazione, un altro aspetto da considerare, per quanto riguarda le
librerie, è la compilazione del codice: infatti ci sono librerie esterne che
richiedono l'aggiunta di flag specifici per la compilazione. In questo caso specifico,
è necessario aggiungere il seguente testo ai comandi di compilazione: \begin{lstlisting}[language=bash, numbers=none]
./lib/xmalloc/xmalloc.c -lsafestring-shared
\end{lstlisting}
Questa parte include infatti, il file \texttt{xmalloc.c} (salvato nella cartella
\texttt{lib/xmalloc/}) e collega la libreria\\\texttt{safestringlib}, con la
flag che indica al compilatore di cercare la libreria durante la fase di linking.

Dopo aver installato le librerie, il codice è stato modificato come mostrato nel~\autoref{lst:safe-libraries}.
\begin{lstlisting}[language=C, caption={Codice mitigato (librerie)}, label={lst:safe-libraries}, style=changes_in_c]
#include "lib/xmalloc/xmalloc.h"
#include <safe_lib.h>

file_tracker_t* tracker_init(const char* directory, const char* log_file) {
  file_tracker_t* tracker = xmalloc(sizeof(file_tracker_t));
  if (!tracker) {
    return NULL;
  }
  
  tracker->monitored_directory = xmalloc(256);
  if(strncpy_s(tracker->monitored_directory, 256, directory, strlen(directory)) != EOK){
    xfree(tracker->monitored_directory);
    xfree(tracker);
    return NULL;
  }

  tracker->log_file_path = xmalloc(256);
  if(strncpy_s(tracker->log_file_path, 256, log_file, strlen(log_file)) != EOK){
    xfree(tracker->monitored_directory);
    xfree(tracker->log_file_path);
    xfree(tracker);
    return NULL;
  }

  // resto del codice
}
\end{lstlisting}

Le modifiche principali consistono nell'utilizzo delle funzioni \texttt{xmalloc}
e \texttt{xfree} per la gestione della memoria, che gestiscono automaticamente
gli errori di allocazione e liberazione della memoria, e nell'uso della funzione
\texttt{strncpy\_s} della libreria \texttt{safestringlib}, che permette di
copiare le stringhe in modo simile a \texttt{strncpy}, ma con controlli
automatici sui limiti e con la terminazione esplicita della stringa destinazione,
restituendo codici specifici per ogni caso di errore.

\subsection*{Test delle mitigazioni}
\label{sec:test-mitigations}

Dal momento che la porzione di codice modificata è già stata testata durante l'analisi
dinamica, si può procedere a riprodurre lo stesso scenario per verificare che la
funzione \texttt{tracker\_init} si comporti correttamente dopo le mitigazioni.

Eseguendo quindi gli stessi comandi utilizzati in precedenza, sia per AddressSanitizer
che per Valgrind, si ottiene un risultato simile per entrambi gli analizzatori: non
vengono rilevate anomalie nell'esecuzione del programma e, grazie ai controlli
implementati, il programma termina correttamente in modo controllato, senza
crash.

\subsection*{Confronto tra le due soluzioni}
\label{sec:comparison-case-study}

Dopo aver applicato le mitigazioni, è possibile confrontare le due soluzioni (\autoref{lst:safe-best-practices}
e \autoref{lst:safe-libraries}) per valutare i pro e i contro di ciascun
approccio.

Partendo dalle best practice, un aspetto sicuramente positivo è il fatto che permette
di mantenere il codice sotto il pieno controllo dello sviluppatore. Inoltre, questo
approccio non richiedel'integrazione di componenti esterni, rendendo il progetto
più leggero, facilmente portabile e privo di dipendenze aggiuntive. D'altra
parte, richiede però una maggiore attenzione e competenza da parte dello sviluppatore,
poiché è facile introdurre errori anche banali. Un altro aspetto negativo riguarda
la manutenzione: essa può risultare più onerosa nel tempo, soprattutto se non esistono
linee guida interne chiare e consolidate.

Al contrario, l'utilizzo di librerie esterne introduce una forma di astrazione: molte
delle operazioni a rischio sono delegate a componenti progettati appositamente per
gestirle in sicurezza. Questo comporta diversi vantaggi, tra cui una maggiore robustezza
e un minor rischio di vulnerabilità dovute a errori umani. Di contro, può aumentare
la complessità, sia in fase di sviluppo (configurazione, compilazione) sia di manutenzione,
in quanto le librerie possono evolversi e richiedere aggiornamenti o modifiche
al codice.

In sintesi, l'approccio basato su best practice è più adatto per progetti piccoli
o per sviluppatori esperti che desiderano un controllo completo sul codice,
mentre l'utilizzo di librerie esterne è preferibile in contesti dove la
sicurezza, la manutenibilità e la rapidità di sviluppo sono prioritarie. In
altri ambienti invece, com'è stato evidenziato più volte in questa tesi,
potrebbe essere vantaggioso un approccio ibrido che combini le due strategie in
modo tale da massimizzare i benefici e minimizzare i rischi.