\chapter{Caso studio}
\label{cha:real_case}

\section{Presentazione}
\label{sec:presentation}

Dopo aver illustrato nei capitoli precedenti le principali vulnerabilità legate alla
gestione della memoria e le relative tecniche di mitigazione, questo capitolo
presenta un caso studio pratico, volto a dimostrare come i concetti teorici possano
essere applicati concretamente a un progetto reale.

\subsection*{Metodologia di selezione del progetto}

Per condurre un'analisi efficace delle vulnerabilità di memory safety, esistono due
alternative principali nella scelta del caso studio: utilizzare un progetto open
source esistente oppure crearne uno appositamente per scopi dimostrativi.

L'analisi di progetti open source presenta tuttavia alcune problematiche
ricorrenti:
\begin{itemize}
  \item I progetti realmente utilizzati in produzione risultano già abbastanza robusti
    e testati, rendendo poco significativa l'analisi di vulnerabilità.

  \item I progetti ``educational'' sono spesso troppo semplificati e artificiali
    per rappresentare scenari realistici di sviluppo software.
\end{itemize}

In considerazione di queste limitazioni, si è optato per la seconda alternativa:
la creazione di un nuovo progetto realistico ma deliberatamente vulnerabile. A tal
fine, è stato impiegato \textit{GitHub Copilot}\footnote{\url{https://github.com/features/copilot}}
in modalità \textit{Agent}, richiedendo esplicitamente la generazione di codice
per un'applicazione plausibile, scritta in C, che includesse vulnerabilità
comuni di memory safety a fini educativi e dimostrativi. Per trasparenza, il
prompt utilizzato è disponibile in appendice (\autoref{appendix:prompt}).

\subsection*{Descrizione del progetto}

Il progetto risultante è un \textbf{file tracker}, ovvero un'applicazione a riga
di comando che monitora una directory specificata, rilevando in tempo reale la
creazione, modifica e cancellazione di file. Il programma supporta le seguenti
funzionalità:
\begin{itemize}
  \item Filtraggio dei file mediante pattern

  \item Configurazione dell'intervallo di polling

  \item Logging su file

  \item Modalità verbosa per debug
\end{itemize}

Nonostante la sua apparente semplicità, il codice è strutturato in più moduli e include
numerosi bug intenzionali che rappresentano vulnerabilità tipiche del linguaggio
C.

Una descrizione dettagliata di ciascuna vulnerabilità è disponibile nel file \texttt{README.md}
del progetto\footnote{Progetto GitHub: \url{https://github.com/Isax03/unsafe-file-tracker}}.

\begin{quote}
  \textbf{Nota:} È importante sottolineare che, in un contesto reale di progettazione
  ex novo, molte delle vulnerabilità analizzate in questo capitolo avrebbero
  potuto essere evitate scegliendo un linguaggio memory safe già in fase di
  design (\autoref{sec:linguaggio}).

  Tuttavia, l'approccio adottato in questo caso studio è deliberatamente diverso:
  si è scelto di partire da un'applicazione scritta in C, linguaggio storicamente
  vulnerabile ma ancora ampiamente utilizzato, per illustrare concretamente l'efficacia
  (e le limitazioni) degli strumenti di analisi e delle tecniche di mitigazione
  nel contesto di un progetto esistente. Questo consente di simulare un caso realistico
  di refactoring e messa in sicurezza di software legacy, situazione frequente
  nel mondo industriale.
\end{quote}

Nel corso del capitolo vengono illustrati:
\begin{itemize}
  \item I principali difetti rilevati tramite analisi statica

  \item Tre scenari di test che scatenano il rilevamento di errori con l'analisi
    dinamica

  \item Il confronto tra l'applicazione di best practices e l'uso di librerie per
    mitigare la vulnerabilità del primo scenario
\end{itemize}

Questo caso studio rappresenta quindi una dimostrazione concreta dell'approccio
proposto nella tesi: identificare, analizzare e mitigare vulnerabilità di
memoria lungo il ciclo di vita del software, anche in assenza di exploit attivi.

\begin{quote}
  \textbf{Dettagli dell'ambiente:} Il progetto è stato analizzato, compilato ed eseguito
  su un sistema \textit{Linux} (Ubuntu 24.04.2 LTS). Di conseguenza, tutti i risultati
  presentati fanno riferimento a tale ambiente.

  Nonostante i tool utilizzati siano generalmente disponibili anche su altri sistemi
  operativi, non si garantisce il corretto funzionamento né la piena
  compatibilità su piattaforme diverse, poiché potrebbero sussistere differenze nei
  compilatori, nelle librerie di sistema o nei meccanismi di gestione della
  memoria che influenzano l'efficacia degli strumenti e la riproducibilità dei
  risultati.
\end{quote}

\section{Analisi preliminare}
\label{sec:initial_analysis}
\section{Applicazione delle mitigazioni}
\label{sec:mitigation_techniques}

Una volta identificate le vulnerabilità presenti nel progetto File Tracker, si
procede con l'applicazione delle tecniche di mitigazione illustrate nella~\autoref{sec:development}.

In questa fase vengono considerati due approcci distinti: da un lato, l'adozione
delle best practice per una gestione sicura della memoria (\autoref{sec:best-practices-codice});
dall'altro, l'impiego di librerie esterne progettate per rafforzare la sicurezza
delle operazioni su stringhe e memoria (\autoref{sec:librerie}). Entrambi gli approcci
saranno applicati al codice responsabile della vulnerabilità di buffer overflow identificata
nel primo scenario di test dell'analisi dinamica, situata nelle righe 14 e 18
del file \texttt{tracker\_core.c} (disponibile nella repository del progetto).

È importante notare che, poiché il progetto è di dimensioni contenute, molte delle
vulnerabilità potrebbero essere mitigate efficacemente tramite un refactoring mirato,
anche senza ricorrere a strumenti esterni. Tuttavia, per finalità didattiche, l'utilizzo
di librerie verrà comunque applicato e comparato con le best practice.

\subsection*{Best Practice}
\label{sec:best-practices-case-study}

Partendo dalla guida di OWASP~\cite{owasp_best_practices}, si può notare che il codice
preso in considerazione, non rispetta nessuna delle linee guida citate nella~\autoref{sec:best-practices-codice}.
In particolare:
\begin{itemize}
  \item non viene effettuato alcun controllo sulla lunghezza delle stringhe di input;

  \item le stringhe non vengono troncate correttamente nel caso in cui la lunghezza
    superi il limite previsto;

  \item non viene liberata la memoria allocata dinamicamente in caso di errore;

  \item vengono utilizzate funzioni di manipolazione delle stringhe comunemente considerate
    vulnerabili (\texttt{strcpy}).
\end{itemize}

Per mitigare la vulnerabilità di buffer overflow, si è proceduto a un
refactoring del codice, seguendo questi quattro punti, ottenendo il codice
mostrato nel branch git \texttt{mitigations/best-practices} della repository del
progetto. In particolare le modifiche consistono nell'introduzione di controlli
sulla lunghezza delle stringhe e nell'uso della variante più sicura di \texttt{strcpy},
ovvero \texttt{strncpy}, che permette di specificare il numero massimo di
caratteri da copiare; è stato inoltre forzato l'inserimento di un terminatore nullo
alla fine della stringa, per evitare che la stringa possa essere troncata senza
un terminatore, nonostante la lunghezza sia corretta; infine, sono stati scritti
controlli per gestire eventuali errori di allocazione della memoria, liberando
le risorse allocate precedentemente.

\subsection*{Librerie esterne}
\label{sec:librerie-case-study}

Dal momento che i due problemi principali del codice unsafe sono legati alla gestione
della memoria e alla manipolazione delle stringhe, si è deciso di utilizzare le
seguenti librerie:
\begin{itemize}
  \item \texttt{xmalloc}\footnote{Repository: \url{https://github.com/rosingh/xmalloc}}
    per la gestione sicura della memoria, che fornisce funzioni di allocazione
    in grado di rilevare automaticamente eventuali problemi e terminare il
    programma in caso di fallimento;

  \item \texttt{safestringlib}\footnote{Repository: https://github.com/intel/safestringlib}
    per la manipolazione sicura, offrendo funzioni di copia e concatenazione che
    gestiscono automaticamente i controlli sui limiti e la corretta terminazione.
\end{itemize}

Il procedimento di mitigazione in questo caso, è meno immediato rispetto alle best
practice, poiché non consiste nella semplice riscrittura del codice, ma richiede
l'installazione delle librerie. Nel contesto di questo esempio, la configurazione
non occupa troppo tempo ma, in progetti reali e più complessi, l'integrazione delle
librerie potrebbe richiedere uno sforzo non trascurabile.

Per quanto riguarda la libreria \texttt{xmalloc}, è sufficiente scaricare un file
header \texttt{xmalloc.h} e un file di implementazione \texttt{xmalloc.c} e includerli
nel progetto. La libreria \texttt{safestringlib} invece, richiede l'installazione
tramite alcuni comandi disponibili nella repository ufficiale.

\medskip
Oltre all'installazione, un altro aspetto da considerare, per quanto riguarda le
librerie, è la compilazione del codice: infatti ci sono librerie esterne che
richiedono l'aggiunta di flag specifici per la compilazione. In questo caso specifico,
è necessario aggiungere il seguente testo in coda ai comandi di compilazione: \begin{lstlisting}[language=bash, numbers=none]
./lib/xmalloc/xmalloc.c -lsafestring-shared
\end{lstlisting}
Questa parte include infatti, il file \texttt{xmalloc.c} (salvato nella cartella
\texttt{lib/xmalloc/}) e collega la libreria \texttt{safestringlib}, con la flag
che indica al compilatore di cercare la libreria durante la fase di linking.

Dopo aver installato le librerie, il codice è stato modificato come mostrato nel
branch git \\ \texttt{mitigations/libraries} della repository del progetto.

Le modifiche principali consistono nell'utilizzo delle funzioni \texttt{xmalloc}
e \texttt{xfree} per la gestione della memoria, che gestiscono automaticamente gli
errori di allocazione e liberazione della memoria, e nell'uso della funzione
\texttt{strncpy\_s} della libreria \texttt{safestringlib}, che permette di copiare
le stringhe in modo simile a \texttt{strncpy}, ma con controlli automatici sui limiti
e con la terminazione esplicita della stringa destinazione, restituendo codici
specifici per ogni caso di errore.

\subsection*{Test delle mitigazioni}
\label{sec:test-mitigations}

Dopo aver applicato le mitigazioni, la porzione di codice interessata è stata
nuovamente sottoposta ai test già utilizzati nell'analisi dinamica, al fine di verificare
il corretto funzionamento della funzione \texttt{tracker\_init}.

Ripetendo gli stessi comandi con AddressSanitizer e Valgrind, entrambi gli strumenti
non segnalano più anomalie: il programma si comporta come previsto e, grazie ai nuovi
controlli, eventuali errori vengono gestiti in modo sicuro, consentendo la terminazione
controllata dell'esecuzione senza blocchi o comportamenti imprevisti.

\subsection*{Confronto tra le due soluzioni}
\label{sec:comparison-case-study}

Dopo aver applicato le due soluzioni al codice, è possibile confrontarle per valutare
i pro e i contro di ciascun approccio.

Partendo dalle best practice, un aspetto sicuramente positivo è il fatto che
permette di mantenere il codice sotto il pieno controllo dello sviluppatore. Inoltre,
questo approccio non richiede l'integrazione di componenti esterni, rendendo il progetto
più leggero, facilmente portabile e privo di dipendenze aggiuntive. D'altra parte,
richiede però una maggiore attenzione e competenza da parte dello sviluppatore, poiché
è facile introdurre errori anche banali. Un altro aspetto negativo riguarda la
manutenzione: essa può risultare più onerosa nel tempo, soprattutto se non
esistono linee guida interne chiare e consolidate.

Al contrario, l'utilizzo di librerie esterne introduce una forma di astrazione:
molte delle operazioni a rischio sono delegate a componenti progettati
appositamente per gestirle in sicurezza. Questo comporta diversi vantaggi, tra
cui una maggiore robustezza e un minor rischio di vulnerabilità dovute a errori
umani. Di contro, può aumentare la complessità, sia in fase di sviluppo (configurazione,
compilazione) sia di manutenzione, in quanto le librerie possono evolversi e richiedere
aggiornamenti o modifiche al codice.

In sintesi, l'approccio basato su best practice è più adatto per progetti
piccoli o per sviluppatori esperti che desiderano un controllo completo sul
codice, mentre l'utilizzo di librerie esterne è preferibile in contesti dove la sicurezza,
la manutenibilità e la rapidità di sviluppo sono prioritarie. In altri ambienti invece,
com'è stato evidenziato più volte in questa tesi, potrebbe essere vantaggioso un
approccio ibrido che combini le due strategie in modo tale da massimizzare i benefici
e minimizzare i rischi.