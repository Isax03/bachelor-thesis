\chapter{Caso studio}
\label{chap:real_case}

\section{Presentazione}
\label{sec:presentation}

Dopo aver illustrato nei capitoli precedenti le principali vulnerabilità legate alla
gestione della memoria e le relative tecniche di mitigazione, questo capitolo
presenta un caso studio pratico volto a dimostrare come i concetti teorici
possano essere applicati concretamente a un progetto reale.

\subsection*{Metodologia di selezione del progetto}

L'obiettivo iniziale era quello di selezionare un progetto open source scritto in
C o C++, per analizzarlo e rafforzarne la memory safety mediante strumenti
statici e dinamici. Tuttavia, durante la fase di selezione, sono emersi dei problemi
ricorrenti:
\begin{itemize}
  \item I progetti realmente utilizzati in produzione risultavano già abbastanza
    robusti e testati, rendendo poco significativa l'analisi

  \item I progetti ``educational'' erano spesso troppo semplificati e artificiali
    per rappresentare scenari realistici
\end{itemize}

Per superare questo dilemma, si è optato per un approccio intermedio: la generazione
di un progetto realistico ma deliberatamente vulnerabile. A tal fine, è stato impiegato
\textit{GitHub Copilot}\footnote{\url{https://github.com/features/copilot}} in modalità
\textit{Agent}, richiedendo esplicitamente la creazione di un'applicazione plausibile,
scritta in C, che includesse vulnerabilità comuni di memory safety a fini educativi
e dimostrativi. Per trasparenza, il prompt utilizzato è disponibile in appendice
(\autoref{appendix:attachment}).

\subsection*{Descrizione del progetto}

Il progetto risultante è un \textbf{file tracker}, ovvero un'applicazione a riga
di comando che monitora una directory specificata, rilevando in tempo reale la creazione,
modifica e cancellazione di file. Il programma supporta le seguenti funzionalità:
\begin{itemize}
  \item Filtraggio dei file mediante pattern

  \item Configurazione dell'intervallo di polling

  \item Logging su file

  \item Modalità verbosa per debug
\end{itemize}

Nonostante la sua apparente semplicità, il codice è strutturato in più moduli e
include numerosi bug intenzionali che rappresentano vulnerabilità tipiche del
linguaggio C.

Nella \autoref{tab:tracker_vulnerabilities} è riportato un elenco delle
vulnerabilità identificate, con le relative posizioni nel codice.

\begin{table}[htbp]
  \centering
  \small
  \begin{tabular}{|l|l|}
    \hline
    \textbf{Bug/Vulnerabilità} & \textbf{Posizione}             \\
    \hline
    Out-of-Bounds Write        & \texttt{main.c:27,42}          \\
    \hline
    Use After Free             & \texttt{main.c:127}            \\
    \hline
    Memory Leak                & \texttt{utils.c:14-18}         \\
    \hline
    Out-of-Bounds Read         & \texttt{utils.c:56}            \\
    \hline
    Dangling Pointer           & \texttt{utils.c:72-79}         \\
    \hline
    Buffer Overflow            & \texttt{tracker\_core.c:14,18} \\
    \hline
    Double Free                & \texttt{tracker\_core.c:45}    \\
    \hline
    Dangling Pointer           & \texttt{tracker\_core.c:56}    \\
    \hline
    Out-of-Bounds Write        & \texttt{tracker\_core.c:81}    \\
    \hline
    Use After Free             & \texttt{tracker\_core.c:161}   \\
    \hline
    Out-of-Bounds Write        & \texttt{tracker\_core.c:194}   \\
    \hline
    Out-of-Bounds Write        & \texttt{tracker\_core.c:216}   \\
    \hline
  \end{tabular}
  \caption{Vulnerabilità presenti nel progetto File Tracker}
  \label{tab:tracker_vulnerabilities}
\end{table}

Una descrizione dettagliata di ciascuna vulnerabilità è disponibile nel file
\texttt{README.md} del progetto\footnote{Progetto GitHub: \url{https://github.com/Isax03/unsafe-file-tracker}}.

\begin{quote}
  \textbf{Nota:} È importante sottolineare che, in un contesto reale di progettazione
  ex novo, molte delle vulnerabilità analizzate in questo capitolo avrebbero
  potuto essere evitate scegliendo un linguaggio memory safe già in fase di
  design (\autoref{sec:linguaggio}).

  Tuttavia, l'approccio adottato in questo caso studio è deliberatamente diverso:
  si è scelto di partire da un'applicazione scritta in C, linguaggio storicamente
  vulnerabile ma ancora ampiamente utilizzato, per illustrare concretamente l'efficacia
  (e le limitazioni) degli strumenti di analisi e delle tecniche di mitigazione
  nel contesto di un progetto esistente. Questo consente di simulare un caso realistico
  di refactoring e messa in sicurezza di software legacy, situazione frequente
  nel mondo industriale.
\end{quote}

Nel corso del capitolo vengono illustrati:
\begin{itemize}
  \item I principali difetti rilevati tramite analisi statica e dinamica

  \item Le correzioni apportate per eliminare le vulnerabilità

  \item Il confronto tra l'esecuzione pre e post-mitigazione, utilizzando
    strumenti di verifica
\end{itemize}

Questo caso studio rappresenta quindi una dimostrazione concreta dell'approccio proposto
nella tesi: identificare, analizzare e mitigare vulnerabilità di memoria lungo il
ciclo di vita del software, anche in assenza di exploit attivi.

\begin{quote}
  \textbf{Dettagli dell'ambiente:} Il progetto è stato analizzato, compilato ed eseguito
  su un sistema \textit{Linux} (Ubuntu 24.04.2 LTS). Di conseguenza, tutti i risultati
  presentati fanno riferimento a tale ambiente.

  I tool utilizzati sono generalmente disponibili anche su altri sistemi operativi,
  inclusi altri ambienti Linux, macOS e in parte Windows. Tuttavia, non si garantisce
  il corretto funzionamento né la piena compatibilità su piattaforme diverse,
  poiché potrebbero sussistere differenze nei compilatori, nelle librerie di sistema
  o nei meccanismi di gestione della memoria che influenzano l'efficacia degli strumenti
  e la riproducibilità dei risultati.
\end{quote}

\section{Analisi preliminare}
\label{sec:initial_analysis}
\section{Applicazione delle tecniche di mitigazione}
\label{sec:mitigation_techniques}
\section{Confronto prima/dopo}
\label{sec:comparison}