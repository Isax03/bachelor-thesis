\chapter{Caso studio}
\label{chap:real_case}

\section{Presentazione}
\label{sec:presentation}

Dopo aver illustrato nei capitoli precedenti le principali vulnerabilità legate alla
gestione della memoria e le relative tecniche di mitigazione, questo capitolo
presenta un caso studio pratico, volto a dimostrare come i concetti teorici possano
essere applicati concretamente ad un progetto reale.

L'obiettivo iniziale era quello di selezionare un progetto open source scritto in
C o C++, in modo da analizzarlo e rafforzarne la \textit{memory safety} mediante
strumenti statici e dinamici. Tuttavia, durante la fase di selezione, è emerso
un problema ricorrente: i progetti realmente utilizzati in produzione risultavano
già abbastanza robusti e testati, rendendo poco significativa l'analisi; mentre,
all'estremo opposto, i progetti "educational" erano spesso troppo semplificati e
artificiali.

Per superare questo dilemma, si è optato per un approccio intermedio: la generazione
di un progetto realistico ma deliberatamente vulnerabile. A tal fine, è stato impiegato
\textit{GitHub Copilot}\footnote{\url{https://github.com/features/copilot}} in modalità
\textit{Agent}, richiedendo esplicitamente la creazione di un'applicazione plausibile,
scritta in C, che includesse vulnerabilità comuni di \textit{memory safety} a
fini educativi e dimostrativi. Il prompt utilizzato è disponibile in appendice (\autoref{appendix:attachment}).
per garantire riproducibilità e trasparenza.

Il progetto risultante è un \textbf{file tracker}, ovvero un'applicazione a riga
di comando che monitora una directory specificata, rilevando in tempo reale la creazione,
modifica e cancellazione di file. Il programma supporta opzioni avanzate per il filtraggio
dei file, la configurazione dell'intervallo di polling, il logging su file e la modalità
verbosa.

Nonostante la sua apparente semplicità, il codice è strutturato in più moduli e
include numerosi bug intenzionali, fra cui:

\begin{itemize}
  \item \textbf{Buffer Overflow} in funzioni di inizializzazione e logging

  \item \textbf{Use After Free} e \textbf{Double Free} in diverse fasi del ciclo
    di vita del tracker

  \item \textbf{Out-of-Bounds Read/Write} nel parsing degli argomenti e nel
    confronto dei pattern

  \item \textbf{Dereferenziazione di puntatori nulli} in condizioni anomale

  \item \textbf{Memory Leak} in caso di errori intermedi
\end{itemize}

Queste vulnerabilità sono documentate direttamente nel file \texttt{README.md} del
progetto, che elenca anche gli strumenti suggeriti per l'analisi. I tool utilizzati
includono \textbf{Valgrind}, \textbf{AddressSanitizer}, \textbf{cppcheck}, \textbf{Clang
Static Analyzer}, nonché compilazioni con flag di sicurezza avanzati (\texttt{-Wall
-Wextra -Wformat-security}).

\begin{quote}
  \textbf{Nota:} È importante sottolineare che, in un contesto reale di progettazione
  ex novo, molte delle vulnerabilità analizzate in questo capitolo avrebbero
  potuto essere evitate scegliendo un linguaggio \textit{memory safe} già in fase
  di design (\autoref{sec:linguaggio}).

  Tuttavia, l'approccio adottato in questo caso studio è deliberatamente diverso:
  si è scelto di partire da un'applicazione scritta in C, linguaggio
  storicamente vulnerabile ma ancora ampiamente utilizzato, per illustrare concretamente
  l'efficacia (e le limitazioni) degli strumenti di analisi e delle tecniche di mitigazione
  nel contesto di un progetto esistente. Questo consente di simulare un caso
  realistico di refactoring e messa in sicurezza di software legacy, situazione frequente
  nel mondo industriale.
\end{quote}

Nel corso del capitolo verranno illustrati:
\begin{itemize}
  \item I principali difetti rilevati tramite analisi statica e dinamica

  \item Le correzioni apportate per eliminare le vulnerabilità

  \item Il confronto tra l'esecuzione pre e post-mitigazione, utilizzando
    strumenti di verifica

  \item Alcuni esempi di \textit{crash} osservati durante l'esecuzione, utili per
    illustrare il comportamento anomalo di programmi affetti da vulnerabilità di
    gestione della memoria
\end{itemize}

Questo caso studio rappresenta quindi una dimostrazione concreta dell'approccio proposto
nella tesi: identificare, analizzare e mitigare vulnerabilità di memoria lungo il
ciclo di vita del software, anche in assenza di exploit attivi.

\section{Analisi preliminare}
\label{sec:initial_analysis}
\section{Applicazione delle tecniche di mitigazione}
\label{sec:mitigation_techniques}
\section{Confronto prima/dopo}
\label{sec:comparison}