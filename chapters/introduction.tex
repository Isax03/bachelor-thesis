\chapter{Introduzione}
\label{cha:introduction}

La sicurezza della memoria (in inglese, \textit{memory safety}) costituisce un
principio fondamentale nel campo della programmazione e della sicurezza
informatica. Essa comprende un insieme di pratiche e tecniche volte a garantire una
gestione sicura della memoria nelle applicazioni, prevenendo errori e
vulnerabilità che potrebbero causare comportamenti anomali o compromettere la
sicurezza del sistema.

\vspace{0.5em}

Dati empirici forniti da organizzazioni come Microsoft\cite{microsoft_proactive_approach}
e Google\cite{chromium_memory_safety} rivelano che approssimativamente il 70\%
delle vulnerabilità di sicurezza nei software contemporanei è direttamente
correlato a problematiche di gestione della memoria. Questa statistica sottolinea
l'importanza dello sviluppo e dell'implementazione di strategie efficaci per mitigare
tali rischi.

\vspace{0.5em}

Nel panorama attuale, caratterizzato da minacce informatiche in continua evoluzione
e architetture software sempre più complesse, la sicurezza della memoria riveste
un ruolo di crescente rilevanza. A fronte di questo scenario, alcune agenzie di
sicurezza di Stati Uniti, Regno Unito, Canada, Australia e Nuova Zelanda hanno recentemente
pubblicato il documento \textit{The Case for Memory Safe Roadmaps}\cite{memory_safe_roadmaps},
incoraggiando i produttori di software a sviluppare e adottare roadmap per la
transizione verso linguaggi di programmazione che garantiscano la sicurezza
della memoria. Questa iniziativa, parte della campagna \textit{Secure by Design},
mira a eliminare le vulnerabilità legate alla gestione della memoria, promuovendo
un approccio proattivo e trasparente nello sviluppo di prodotti software sicuri.

\vspace{0.5em}

L'importanza della sicurezza della memoria è ulteriormente evidenziata dalla
classifica \textit{CWE Top 25 Most Dangerous Software Weaknesses}\cite{cwe_top25_2024}
del 2024, compilata da MITRE. Tra le 25 debolezze più pericolose identificate, diverse
sono attribuibili a problematiche di gestione della memoria, tra cui:

\begin{itemize}
  \item \textbf{CWE-787}: Out-of-bounds Write (posizione \#2)

  \item \textbf{CWE-125}: Out-of-bounds Read (\#6)

  \item \textbf{CWE-416}: Use After Free (\#8)

  \item \textbf{CWE-119}: Improper Restriction of Operations within the Bounds of
    a Memory Buffer (\#20)

  \item \textbf{CWE-476}: NULL Pointer Dereference (\#21)

  \item \textbf{CWE-190}: Integer Overflow or Wraparound (\#23)
\end{itemize}

Queste vulnerabilità evidenziano come la gestione insicura della memoria continui
a rappresentare una delle principali cause di rischio nei sistemi software moderni.

\vspace{0.5em}

Per tale motivo, è fondamentale che gli sviluppatori e i professionisti della sicurezza
informatica comprendano le nozioni di base della sicurezza della memoria e adottino
pratiche di programmazione sicure. Ciò include l'uso di linguaggi di
programmazione che forniscono garanzie di sicurezza della memoria, l'implementazione
di tecniche di verifica statica e dinamica del codice, e l'adozione di strumenti
di analisi e mitigazione delle vulnerabilità.