\chapter{Introduzione}
\label{cha:introduction_new}

\section{Motivazioni e obiettivo}
La sicurezza della memoria (comunemente, \textit{memory safety}) è un principio
cardine nella programmazione moderna e nella sicurezza informatica. Essa comprende
un insieme di pratiche e tecniche volte a garantire una gestione sicura della memoria
nelle applicazioni, prevenendo errori e vulnerabilità che potrebbero causare
comportamenti anomali o compromettere la sicurezza del sistema.

Stime recenti di Microsoft\cite{microsoft_proactive_approach} e Google\cite{chromium_memory_safety}
indicano che circa il 70\% delle vulnerabilità nei software moderni è collegato a
difetti di gestione della memoria. Questo dato sottolinea quanto sia cruciale
adottare strategie e strumenti in grado di mitigare tali classi di errori prima
che il codice entri in produzione.

Un'ulteriore conferma dell'importanza della memory safety si ricava dalla
classifica \textit{CWE Top 25 Most Dangerous Software Weaknesses}\cite{cwe_top25_2024}
stilata da MITRE per il 2024. Numerose posizioni sono occupate da debolezze
legate alla memory safety:
\begin{itemize}
  \item \textbf{CWE-787} Out-of-bounds Write (2° posto)

  \item \textbf{CWE-125} Out-of-bounds Read (6° posto)

  \item \textbf{CWE-416} Use After Free (8° posto)

  \item \textbf{CWE-119} Improper Restriction of Operations within the Bounds of
    a Memory Buffer (20° posto)

  \item \textbf{CWE-476} NULL Pointer Dereference (21° posto)

  \item \textbf{CWE-190} Integer Overflow or Wraparound (23° posto)
\end{itemize}
La persistenza e la gravità delle vulnerabilità legate alla gestione della memoria
ne fanno una delle principali cause sia di exploit attivi che di incidenti su scala
globale.

I seguenti esempi concreti di attacchi e incidenti legati alla memory safety evidenziano
l'importanza di intervenire per mitigare le vulnerabilità:

\begin{itemize}
  \item \textbf{WannaCry (maggio 2017)}: un attacco ransomware su scala globale ha
    sfruttato una vulnerabilità di buffer overflow nel protocollo SMBv1 di Windows,
    nota come EternalBlue (CVE-2017-0144), per propagarsi rapidamente tra sistemi
    non aggiornati. Il malware ha infettato oltre 300.000 computer in oltre 150
    Paesi, criptando i dati e richiedendo un riscatto in Bitcoin.

    Tra le vittime più colpite vi è stato il Servizio Sanitario Nazionale del Regno
    Unito (NHS), che ha dovuto cancellare migliaia di appuntamenti e deviare
    ambulanze, con un impatto stimato di circa 92 milioni di sterline. I danni
    economici globali sono stati stimati fino a 4 miliardi di dollari.\cite{wannacry_kaspersky}

  \item \textbf{CrowdStrike Falcon Sensor (luglio 2024)}: un aggiornamento difettoso
    ha causato un problema diffuso dovuto a un errore di accesso alla memoria (lettura
    out-of-bounds) che ha provocato crash di sistema su circa 8,5 milioni di dispositivi
    Windows globalmente. L'incidente ha paralizzato infrastrutture critiche in
    diversi settori come compagnie aeree, banche e strutture sanitarie, con
    danni economici stimati intorno ai 15 miliardi di dollari.\cite{crowdstrike_bug_wired}\cite{crowdstrike_bug_wired_cost}
\end{itemize}

\noindent
Questo lavoro si propone di offrire una panoramica critica delle tecniche di
mitigazione attualmente disponibili per rafforzare la memory safety, dai linguaggi
a gestione automatica della memoria agli strumenti di verifica statica e dinamica,
fornendo una guida pratica per gli sviluppatori. La tesi propone anche un caso studio
di un'applicazione scritta in C/C++ intenzionalmente bacata, su cui sono stati
applicati gli strumenti e le tecniche illustrate.

\section{Struttura del lavoro}
La tesi è divisa in quattro capitoli principali. Inizialmente, viene fornita una
panoramica generale di cosa significhi la memory safety e di quali siano le
principali vulnerabilità legate alla memoria.

Successivamente, verranno illustrate tecniche di mitigazione per rafforzare la memory
safety, dividendole tra le varie fasi del ciclo di vita del software, dalla progettazione
al mantenimento.

Il terzo capitolo è dedicato a un caso studio pratico, in cui viene presentata
un'applicazione scritta in C/C++ con vulnerabilità di memory safety. Viene illustrato
il processo di analisi e correzione delle vulnerabilità, utilizzando strumenti e
tecniche discusse nei capitoli precedenti.

Infine, il capitolo conclusivo prevede una sintesi dei risultati ottenuti e l'esplorazione
di possibili sviluppi futuri, con l'obiettivo di migliorare ulteriormente la
memory safety nei progetti software.