\chapter{Introduzione}
\label{cha:introduction}

\section*{Motivazioni e obiettivo}
\label{sec:motivation} La sicurezza della memoria (comunemente, \textit{memory
safety}) è un principio cardine nella programmazione moderna e nella sicurezza
informatica. Essa comprende un insieme di pratiche e tecniche volte a garantire una
gestione sicura della memoria nelle applicazioni, prevenendo errori e
vulnerabilità che potrebbero causare comportamenti anomali o compromettere la
sicurezza del sistema.

Stime recenti di Microsoft~\cite{microsoft_proactive_approach} e Google~\cite{chromium_memory_safety}
indicano che circa il 70\% delle vulnerabilità nei software moderni riguardano problematiche
legate alla gestione della memoria. Questo dato sottolinea quanto sia cruciale
adottare strategie e strumenti in grado di mitigare tali classi di errori prima
che il codice entri in produzione.

Un'ulteriore conferma dell'importanza della memory safety si ricava dalla
classifica \textit{CWE Top 25 Most Dangerous Software Weaknesses}~\cite{cwe_top25_2024}
stilata da MITRE per il 2024. Numerose posizioni sono occupate da debolezze legate
alla memory safety:
\begin{itemize}
  \item \textbf{CWE-787} Out-of-bounds Write (2° posto)

  \item \textbf{CWE-125} Out-of-bounds Read (6° posto)

  \item \textbf{CWE-416} Use After Free (8° posto)

  \item \textbf{CWE-119} Improper Restriction of Operations within the Bounds of
    a Memory Buffer (20° posto)

  \item \textbf{CWE-476} NULL Pointer Dereference (21° posto)

  \item \textbf{CWE-190} Integer Overflow or Wraparound (23° posto)
\end{itemize}
La persistenza e la gravità delle vulnerabilità legate alla gestione della
memoria ne fanno una delle principali cause sia di exploit attivi che di
incidenti su scala globale.

\bigskip
\noindent
Due casi emblematici, accaduti a distanza di anni ma accomunati dalla natura della
vulnerabilità, aiutano a comprendere l'importanza critica della memory safety: uno
riguarda una debolezza strutturale del linguaggio e dell'ecosistema in cui è stato
scritto il software; l'altro evidenzia come anche in ambienti moderni, apparentemente
più sicuri, l'errore umano possa causare incidenti su scala globale.

\paragraph{WannaCry (maggio 2017)}

L'attacco ransomware WannaCry sfruttò una vulnerabilità di \textit{buffer
overflow} nel protocollo SMBv1 di Windows, nota come EternalBlue (CVE-2017-0144),
per propagarsi rapidamente tra sistemi non aggiornati. Si trattava di una tipica
debolezza legata alla gestione della memoria, comune in software scritto in C/C++,
in cui il controllo sui limiti dei buffer è lasciato completamente al
programmatore.

Il malware ha infettato oltre 300.000 computer in oltre 150 Paesi, criptando i dati
e richiedendo un riscatto in Bitcoin. Tra le vittime più colpite vi è stato il
Servizio Sanitario Nazionale del Regno Unito (NHS), che ha dovuto cancellare migliaia
di appuntamenti e deviare ambulanze, con un impatto stimato di circa 92 milioni
di sterline. I danni economici globali sono stati stimati fino a 4 miliardi di dollari.~\cite{wannacry_kaspersky}

L'attacco evidenzia come l'utilizzo di linguaggi a gestione manuale della memoria,
come C e C++, possa esporre i sistemi a vulnerabilità critiche se non vengono adottate
pratiche rigorose di sicurezza e aggiornamento.

\paragraph{CrowdStrike Falcon Sensor (luglio 2024)}

Nel luglio 2024, un aggiornamento difettoso del software di sicurezza
CrowdStrike Falcon Sensor ha causato crash di sistema su scala globale, colpendo
circa 8,5 milioni di dispositivi Windows. L'incidente è stato causato da un
errore di accesso alla memoria (lettura \textit{out-of-bounds}) nel codice,
sfuggito ai controlli e ai test prima del rilascio.

L'incidente ha paralizzato infrastrutture critiche in diversi settori come
compagnie aeree, banche e strutture sanitarie, con danni economici stimati
intorno ai 15 miliardi di dollari.~\cite{crowdstrike_bug_wired}~\cite{crowdstrike_bug_wired_cost}

Questo episodio dimostra come anche errori interni nei software di sicurezza, derivanti
da sviste umane nella gestione della memoria, possano generare conseguenze
disastrose su scala globale, evidenziando che la memory safety rimane un requisito
critico anche per i sistemi più avanzati e affidabili.

\bigskip
\noindent
Questo lavoro si propone di offrire una panoramica critica delle tecniche di mitigazione
attualmente disponibili per rafforzare la memory safety, dai linguaggi a
gestione automatica della memoria agli strumenti di verifica statica e dinamica,
fornendo una guida pratica per gli sviluppatori. La tesi propone anche un caso
studio di un'applicazione scritta in C intenzionalmente bacata, su cui sono stati
applicati gli strumenti e le tecniche illustrate.

\section*{Struttura del lavoro}
\label{sec:structure} La tesi è divisa in quattro capitoli principali.
Inizialmente, viene fornita una panoramica generale del concetto di memory safety
e di quali siano le principali vulnerabilità legate alla memoria.

Successivamente, verranno illustrate tecniche di mitigazione per rafforzare la
memory safety, dividendole tra le varie fasi del ciclo di vita del software,
dalla progettazione al mantenimento.

Il terzo capitolo è dedicato a un caso studio pratico, in cui viene presentata un'applicazione
scritta in C con vulnerabilità di memory safety. Viene illustrato il processo di
analisi e correzione delle vulnerabilità, utilizzando strumenti e tecniche discusse
nei capitoli precedenti.

Infine, il capitolo conclusivo prevede una sintesi dei risultati ottenuti e l'esplorazione
di possibili sviluppi futuri, con l'obiettivo di migliorare ulteriormente la memory
safety nei progetti software.