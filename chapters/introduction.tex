\chapter{Introduzione}
\label{cha:introduction}

\section*{Motivazioni e obiettivi}
\label{sec:motivation} La sicurezza della memoria (in inglese, \textit{memory
safety}) è un principio cardine nella programmazione moderna e nella sicurezza
informatica. Essa comprende un insieme di pratiche e tecniche volte a garantire una
gestione sicura della memoria nelle applicazioni, prevenendo errori e
vulnerabilità che potrebbero causare comportamenti anomali o compromettere la
sicurezza del sistema.

Stime di Microsoft\cite{microsoft_proactive_approach} indicano che circa il 70\%
delle vulnerabilità nei loro prodotti deriva da problemi di memory safety. Analisi
di Google\cite{google_memory_safety} riportano percentuali simili per software
come iOS e macOS. Questi dati sottolineano l'importanza di adottare strategie efficaci
per prevenire errori di gestione della memoria prima della messa in produzione.

Un'ulteriore conferma dell'importanza della memory safety si ricava dalla classifica
\textit{CWE Top 25 Most Dangerous Software Weaknesses}\cite{cwe_top25_2024} stilata
dalla MITRE Corporation\footnote{Sito: https://www.mitre.org/} per il 2024.
Numerose posizioni sono occupate da debolezze legate alla memory safety:
\begin{itemize}
  \item \textbf{CWE-787} Out-of-bounds Write (2° posto): scrittura di dati oltre
    i limiti di un buffer, un array o una struttura dati.

  \item \textbf{CWE-125} Out-of-bounds Read (6° posto): lettura di dati al di fuori
    dei limiti di un buffer, un array o una struttura dati.

  \item \textbf{CWE-416} Use After Free (8° posto): accesso a blocchi di memoria
    deallocati e contrassegnati come liberi, potenzialmente già riutilizzati e
    contenenti dati diversi da quelli precedenti.

  \item \textbf{CWE-119} Improper Restriction of Operations within the Bounds of
    a Memory Buffer (20° posto): mancato controllo dei limiti di un buffer durante
    operazioni di lettura o scrittura (può portare alle CWE-787 e CWE-125).

  \item \textbf{CWE-476} NULL Pointer Dereference (21° posto): dereferenziazione
    di un puntatore nullo.

  \item \textbf{CWE-190} Integer Overflow or Wraparound (23° posto): superamento
    dei limiti di rappresentazione di un intero; non è una vulnerabilità
    direttamente legata alla memoria, ma può causare errori di allocazione o accesso
    a memoria se non gestita correttamente.
\end{itemize}
La persistenza e la gravità delle vulnerabilità legate alla gestione della memoria
ne fanno una delle principali cause sia di exploit attivi che di incidenti su scala
globale.

\bigskip
\noindent
Due casi emblematici, accaduti a distanza di anni ma accomunati dalla natura
della vulnerabilità, aiutano a comprendere l'importanza critica della memory safety:
uno riguarda una debolezza strutturale del linguaggio e dell'ecosistema in cui è
stato scritto il software; l'altro evidenzia come anche in ambienti moderni,
apparentemente più sicuri, l'errore umano possa portare a conseguenze disastrose.

\paragraph{WannaCry (maggio 2017)}

L'attacco ransomware\footnote{Tipo di malware che cripta i file del sistema
vittima e richiede un riscatto per fornire la chiave di decrittazione} WannaCry
sfruttò una vulnerabilità di \textit{buffer overflow} nel protocollo SMBv1\footnote{Protocollo
di rete utilizzato per la condivisione di file, stampanti e altre risorse tra dispositivi
in una rete} di Windows, nota come EternalBlue (CVE-2017-0144), che corrisponde
alla debolezza CWE-787 (Out-of-bounds Write). Microsoft aveva rilasciato una patch
di sicurezza per questa vulnerabilità nel marzo 2017, ma molti sistemi rimasero
non aggiornati, permettendo al malware di propagarsi rapidamente. Si trattava di
una tipica debolezza legata alla gestione della memoria, comune in software scritto
in C/C++, in cui il controllo sui limiti dei buffer è lasciato completamente al
programmatore.

Il malware infettò oltre 300.000 computer in oltre 150 Paesi, criptando i dati e
richiedendo un riscatto in Bitcoin. Tra le vittime più colpite vi fu il Servizio
Sanitario Nazionale del Regno Unito (NHS), che dovette cancellare migliaia di appuntamenti
e deviare ambulanze, con un impatto stimato di circa 92 milioni di sterline. I danni
economici globali furono stimati fino a 4 miliardi di dollari.\cite{wannacry_kaspersky}

L'attacco evidenzia come l'utilizzo di linguaggi a gestione manuale della
memoria, come C e C++, possa esporre i sistemi a vulnerabilità critiche quando
non vengono adottate pratiche rigorose di sicurezza e aggiornamento tempestivo.

\paragraph{CrowdStrike Falcon Sensor (luglio 2024)}

Recentemente, un aggiornamento difettoso del software di sicurezza CrowdStrike Falcon
Sensor ha causato malfunzionamenti e blocchi di sistema su vasta scala, colpendo
circa 8,5 milioni di dispositivi aventi installato Windows. L'incidente è stato causato
da un errore di accesso alla memoria (lettura \textit{out-of-bounds}) nel codice,
corrispondente alla debolezza CWE-125 (Out-of-bounds Read), sfuggito ai
controlli e ai test prima del rilascio.

L'incidente ha paralizzato infrastrutture critiche in diversi settori come
compagnie aeree, banche e strutture sanitarie, con danni economici stimati
intorno ai 15 miliardi di dollari.\cite{crowdstrike_bug_wired}\cite{crowdstrike_bug_wired_cost}

Questo episodio dimostra come anche errori interni nei software, derivanti da sviste
umane nella gestione della memoria, possano generare conseguenze disastrose a
livello internazionale, evidenziando che la memory safety rimane un requisito critico
anche per i sistemi più avanzati e affidabili.

\bigskip
\noindent
I principali contributi di questo elaborato sono:
\begin{itemize}
  \item L'\textbf{analisi critica} delle tecniche di mitigazione secondo le fasi
    del SDLC, fornendo un framework operativo per introdurre la sicurezza \textit{by
    design}.

  \item La \textbf{dimostrazione pratica} dell'efficacia e delle limitazioni
    degli strumenti di analisi attraverso un caso studio replicabile su un'applicazione
    C vulnerabile.

  \item Il \textbf{confronto} tra approcci di mitigazione (best practice vs.
    librerie specializzate) con valutazione di vantaggi/svantaggi per diversi contesti
    applicativi.
\end{itemize}

\section*{Struttura del lavoro}
\label{sec:structure} La tesi è divisa in quattro capitoli principali.
Inizialmente, viene fornita una panoramica generale del concetto di memory safety
e di quali siano le principali vulnerabilità legate alla memoria (\autoref{cha:background}).

Successivamente, vengono illustrate tecniche di mitigazione per rafforzare la
memory safety, dividendole tra le varie fasi del ciclo di vita del software,
dalla progettazione al mantenimento (\autoref{cha:sdlc}).

Il terzo capitolo è dedicato a un caso studio pratico, in cui viene presentata un'applicazione
scritta in C con vulnerabilità di memory safety. Viene illustrato il processo di
analisi e correzione delle vulnerabilità, utilizzando strumenti e tecniche discusse
nei capitoli precedenti (\autoref{cha:real_case}).

Infine, il capitolo conclusivo prevede una sintesi dei risultati ottenuti e l'esplorazione
di possibili sviluppi futuri, con l'obiettivo di migliorare ulteriormente la memory
safety nei progetti software (\autoref{cha:conclusioni}).