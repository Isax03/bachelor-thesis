\chapter{Introduzione}
\label{cha:introduction}

La sicurezza della memoria (in inglese, \textit{memory safety}) costituisce un
principio fondamentale nel campo della programmazione e della sicurezza
informatica. Essa comprende un insieme di pratiche e tecniche volte a garantire una
gestione sicura della memoria nelle applicazioni, prevenendo errori e
vulnerabilità che potrebbero causare comportamenti anomali o compromettere la
sicurezza del sistema.

\vspace{0.5em}

Dati empirici forniti da organizzazioni come Microsoft\cite{microsoft_proactive_approach}
e Google\cite{chromium_memory_safety} rivelano che approssimativamente il 70\%
delle vulnerabilità di sicurezza nei software contemporanei è direttamente
correlato a problematiche di gestione della memoria. Questa statistica sottolinea
l'importanza dello sviluppo e dell'implementazione di strategie efficaci per mitigare
tali rischi.

\vspace{0.5em}

Nel panorama attuale, caratterizzato da minacce informatiche in continua evoluzione
e architetture software sempre più complesse, la sicurezza della memoria riveste
un ruolo di crescente rilevanza. Gli attacchi informatici moderni sfruttano frequentemente
vulnerabilità legate alla gestione della memoria per compromettere sistemi,
sottrarre informazioni sensibili o eseguire codice arbitrario con privilegi
elevati.

\vspace{0.5em}

Per tale motivo, è fondamentale che gli sviluppatori e i professionisti della
sicurezza informatica comprendano le nozioni di base della sicurezza della
memoria e adottino pratiche di programmazione sicure. Ciò include l'uso di linguaggi
di programmazione che forniscono garanzie di sicurezza della memoria, l'implementazione
di tecniche di verifica statica e dinamica del codice, e l'adozione di strumenti
di analisi e mitigazione delle vulnerabilità.