\chapter*{Sommario}
\label{cha:sommario}
\addcontentsline{toc}{chapter}{Sommario}

La gestione sicura della memoria rappresenta un pilastro fondamentale nella
programmazione moderna e nella cybersecurity, essenziale per prevenire vulnerabilità
ed errori che possono compromettere la stabilità e la sicurezza dei sistemi informatici.
Statistiche recenti mostrano che la maggior parte delle debolezze nelle grandi
codebase è riconducibile a problematiche di \textit{memory safety}, rendendo
questo tema una priorità per la sicurezza del software.

\vspace{0.5em}
\noindent
Questo elaborato offre una guida pratica per rafforzare la sicurezza della
memoria lungo l'intero ciclo di vita del software (SDLC). Viene innanzitutto introdotto
il concetto di memory safety, sia dal punto di vista formale che pratico, seguito
da una panoramica delle principali classi di vulnerabilità (come buffer overflow,
use-after-free, null pointer dereference, etc.) e dei casi reali più rilevanti ad
esse associati. Segue un'analisi approfondita delle tecniche di mitigazione
applicabili nelle diverse fasi dello sviluppo, con attenzione ai vantaggi,
limiti e contesti d'uso.

\vspace{0.5em}
\noindent
A supporto degli aspetti teorici, viene presentato un caso studio basato su un'applicazione
C intenzionalmente vulnerabile, analizzata mediante strumenti di analisi statica
e dinamica. Vengono inoltre messe a confronto due tecniche di mitigazione: l'adozione
di best practice e l'impiego di librerie esterne specializzate; entrambe le soluzioni
sono applicate a una vulnerabilità individuata in fase di analisi dinamica, al
fine di valutarne efficacia e impatto sul ciclo di vita del software.

\vspace{0.5em}
\noindent
Il contributo principale di questo lavoro consiste nella sistematizzazione delle
tecniche di mitigazione secondo il SDLC e nella dimostrazione pratica della loro
efficacia. I contributi di questo elaborato mirano a supportare studenti e sviluppatori
interessati a ridurre le superfici di attacco e il rischio di incidenti causati da
una gestione non sicura della memoria, fornendo indicazioni operative e
strumenti replicabili.