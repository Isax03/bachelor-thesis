\chapter*{Sommario}
\label{cha:sommario}
\addcontentsline{toc}{chapter}{Sommario}

\vspace{0.5em}

% \begin{abstract}
La \textit{Memory Safety} è una proprietà fondamentale per lo sviluppo di software
robusto e sicuro. Le vulnerabilità legate a una gestione errata della memoria, come
\textit{buffer overflow}, \textit{use-after-free} e \textit{dangling pointers},
continuano a rappresentare una delle principali cause di instabilità dei sistemi
e vettori di attacco informatico. In un panorama software sempre più complesso e
interconnesso, garantire l'integrità e la correttezza nell'accesso alla memoria
è un obiettivo cruciale per gli sviluppatori.

\vspace{0.25em}

La presente tesi parte con un'analisi teorica del concetto di memory safety,
confrontando la definizione tradizionale con una definizione formale proposta in
letteratura. Quest'ultima, pur offrendo una base rigorosa supportata da teoremi,
si fonda su assunzioni forti, quali memoria infinita e assenza di \textit{side-channel},
che ne limitano le applicazioni reali. Il confronto serve a mettere in luce come,
in contesti reali, la memory safety non sia raggiungibile in modo assoluto, ma solo
approssimabile tramite misure di mitigazione e prevenzione.

\vspace{0.25em}

Successivamente, la tesi analizza le principali categorie di vulnerabilità
memory unsafe, con esempi concreti di exploit che hanno colpito software reali.
Questa parte ha lo scopo di illustrare l'impatto concreto che errori nella
gestione della memoria possono avere in termini di sicurezza, affidabilità e costo
economico, rafforzando la motivazione all'adozione di strategie preventive.

\vspace{0.25em}

Partendo da questa consapevolezza, il lavoro prosegue con un'analisi strutturata
delle principali pratiche e strumenti pensati per ridurre il rischio di
comportamenti non memory-safe. L'analisi è organizzata lungo le fasi del \textit{Software
Development Lifecycle} (SDLC), accompagnando lo sviluppatore dalla scelta del linguaggio
di programmazione fino alla distribuzione e manutenzione del software. In
particolare, vengono discusse le principali classi di mitigazione: linguaggi di
programmazione più o meno memory-safe, analisi statica e dinamica, librerie di
supporto, pratiche di codifica sicura promosse da organizzazioni come \textsc{OWASP}
e \textsc{SEI CERT}, e tecniche avanzate come l'autenticazione dei puntatori.

\vspace{0.25em}

Il contributo principale della tesi è la costruzione di una guida pratica che permetta
agli sviluppatori di orientarsi tra le diverse soluzioni disponibili, senza
raccomandare strumenti specifici, ma descrivendone il ruolo, le potenzialità e i
limiti. L'obiettivo è fornire un riferimento utile per lo sviluppo di software sicuro
dal punto di vista della gestione della memoria, facilitando l'integrazione
della sicurezza fin dalle prime fasi del processo di progettazione.
% \end{abstract}