\chapter{Conclusioni}
\label{cha:conclusioni}

La sicurezza della memoria rappresenta una sfida continua nel campo della
programmazione e dell'ingegneria del software. Attraverso l'analisi teorica e pratica
proposta in questo lavoro, è stato mostrato come la \textit{memory safety} non costituisca
una proprietà assoluta, bensì un obiettivo da perseguire costantemente lungo l'intero
ciclo di vita del software (SDLC).

Le tecniche illustrate evidenziano che, anche partendo da codice scritto in linguaggi
privi di garanzie di sicurezza della memoria, è possibile adottare strategie e
strumenti che riducono sensibilmente il rischio di vulnerabilità, migliorando la
resilienza complessiva del sistema.

\section{Lavoro Futuro}
\label{sec:lavoro_futuro}

Questo elaborato ha cercato di fornire una guida strutturata per affrontare il
tema della \textit{memory safety}. Un'idea non sviluppata per motivi di tempo, ma
che potrebbe rappresentare uno sviluppo futuro interessante, è la realizzazione
di un sito web o di una webapp interattiva.

L'obiettivo sarebbe quello di creare uno strumento che, in base alla fase del progetto
in cui ci si trova (come design, sviluppo o mantenimento) e al tipo di software
(ad esempio applicazioni embedded o web), suggerisca in modo mirato tecniche,
strumenti e mitigazioni da applicare.

Un supporto di questo genere potrebbe essere utile per aiutare team di sviluppo
a integrare buone pratiche di sicurezza in modo semplice e adattabile al
contesto specifico. Realizzare questo tipo di strumento potrebbe rendere più accessibile
l'adozione di misure di \textit{memory safety}, anche in ambienti con meno esperienza
o risorse, colmando il divario tra teoria e pratica.