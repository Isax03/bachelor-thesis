\chapter{Esempio SEI CERT}
\label{appendix:sei_cert}

\begin{quote}
  \textbf{ARR30-C\protect\footnote{\url{https://wiki.sei.cmu.edu/confluence/display/c/ARR30-C.+Do+not+form+or+use+out-of-bounds+pointers+or+array+subscripts},
  ultimo accesso: 26 maggio 2025}}: \textit{Do not form or use out-of-bounds
  pointers or array subscripts}

  Questa regola richiede che ogni accesso ad array avvenga esclusivamente all'interno
  dei limiti dichiarati, evitando la formazione di puntatori che puntino oltre
  la fine dell'array o prima del suo inizio.

  \noindent
  Il~\autoref{lst:non-compliant} mostra un esempio di codice che viola questa
  regola, effettuando il controllo solo sul limite superiore dell'array (e non
  su quello inferiore), mentre il~\autoref{lst:compliant} mostra una versione
  conforme che verifica entrambi i limiti prima di accedere all'array.
\end{quote}


\begin{minipage}[t]{0.4\textwidth}
  \begin{lstlisting}[language=C, caption={Codice \textit{Non-Compliant}}, label={lst:non-compliant}]
enum { TABLESIZE = 100 };

static int table[TABLESIZE];

int *f(int index) {
  if (index < TABLESIZE) {
    return table + index;
  }
  return NULL;
}
\end{lstlisting}
\end{minipage}
\hfill
\begin{minipage}[t]{0.5\textwidth}
  \begin{lstlisting}[language=C, caption={Codice \textit{Compliant}}, label={lst:compliant}, style=changes_in_c]
enum { TABLESIZE = 100 };

static int table[TABLESIZE];

int *f(int index) {
  if (@index >= 0 &&@ index < TABLESIZE) {
    return table + index;
  }
  return NULL;
}
\end{lstlisting}
\end{minipage}